\documentclass[amtd, online, hvmath]{copernicus}

\begin{document}\hack{\sloppy}

\title{Calibrating airborne measurements of airspeed, pressure and temperature
using a~Doppler laser air-motion sensor}

\Author{W.~A.}{Cooper}
\Author{S.~M.}{Spuler}
\Author{M.}{Spowart}
\Author{D.~H.}{Lenschow}
\Author{R.~B.}{Friesen}

\affil{National Center for Atmospheric Research, Boulder CO, USA}

\runningtitle{Calibrations with a~laser air-motion sensor}

\runningauthor{W.~A.~Cooper et~al.}

\correspondence{W.~A.~Cooper (cooperw@ucar.edu)}


\received{1~February 2014}
\accepted{24~February 2014}
\published{}


\firstpage{1}

\maketitle


\begin{abstract}
  A~new laser air-motion sensor measures the true airspeed with an
  uncertainty of less than 0.1\,\unit{m\,s^{-1}} (standard error)
  and so reduces uncertainty in the measured component of the relative
  wind along the longitudinal axis of the aircraft to about the same
  level. The calculated pressure expected from that airspeed at the
  inlet of a~pitot tube then provides a~basis for calibrating the
  measurements of dynamic and static pressure, reducing standard-error
  uncertainty in those measurements to less than 0.3\,hPa and the
  precision applicable to steady flight conditions to about
  0.1\,hPa. These improved measurements of pressure, combined with
  high-resolution measurements of geometric altitude from the Global
  Positioning System, then indicate (via integrations of the
  hydrostatic equation during climbs and descents) that the offset and
  uncertainty in temperature measurement for one research aircraft are
  $+0.3\pm 0.3\,\unit{{\degree}C}$. For airspeed, pressure and temperature
  these are significant reductions in uncertainty vs.~those obtained
  from calibrations using standard techniques. Finally, it is shown
  that the new laser air-motion sensor, combined with parametrized
  fits to correction factors for the measured dynamic and ambient
  pressure, provides a~measurement of temperature that is independent
  of any other temperature sensor.
\end{abstract}

\introduction

Many of the core measurements made from research aircraft are
interconnected.  To measure temperature, corrections must be made for
dynamic heating caused by the motion of the aircraft; to measure
airspeed, measurements of dynamic pressure, ambient pressure and
temperature are needed; corrections are often made to the measured
pressure that depend on the airspeed and/or orientation of the
aircraft; accurate measurement of airspeed depends on knowing the
humidity of the air and so the appropriate gas constants and specific
heats; measurements of humidity by dew-point sensors must be corrected
for differences between ambient and sensor pressures; etc. There are
seldom standards or reliable references for any of these, so
uncertainty analyses involve complicated and multi-dimensional
examinations of these interactions and of how flight conditions might
influence measurements from otherwise carefully calibrated sensors.

If one could obtain a~reliable reference for any of these interlinked
measurements, it could be of great value in reducing measurement
uncertainty.  A~new instrument, a~Laser Air Motion Sensor (LAMS), now
provides such a~reference on the National Science Foundation/National
Center for Atmospheric Research (NSF/NCAR) Gulfstream GV and C-130
research aircraft (hereafter referred to as the GV and C-130). This
paper explores how measurements from that instrument can reduce
measurement uncertainties in some key measurements made on those
aircraft. The new sensor is compact and designed to be mounted in
standard instrument canisters, so the measurements and approach taken
here can be extended readily to most other research aircraft.

Calibration techniques that have been used to reduce uncertainties in
measurements of pressure include the following:
\begin{enumerate}
\item \textit{The trailing cone.} This is usually considered the best
  standard.  A~tube with inlets around its circumference is trailed so
  as to be aligned with its long axis along the airflow. A~cone is
  attached to the end of the line to keep it aligned
  along the airflow, and the sensor is trailed behind the aircraft at
  a~distance and vertical displacement sufficient to be outside
  airflow effects of the aircraft, The inlets are connected by tubing
  to sensors inside the aircraft, and the measurement so obtained is
  compared to that from the sensors being
  calibrated. \citet{IkhtiariMarth1964} and many others have described
  this system. It can be used while the aircraft airspeed, altitude
  and attitude angles are changed through the normal flight
  envelope. Disadvantages are that the system usually requires
  a~special and difficult installation, which can be particularly
  problematic for a~pressurized aircraft flying at low pressure, and
  the trailing cone is not suitable for routine measurement. When
  available, though, it provides accurate calibration;
  \citet{BrownTN1988} obtained a~pressure calibration of a~high-speed
  aircraft with standard-error uncertainty of about 0.2\,\unit{hPa},
  in ideal conditions, using a~trailing cone. (Throughout this paper,
  standard-error uncertainties will be quoted.)
\item \textit{Inter-comparisons.} Research aircraft are often flown in
  formation to collect measurements that identify differences. There
  are many published examples, but most identify differences outside
  the claimed error limits and seldom are able to determine which
  measurement is at fault.
\item \textit{Flight past towers.} Flight past high towers or tethered
  balloon sensors can provide limited checks on the accuracy of
  measured pressures, but these are only possible at low altitude and
  low airspeed so are not suited to calibration of an aircraft like
  the GV.
\item \textit{Calibration by the Global Positioning System (GPS) where
    the wind is known.} If the wind is known accurately by independent
  measurement, the drift measured by GPS can be compared to the drift
  expected in the wind measured by the aircraft, and the associated
  dynamic pressure can be corrected to minimize the difference from
  the independent measurement of wind. Examples are discussed by
  \citet{FosterCunningham2010} and by \citet{MartosEtAl2011}, where
  dynamic pressure was calibrated by comparing wind measured on the
  aircraft to that measured from a~tethered balloon. GPS measurements
  have also been used without an independent reference, with flight
  manoeuvres and Kalman filtering, to calibrate dynamic pressure
  \citep{ISI:000286931800009}.
\item \textit{Use of measurements at ports around a~sphere.}
  \citet{RodiLeon2012} showed that multiple measurements of pressure
  at ports on the surface of a~sphere can be used to determine the
  error in measured ambient pressure and, when combined with GPS
  measurements, can lead to corrections for errors introduced by
  accelerated motion of the aircraft.
\end{enumerate}
The analysis that follows demonstrates that the LAMS provides another
means of calibrating pressure, one that matches the trailing cone for
accuracy while providing \mbox{calibrations} that can be available for
routine use. The operating principles of the LAMS are discussed in the
next section. The absolute measurement of airspeed that the LAMS
provides makes it possible to deduce the expected dynamic pressure (or
the pressure increase above ambient or ``static'' pressure that occurs
when air is brought to a~stagnation point in flight) with improved
accuracy. It will be argued that this measured correction to the
dynamic pressure can then be used to improve measurements of the
ambient pressure.  Once pressure is known with small uncertainty,
temperature differences can be determined during altitude changes by
integration of the hydrostatic pressure between flight levels because
the geometric altitudes of the bounding flight levels are also known
with improved accuracy from recent improvements in measurements from
the global positioning system.  Finally, it is shown that the LAMS
provides a~direct measurement of temperature that is independent of
normal temperature sensors. This measurement should be valid during
cloud penetrations as well as in clear air. The conclusions of the
paper then will summarize how this analysis has reduced measurement
uncertainty for key state-parameter measurements from these research
aircraft.

\section{The NCAR laser air-motion sensor}

The laser air-motion sensor (LAMS) used in this study is that
described by \citet{SpulerEtAl2011}. It is a~single-beam system in
which a~laser is focused ahead of the aircraft in undisturbed air so
that, from the Doppler shift in laser light returned from ambient
aerosols, the airspeed can be measured outside the disturbed airflow
caused by the aircraft. Two different configurations were used in this
study.  In both cases, the instrument was mounted under the wing and was
aligned about 3{\degree} downward relative to the aircraft
centre line to compensate for the normal angle-of-attack. For the GV,
the focus was 30\,m ahead of the instrument, or 16\,m ahead of the
nose of the aircraft. For the C-130, the focal distance was 15\,m
ahead of the instrument. Different lens $f$~numbers were used, such that in
both cases the returned signal was dominantly from a~volume extending
about 2.5\,m along the direction of flight, as given by the
full-width-half-maximum distance of the telescope gain
pattern. A~small inertial system (Systron Donner C-MIGITS INS/GPS)
mounted in the wing pod with the LAMS measured deviations in
orientation caused by wing flex or other vibration of the pod relative
to the aircraft centre axis, where the aircraft orientation was
measured by a~separate Honeywell Laseref IV or V SM~inertial reference
system. Both provided measurements of attitude angles with
standard-error uncertainties of about 1\,\unit{mrad} and about
$0.1\,\unit{m\,s^{-1}}$ uncertainty in aircraft velocity.

Earlier versions of laser wind sensors operating at
10.6\,\unit{{\mu}m} wavelength were designed for use on NCAR aircraft
in the 1980s and 1990s, as discussed by \citet{KeelerEtAl1987},
\citet{KristensenLenschow1987} and \citet{MayorEtAl1997}, but
developments in fibre optics now have made a~much improved system
practical. For the present system, the wavelength used is about
1.56\,\unit{{\mu}m}; \citet{SpulerEtAl2011} estimated that a~particle
concentration of about $2\,\unit{cm^{-3}}$ with diameter in the range
from 0.1 to 3\,\unit{{\mu}m} is needed to provide a~detectable
signal, but the sensitivity has been improved since that early test.
Successful detection of the backscattered signal has been possible at
altitudes extending to above 13\,km, although with present sensitivity
there are still times when the signal is too small for a~valid
measurement.

The precision estimated in \citet{SpulerEtAl2011} is
0.05\,\unit{m\,s^{-1}} for 1\,s samples (as will be used in the
present analyses), although the system can provide data at much higher
rates because individual samples are recorded at 100\,\unit{Hz} after
averaging of individual spectra sampled at rates of about
200\,kHz. The light source is a~distributed feedback fibre laser
module (NKT Basik E15) with wavelength 1559.996\,nm in vacuum and
0.1\,\unit{pm\,({\degree}C)^{-1}} stability. The laser is maintained
within 1\,\unit{{\degree}C} of a~constant temperature, so wavelength
drift is below 0.001\,nm. The conversion from measured Doppler shift
to airspeed involves only the wavelength of the laser and the speed of
light, so the conversion from Doppler frequency shift to airspeed
introduces negligible error.

The peak Doppler frequency can be measured to an accuracy that,
converted to airspeed, is better than 0.1\,\unit{m\,s^{-1}}. The
precision estimate from \citet{SpulerEtAl2011} also supports an
accuracy estimate in this range if there is no bias in the selection
of the peak in the shifted frequency spectrum, as is supported by
careful examination of the recorded spectra and the operation of the
algorithm that identifies the peak.  When the signal-to-noise ratio
indicates that there is inadequate signal from which to obtain
a~Doppler shift, the measurements are flagged as missing and are not
used in the analysis to follow.

\section{Calibrating the pressure-sensing system}

\subsection{Dynamic pressure}

The most straightforward application of measurements from the LAMS is
to predict the dynamic pressure $q$. If $p$ is the ambient pressure,
$c_v$ and $c_p$ the respective specific heats of air at constant
volume or constant pressure, $T$ the absolute temperature, and
$R_{\mathrm{a}}$ the gas constant for air, the Mach number $M$ (ratio
of flight speed $v$ to the speed of sound $\sqrt{\gamma
  R_{\mathrm{a}}T}$, with $\gamma=c_p/c_v$)
is given by the following equation (cf., e.g., \citealp{NCAR_OpenSky_TECH-NOTE-000-000-000-064}):
\begin{equation}
  M=\frac{v}{\sqrt{\gamma
      R_{\mathrm{a}}T}}=\left\{\left(\frac{2c_v}{R_{\mathrm{a}}}\right)\left[\left(\frac{p+q}{p}\right)^{R_{\mathrm{a}}/c_p}-1\right]\right\}
    ^{1/2}\,.\label{eq:MachEq-1}
\end{equation}

Solving for the dynamic pressure gives
\begin{equation}
q=p\left\{\left(\frac{v^2}{2c_pT}+1\right)^{c_p/R_{\mathrm{a}}}-1\right\} \label{eq:qFromLAMS-1}
\end{equation}
which shows that, with knowledge of $p$ and $T$, LAMS (measuring $v$)
can provide an independent prediction of the dynamic pressure
$q$. Furthermore, small errors in $p$ and $T$ will have a~small effect
on the deduced dynamic pressure because expected errors are a~small
fraction of the total ambient pressure or the absolute temperature.

Corrections are usually applied to measurements of dynamic pressure on
research aircraft, including the NSF/NCAR research aircraft, so
comparing $q$ as provided by Eq.~(\ref{eq:qFromLAMS-1}) to the
uncorrected measurement of dynamic pressure $q_{\mathrm{m}}$ is an exaggeration
of the improvement that LAMS provides. Nevertheless,
Fig.~\ref{fig:QvsQLAMS} shows that the difference between the
predicted value from LAMS using Eq.~(\ref{eq:qFromLAMS-1}) and the
direct measurement (from one of the pitot tubes on the C-130
referenced to the static pressure source) is substantial and exhibits
both a~large bias and significant scatter. Applying corrections to the
direct measurement is therefore important if the air motion relative
to the aircraft is to be determined accurately.

\subsection{Ambient or static pressure}

The normal measurement of total pressure $p_{\mathrm{t}}=p+q$ is
obtained on the GV and C-130 and most other research aircraft by
measuring the pressure delivered by a~pitot tube aligned approximately
along the airflow. This measurement is made by adding two
measurements, one of ambient pressure $p$ (measured by
a~Parascientific Model 1000 absolute pressure transducer with
measurement uncertainty 0.1\,hPa, connected in parallel to static
ports on each side of the fuselage of the aircraft) and a~second of
dynamic pressure $q$ (measured by a~Honeywell PPT (0--5 PSI)
differential sensor with measurement uncertainty 0.02\,hPa, connected
between the static ports and the pressure delivered by a~pitot
tube). Two independent systems, with separate static ports, are
available on the C-130, but only one on the GV. On both, there are
also measurements from another independent system that supplies
information to the flight crew but is also recorded for research use.

Pitot tubes are generally insensitive to small deviations from normal
flow angles, typically delivering accurate total pressure within about
0.1\,{\%} for flow angles up to several degrees from the centreline of
the pitot tube (e.g.,
\citealp{NACATN2331,balachandran2006fundamentals,springerhdbk2007};
see also Appendix B). However, static ports can deliver \mbox{pressures} that
depart much more from the true ambient pressure at the flight level
when flow around the fuselage varies, and they can also produce biases
even at normal flight angles, so the largest error is expected in $p$
and consequently in $q$ while their sum $p_{\mathrm{t}}$ has
a~substantially smaller error. This was checked on the GV and on the
C-130 by comparing redundant sources for these measurements. For
example, on the NSF/NCAR C-130, there are two independent sets of
static ports and pitot tubes, and on the GV there are also redundant
sensors as part of the avionics package for the aircraft.  Comparisons
examining these independent pairs showed that the results for
$p_{\mathrm{t}}$ were remarkably consistent among all pairs (agreeing
to within 0.1\,hPa) but there was significant variability in the
redundant measurements of both $p$ and $q$, often at the few-hPa
level.

For example, Fig.~\ref{fig:Pt-comparison} compares two redundant
measurements of total pressure on the C-130, each based on a~different
pitot source and static source. This and other similar comparisons
suggest that a~good approximation is to consider $p_{\mathrm{t}}$ accurately
measured and to assume that $\Delta q$, the error in the measurement
$q_{\mathrm{m}}$ of dynamic pressure, is equal to the negative of $\Delta p$,
the error in the measurement $p_{\mathrm{m}}$ of ambient pressure, because
both arise from the ``static defect'' or error in the pressure present
at the static source:
\begin{equation}
\Delta q=q_{\mathrm{m}}-q=-\Delta p=-(p_{\mathrm{m}}-p)\,\,.\label{eq:PCOR1-1}
\end{equation}

As a~result, the correction to dynamic pressure obtained from LAMS
also provides a~correction to ambient pressure, and these corrections
can be applied simultaneously in Eq.~(\ref{eq:PCOR1-1}) using
Eq.~(\ref{eq:qFromLAMS-1}):
\begin{equation*}
\Delta q=q_{\mathrm{m}}-p\chi(v,T)
\end{equation*}
where, to simplify the notation, $\chi(v,T)$ is
\begin{equation}
\chi(v,T)=\left(\frac{v^2}{2c_pT}+1\right)^{c_p/R_{\mathrm{a}}}-1\,\,.\label{eq:ChiEquation-1}
\end{equation}
Then, because $p=p_{\mathrm{m}}-\Delta p,$
\begin{equation}
p_{\mathrm{c}}=-\Delta p=\frac{q_{\mathrm{m}}-p_{\mathrm{m}}\chi}{1+\chi}\label{eq:PCOR2-1}
\end{equation}
which gives the correction to ambient pressure $p_{\mathrm{c}}$ in
terms of the measurements of ambient and dynamic pressure, the
airspeed measured by the LAMS and the absolute temperature. The
negative sign arises because the correction needed is the negative of
the measurement error.

The temperature is needed to calculate $\chi$, but it can be assumed
tentatively that the accuracy of the temperature measurement is
adequate for this analysis. Once pressure corrections are found, the
accuracy of this assumption can be checked and the process can be
iterated as necessary. Equations~(\ref{eq:ChiEquation-1}) and
(\ref{eq:PCOR2-1}) then can be used with measurements from the LAMS to
estimate both the correction to be applied to the ambient pressure
and, with reversed sign, the correction to be applied to the dynamic
pressure.

\subsection{Some refinements}

The goal of these analyses is to determine state variables like
pressure with significantly less uncertainty than has been possible in
the past, so this objective requires attention to some minor error
sources.  Specifically, it was necessary to consider: (i)~the humidity
of the air and its effect on thermodynamic properties like the gas
constant and specific heats; (ii)~possible effects of flow angles on
the total pressure measured by the pitot tube; and (iii)~the effect of
small departures of the pointing angle of the LAMS beam from the
direction of the relative wind.
\begin{enumerate}
\item[i.] \textit{Effects of humidity.} The first was determined in
  a~straightforward way by considering the properties determined from
  weighted averages of the properties of dry air and humid air, in
  standard ways (described in Appendix A).
\item[ii.] \textit{Effect of flow not parallel to the pitot tube.}
  According to information provided by manufacturers, the typical
  sensitivity of a~pitot tube to flow direction is less than 1\,{\%}
  at flow angles up to 10{\degree} and less than 0.2\,{\%} for flow
  angles up to 5{\degree}. The error is in the direction of measuring
  too low a~total pressure as the flow angle increases, and to some
  extent it is compensated by orienting the pitot tubes along the
  average flow direction expected in normal flight. To verify that the
  pitot-tube measurements are insensitive to the variations in flow
  angles expected in normal flight, an analysis is presented in
  Appendix B based on yaw manoeuvres flown with the LAMS. The analysis
  indicates that measured total pressure remains constant, at a~level
  of about 0.1\,hPa, for flow angles from 0 to 3{\degree} from the
  centreline of the pitot tube, as is characteristic of most flight
  conditions on the GV and C-130.
\item[iii.] \textit{Effects of LAMS orientation.} Because there was
  an inertial reference system mounted on the same under-wing pylon
  that carried the LAMS, it was possible to correct for small
  departures in pointing angle relative to the aircraft reference
  line, the orientation of which was also measured by a~separate
  inertial reference system.  In addition, a~correction is needed if
  the airflow angle is not along the centreline of the pitot tube, as
  when the angle of attack differs from that represented by the
  orientation of the pitot tube. Although the pitot tube is relatively
  insensitive to flow angles and so measures the total dynamic
  pressure, LAMS measures the relative wind in a~specific
  direction. For LAMS, the effect of a~flow angle $\theta$ relative to
  the beam is that it measures $v_{\mathrm{l}}=v\,\cos(\theta)$. The beam is
  oriented close to but slightly offset from the longitudinal axis of
  the aircraft, at viewing angles $\theta_1$ above and $\theta_2$ to
  the starboard side of the longitudinal axis. Then, with side-slip
  $\beta$ positive for relative wind approaching from the starboard
  side of the aircraft, to a~sufficient approximation
  $\cos\theta=\cos(\theta_1+\alpha)\cos(\theta_2-\beta)$ with $\alpha$
  the angle of attack. The resulting equations for the pressure
  correction are then Eqs.~(\ref{eq:ChiEquation-1}) and
  (\ref{eq:PCOR2-1}) where, in Eq.~(\ref{eq:ChiEquation-1}), $v$ is
  replaced by $v_{\mathrm{l}}/\cos\theta$.
\end{enumerate}

\subsection{Uncertainty in the corrections}

When the LAMS is operating, the corrections to ambient and dynamic
pressure can be determined directly from Eqs.~(\ref{eq:PCOR1-1}) and
(\ref{eq:PCOR2-1}), and these corrections have much stronger
justification than the empirical corrections used previously. The LAMS
evaluation \citep{SpulerEtAl2011} suggests that the uncertainty in
line-of-sight $v$ is about 0.05\,\unit{m\,s^{-1}}, so this is also
approximately the uncertainty in the component of the relative wind
along the axis of LAMS. The total derivative of
Eq.~(\ref{eq:qFromLAMS-1}) provides a~basis for evaluating the
uncertainty in the value of $q$ estimated from
Eq.~(\ref{eq:qFromLAMS-1}):
\begin{equation}
\frac{\delta q}{p}=\left(\frac{v^2}{2c_pT}+1\right)^{\frac{c_p}{R_{\mathrm{a}}}-1}\frac{v^2}{R_{\mathrm{a}}T}\left(\frac{\delta v}{v}-\frac{1}{2}\frac{\Delta T}{T}\right)\,\,.\label{eq:qDerivative}
\end{equation}
The temperature uncertainty thus contributes significantly to the
uncertainty in $q$, often more than the uncertainty in $v$ from LAMS,
because $\delta v/v\approx 0.05/220 \approx 0.00023$ while for
temperature an uncertainty of 0.3\,\unit{{\degree}C} contributes
typically a~larger fractional contribution of $0.5\times 0.3/223
\approx 0.00067$ (for the GV). To reduce the uncertainty in
temperature used in Eq.~(\ref{eq:qFromLAMS-1}), the airspeed from LAMS
can be used directly in the correction for dynamic heating, avoiding
uncertainty in the conventional airspeed arising from error in
measured dynamic pressure. That is, the temperature should be
determined directly from
\begin{equation}
T=T_{\mathrm{r}}-\alpha_T\frac{v_{\mathrm{l}}^2}{2c_p}\label{eq:dynHeatingcorrUsingLAMS}
\end{equation}
with $T_{\mathrm{r}}$ the measured or ``recovery'' temperature,
$\alpha_T$ the recovery factor for the sensor measuring
$T_{\mathrm{r}}$, and $v_{\mathrm{l}}^2$ provided by LAMS rather than
the conventional solution for the Mach number determined from ambient
and dynamic pressure.

Interpreted as an uncertainty in dynamic pressure $q$, the uncertainty
in the prediction of $q$ from LAMS determined from
Eq.~(\ref{eq:qDerivative}) is typically about 0.13\,hPa (for flight at
125\,\unit{m\,s^{-1}} where the pressure is 760\,hPa and the
temperature 0\,\unit{{\degree}C}).  The uncertainty in the uncorrected
measurement of $p_{\mathrm{m}}$, from instrument characteristics, is
also about 0.1\,hPa, so using the LAMS correction yields an ambient
pressure that has an uncertainty of around 0.16\,hPa.  Evaluation at
150\,\unit{hPa} leads to a~similar estimate of uncertainty.  When LAMS
is present, it is thus possible to be confident that the measurements
of the longitudinal component of the relative wind and of the ambient
pressure have associated standard-error uncertainties of
$<0.1\,\unit{m\,s^{-1}}$ and 0.16\,hPa, respectively.

\subsection{Fits to the corrections}\label{sub:Determining-the-Corrections-1}

There is still value in determining fits to the corrections provided
by LAMS in terms of flight characteristics like flight level, angle of
attack, Mach number, etc., because then corrections can be applied in
cases where the LAMS is not present or does not detect enough signal
to provide a~valid airspeed. Such fits can be applied retrospectively
to data collected before the LAMS was available, and the fits can also
be compared to other means of estimating the corrections. A~further
reason for developing fits is that the LAMS measurement, being offset
from the nose of the aircraft, represents a~region where there may be
a~fluctuating difference in airspeed vs.~that present at the nose, and
averaging over such fluctuations as provided by functional fits
smooths the predicted corrections. Fits to the measurements can
therefore be more accurate than those corrected directly using the
LAMS airspeed $v_{\mathrm{l}}$. For these reasons, fits to the
measurements provided by Eq.~(\ref{eq:PCOR2-1}) were explored until
adequate representations of the predicted fits were found. Dependent
variables in the fits included ambient pressure, dynamic pressure,
Mach number, angle of attack, side-slip, airspeed, and other
characteristics of flight.  The following analyses use flights during
which the LAMS provided valid measurements almost continuously and
during which there were many altitude changes and speed variations.

\subsubsection{GV}

For the GV, the best representation of $\Delta p$, obtained after
trying many options, was
\begin{align}
    &\frac{\Delta p}{p}  = a_0+a_1M^2+a_2M^3
    +a_3\frac{\Delta p_{\alpha}}{\Delta
      q_{\mathrm{r}}}+a_4\left(\frac{\Delta p_{\alpha}}{\Delta
        q_{\mathrm{r}}}\right)^2+a_5\left(\frac{\Delta
        p_{\alpha}}{\Delta q_{\mathrm{r}}}\right)^3\label{eq:PCORforGV}
\end{align}
where $\Delta p_{\alpha}$ is the pressure difference between
vertically separated pressure ports on the radome (normally used to
calculate the angle of attack; cf.~\citealp{BrownFrieheLenschow1983})
and $\Delta q_{\mathrm{r}}$ is the pressure difference measured
between the centre port on the radome and the static source. The terms
involving $\Delta p_{\alpha}/\Delta q_{\mathrm{r}}$ introduce
dependence on angle-of-attack. The dimensionless coefficients $\{a_0,
a_1, a_2, a_3, a_4, a_5\}$ for the best fit to the measurements from
a~GV flight with LAMS operating were, respectively, $\{-0.0134,
0.0426, -0.0718, -0.363, -3.62, -9.7\}$, where the quoted significant
digits reflect the standard error in determining these
coefficients. In the analysis of significance for the fit, all these
coefficients were needed to represent the variance, at significance
levels less than 0.001. The correlation coefficient between the
measured pressure corrections and those predicted by
Eq.~(\ref{eq:PCORforGV}) was 0.98 and the standard error was 0.00089
(i.e., 0.089\,{\%} of the measured pressure, or about 0.3\,hPa at
a~typical ambient pressure of $p=350$\,hPa). This standard error
reflects individual measurements, for which some scatter arises
because the LAMS and pressure-sensing systems detect air parcels
slightly displaced from each other and so potentially having different
air motions. Because the fit determining the coefficients in
Eq.~(\ref{eq:PCORforGV}) is based on more than 10\,000 measurements
each characterizing one second, but with correlations among the
measurements likely extending over 10 to 100\,s, one would expect the
uncertainty in predictions from the fit to be at least 10 to 30 times
smaller than the standard error, or typically less than 0.03\,hPa. The
high correlation coefficient indicates that the fit accounts for
$>96$\,{\%} of the variance between the predicted and measured
pressure corrections. Because including additional functional
dependence terms in Eq.~(\ref{eq:PCORforGV}) did not reduce the
residual variance beyond this limit, the remaining variance likely
arises from real variance in airspeed in parcels at the radome
vs.~those in the sample volume ahead of the LAMS.

The LAMS measurements indicate that, for this set of flight
conditions, the ambient pressure should be corrected by 3.5\,hPa and
the standard deviation in that correction is 1.45\,hPa. If
Eq.~(\ref{eq:PCORforGV}) accounts for 96\,{\%} of that variance, the
remaining variance is equivalent to a~residual standard deviation of
$<0.3$\,hPa. Because most of that variance arises from turbulent regions
where the volumes sampled by the LAMS and the pressure-sensing system
can be moving differently, this can be interpreted as an upper limit
to the uncertainty in the pressure correction. Thus using the LAMS
measurement of airspeed has removed a~3.5\,hPa error and reduced the
residual uncertainty (from this source alone) to $<0.3$\,hPa.

A~concern regarding Eq.~(\ref{eq:PCORforGV}) is that, during the
flight from which this fit was determined, the variable $\Delta
p_{\alpha}/\Delta q_{\mathrm{r}}$ varied only from about $-$0.2 to
$-$0.03, while the full flight envelope of the GV spans a~larger
range. There is danger that the cubic dependence on this term in
Eq.~(\ref{eq:PCORforGV}) might extrapolate to erroneous corrections
outside that range. To guard against such errors, other fits were
developed that, although slightly less accurate, should extrapolate to
new conditions better. One example was the following:
\begin{equation}
\frac{\Delta p}{p}=a_0^{\prime}+a_1^{\prime}\frac{q_{\mathrm{m}}}{p_{\mathrm{m}}}+a_2^{\prime}M^3+a_3^{\prime}\frac{\Delta p_{\alpha}}{\Delta q_{\mathrm{r}}}\label{eq:PCOR_GV_simple}
\end{equation}
with values of the coefficients $\{a_{i, i=0-3}^{\prime}\}$
respectively \{$-$0.00071, 0.073, $-$0.0861, 0.0460\}. This fit to the
LAMS measurements accounted for 95\,{\%} of the variance, vs. 96\,{\%}
for Eq.~(\ref{eq:PCORforGV}), so it may be preferable to use
Eq.~(\ref{eq:PCOR_GV_simple}) in cases where flight conditions might
fall outside the normal range of angle of attack used to determine
Eq.~(\ref{eq:PCORforGV}).

\subsubsection{C-130}

Fits to the values of Eq.~(\ref{eq:PCOR2-1}) obtained as above were
also explored for a~C-130 flight with LAMS operating. For one pair of
measurements of ambient pressure and dynamic pressure, the best fit
with all coefficients highly significant (significance level
$<0.001$) was the following:
\begin{equation}
\frac{\Delta p}{p}=b_0+b_1\frac{\Delta p_{\alpha}}{\Delta q_{\mathrm{r}}}+b_2M+b_3M^2+b_4M^3+b_5\frac{q}{p}+b_6\frac{\Delta p_{\beta}}{\Delta q_{\mathrm{r}}}\label{eq:PCORfigQCF1col}
\end{equation}
where $\Delta p_{\beta}$ is analogous to $\Delta p_{\alpha}$ but for
the side-slip angle. The standard error for this fit was 0.00042,
corresponding to a~pressure uncertainty at 700\,hPa of about 0.3\,hPa
for the individual measurements. The second term gave the largest
reduction in residual error; using this variation alone gave
a~residual standard error of 0.00050. An adequate and simpler fit
using only the first three terms on the right side of
Eq.~(\ref{eq:PCORfigQCF1col}) gave a~residual standard error of
0.00044 or an additional error contribution to the corrected pressure
of typically 0.014\,hPa, which is insignificant in comparison to other
expected error sources. The coefficients, with quoted significant
digits determined with consideration the standard errors in the fit,
are $\{b_0^{\prime},\,b_1^{\prime},\,b_2^{\prime}\}=\{0.00186, 0.0202,
0.0135\}$. While the residuals from this fit are small, the mean
offset it produces is about 2\,hPa, so (as illustrated by
Fig.~\ref{fig:QvsQLAMS}) the effect on the measurements of ambient and
dynamic pressure is quite significant.

For both aircraft, direct use of the LAMS measurements can reduce the
uncertainty in measurements of ambient and dynamic pressure to around
0.15\,hPa. Even when the LAMS is not present, parametric fits to LAMS
measurements can reduce the uncertainty in pressure to less than
0.3\,hPa.

\subsection{Comparisons to other evidence}

There are several comparisons possible that can test these results.
Three are discussed in this section.

\subsubsection{Wind measurements in reverse-heading manoeuvres}

A~reverse-heading manoeuvre is one in which a~straight-and-level
flight leg is flown for a~short time (2 to 5\,min) and then the
aircraft reverses course and flies the same leg in the opposite
direction.  Usually these are flown approximately along and against
the wind direction.  A~test of the accuracy of the measurement of
dynamic pressure is that the longitudinal component of the wind should
reverse direction but have the same magnitude in reverse-heading
manoeuvres when the aircraft is flown over the same (drifting) flight
leg twice with opposite headings.  To isolate the effect of the
measurement of $q$ and hence true airspeed, the best wind component to
use is that along the axis of the aircraft, which is
$v_{\mathrm{g}}\cos\delta-v_{\mathrm{t}}$ where $v_{\mathrm{g}}$ is the ground speed of the
aircraft, $v_{\mathrm{t}}$ the true airspeed relative to the air, and
$\delta$ is the angle between the ground-speed vector and the heading
of the aircraft. The GPS system provides the ground-speed magnitude
$v_{\mathrm{g}}$ and the ground track angle $\Phi$, so $\delta= \Phi-\Psi$ where
$\Psi$ is the heading of the aircraft. Then the wind component along
the longitudinal axis of the aircraft is
\begin{equation}
v_x=v_{\mathrm{g}}\cos(\Phi-\Psi)-v_{\mathrm{t}}\label{eq:LongWindEq}
\end{equation}
where $v_{\mathrm{t}}$ is provided either directly from LAMS or from
the corrected dynamic pressure via Eq.~(\ref{eq:PCORforGV}) for the GV
or Eq.~(\ref{eq:PCORfigQCF1col}) for the C-130. The expectation is
that the longitudinal component of the wind given by
Eq.~(\ref{eq:LongWindEq}) will reverse sign between the two legs of
the reverse-heading manoeuvre. Within statistics imposed by
atmospheric fluctuations, this is then a~test of the validity of the
longitudinal component of the wind measurements.

A~GV flight with a~large number of reverse-heading manoeuvres, but
without the LAMS, was used for the test described in this section.
Table~\ref{tab:Pairs-of-reverse-heading} shows the results for 12
reverse-heading pairs of legs from this flight. The mean difference on
legs along opposing headings was $-0.12\pm0.91$\,\unit{m\,s^{-1}},
but there are two pairs of legs (marked with asterisks in the table)
that appear to be outliers such as would be expected if the wind
conditions changed between the two legs. If these are excluded, the
remainder give a~standard deviation such that the excluded legs would
be more than two standard deviations from the mean. Excluding these
two legs gives 10 legs with a~mean difference of
$-0.26\pm 0.43$\,\unit{m\,s^{-1}}, with standard error in the mean
of 0.14\,\unit{m\,s^{-1}}. This result suggests that the error in
measurement of longitudinal wind is
$-0.13\pm0.07$\,\unit{m\,s^{-1}}, which is consistent with
estimates of the uncertainty associated with the applied correction to
airspeed based on Eq.~(\ref{eq:PCOR_GV_simple}). This provides
supporting evidence that the standard-error uncertainty in the
measurement of the longitudinal component of the relative wind after
correction is about 0.1\,\unit{m\,s^{-1}}.

\subsubsection{The avionics pressure system of the GV}

The ambient pressure measurement from the avionics system on the GV is
more reliable than those on many research aircraft because the GV is
certified to fly on RVSM (reduced vertical separation minimum) levels
so the flight-deck pressure measurement has met strict Federal
Aviation Administration requirements. Appendix G to Federal Aviation
Regulations Part 91 specifies that the maximum allowable error in
altitude is 80\,ft, or about 24\,m. In the RVSM altitude range (flight
levels 290 to 410), this corresponds to a~requirement that the error
in pressure be in the range from about 0.68\,hPa (near FL410) to
1.1\,hPa (near FL290). For the GV flight used above, the mean
difference between the pressure provided by the avionics system and
that measured with correction by LAMS, for the RVSM altitude range,
was +0.36\,hPa with standard deviation 0.19\,hPa, so within the
tolerance required by RVSM standards the avionics pressure is
consistent with the measured pressure as calibrated in this study.

\subsubsection{``$d$~Value'' measurements during speed runs}

The dominant dependence in the pressure correction represented in
Eq.~(\ref{eq:PCORforGV}) is that on Mach number, so testing that
dependence is a~useful constraint on the validity of the
corrections. Repeatedly during the flight used to determine the
pressure calibration in this study, the GV was flown in level flight
moving from near its low-speed limit to near its high-speed limit. If
the pressure corrections are adequate, such manoeuvres should not
introduce perturbations into the measured pressure fields. A~stringent
test of this expected independence of Mach number is to consider the
difference between the geometric altitude and the pressure altitude,
or ``$d$~value'' (cf.~\citealp{Bellamy1945}) during the manoeuvre.
This compensates for small altitude changes of the aircraft and should
show a~continuous change not perturbed by the airspeed changes or
small altitude changes.

When the aircraft, at about 450\,hPa flight altitude, was slowed to
its minimum speed of about 0.45 Mach, there was a~clear perturbation
in the $d$~value plot during the transitions from Mach~0.67 to Mach~0.73 and back, as shown in
Fig.~\ref{fig:D-value-measurements}.  However, during the flight
segments at steady speed the various measurements of $d$~value are
consistent to within about 3\,m, a~change in $d$~value corresponding
to a~pressure change of only about 0.2\,hPa. In the higher Mach-number
range of the flight envelope, deviations were still smaller,
consistent with linear change with time to within about 0.1\,hPa.
This is an indication that the larger deviations of the fit from the
LAMS-measured values occur at the extremes of the flight envelope, and
that errors in the corrections represented by Eq.~(\ref{eq:PCORforGV})
are reduced if the aircraft remains close to its normal flight
envelope in unaccelerated flight. However, the consistency of the
trend suggests that the dependence of the correction on Mach number is
appropriate to within an uncertainty of about 0.2\,hPa.

\section{Correcting the measured airspeed}

The LAMS provides a~direct measurement of airspeed, but it is still
useful to use the pressures as determined in the preceding section to
determine airspeed by solving Eq.~(\ref{eq:qFromLAMS-1}) for $v$ as
a~function of $p$ and $q$. Because the volume in which LAMS senses the
airspeed is displaced from the nose of the aircraft, the airspeed that
it senses may differ slightly from that sensed at the radome of the
aircraft. For the GV, the difference between the airspeed measured by
LAMS and that determined from the corrected dynamic and ambient
pressures has a~standard deviation of 0.35\,\unit{m\,s^{-1}}, and
estimates based on measured turbulence levels indicate that this is
similar to the difference expected for sample locations separated by
about 16\,\unit{m}, the distance between the LAMS sensing volume and
the nose of the GV. For this reason, airspeed used to determine the
wind is better if based on the corrected pressures even when LAMS is
present.

On the GV, the mean change in true airspeed introduced by this
calibration is $-0.8\,\unit{m\,s^{-1}}$. The standard error in
determination of this offset is much smaller than the expected
uncertainty in the measurement from LAMS (which is
$<0.1$\,\unit{m\,s^{-1}}), so calibration using LAMS has removed
a~$-0.8\,\unit{m\,s^{-1}}$ error and reduced the uncertainty in this
measurement to $<0.1\,\unit{m\,s^{-1}}$.  For the C-130, the
corresponding correction is $+0.5\,\unit{m\,s^{-1}}$.  These
measurements are used along with measurements from GPS and an inertial
reference system (IRS) to determine the wind, and the GPS/IRS also
provides measurements with an uncertainty of about
$0.1\,\unit{m\,s^{-1}}$, so the calibration based on LAMS has reduced
the uncertainty in the component of the wind along the aircraft axis
to $<0.2\,\unit{m\,s^{-1}}$.

\section{Checking the calibrations of thermometers}

With accurate measurements of both pressure and geometric altitude, it
is possible to test calibrations of the temperature sensors on the
research aircraft by calculating height differences from integration
of the hydrostatic equation and comparing to measured height
differences.  The latter are provided with low uncertainty by modern
GPS measurements of geometric altitude. The improved accuracy in the
measurement of pressure provided by LAMS reduces the uncertainty in
the measurement of pressure differences and enables a~more stringent
test of the validity of the measurements of temperature.

The hydrostatic equation can be expressed in this form:
\begin{equation}
\delta p_i=-\frac{g\,p_i}{R_{\mathrm{a}}T_i}\delta z_i\label{eq:hydrostaticEq-1}
\end{equation}
where $\{p_i,\,T_i\}$ are the values of ambient pressure and
absolute temperature for the $i$th measurement and $\delta p_i$
is the change in pressure for the $i$th step, during which the geometric
altitude changes by $\delta z_i$. This equation can be rearranged
to obtain an estimate of the temperature:
\begin{equation}
T_i=-\frac{g}{R_{\mathrm{a}}}\frac{\delta z_i}{\delta\ln p_i}\,\,.\label{eq:Tfromdzdp-1}
\end{equation}
Measurement uncertainty of 0.1\,{\%} in derived temperature (i.e.,
a~typical uncertainty of 0.3\,\unit{{\degree}C}) requires at least
0.1\,{\%} precision in the measurement of $\delta z$, a~precision now
provided by differential GPS receivers (such as the NovAtel Model
OEM-4 L1/L2 Differential GPS system in use on both NCAR aircraft) for
height differences as small as 100\,m. The requirement is more
stringent on the measurement of pressure. In 10\,s at
10\,\unit{m\,s^{-1}} climb the pressure change is less than 10\,hPa,
and it seems likely that differences in pressure cannot be measured
confidently to better than 0.1\,hPa, so this would introduce an error
of 1\,{\%} in the deduced (absolute) temperature. This is inadequate,
so a~larger altitude difference or the average of many measurements is
required to obtain a~useful estimate of the temperature.

\subsection{C-130}

About 30\,min of flight with the LAMS on the C-130 was devoted to
repeated climbs and descents and included about 1800 measurements of
1\,s differences, so it might be expected that the standard error in
the determination of temperature from Eq.~(\ref{eq:Tfromdzdp-1}) could
be reduced by $\sqrt{1800}=42$, or to around 0.5\,\unit{{\degree}C},
by this procedure. Alternately, an appropriately weighted ``mean''
temperature between two levels can be determined from
Eq.~(\ref{eq:Tfromdzdp-1}); for this flight segment, climbs were
repeated from about 12\,000 to 16\,000\,ft, or over a~pressure range
of about 100\,hPa. An uncertainty of 0.1\,hPa in a~100\,hPa pressure
change leads to about an uncertainty of 0.1\,{\%} or, in absolute
temperature, an uncertainty of about 0.3\,\unit{{\degree}C} in the
mean temperature between the layers. It should therefore be possible
to test the temperature measurements with about this level of
confidence.

Specifically, three sums were calculated between different flight
levels:
\begin{align}
&S_1=\sum\limits_i\frac{R_{\mathrm{a},i}}{g_i}\ln\left(\frac{p_i}{p_{i-1}}\right)\label{eq:Sum1}\\
&S_2=\sum\limits_i(z_i-z_{i-1})\label{eq:Sum2}\\
&S_3=\sum\limits_i\frac{z_i-z_{i-1}}{T_{\mathrm{m},i}}\label{eq:Sum3}
\end{align}
where $R_{\mathrm{a},i}$ and $g_i$ are respectively the gas constant
(adjusted for humidity) and the acceleration of gravity (adjusted for
latitude and altitude) and $T_{\mathrm{m},i}$ is the measured
temperature in absolute units, corrected for airspeed but based on the
standard sensors being tested. The predicted mean temperature for the
layer, weighted by altitude, is given by $T_{\mathrm{p}}=-S_1/S_2$,
while the corresponding weighted-mean measured temperature is
$\overline{T}_{\mathrm{m}}=S_2/S_3$, so a~comparison of
$\overline{T}_{\mathrm{m}}$ to $T_{\mathrm{p}}$ tests the validity of
the temperature measurement.

Table~\ref{tab:C-130-climbs} shows some measurements from selected
flight legs of the C-130. The evidence from these climbs indicates
that the measured temperature was about 0.5\,\unit{{\degree}C} too
high and that the offset perhaps increases as the temperature
decreases. After this result was obtained, an investigation discovered
an error of about this magnitude in the calibration of the temperature
sensor, illustrating the value of the independent calibration provided
by the LAMS.

\subsection{GV}

A~similar approach could be taken for the GV, with promise of a~larger
range of calibration points because of the large altitude changes
present on many of the flights. However, because there have been many
flights with frequent altitude changes, it was decided instead to use
a~large data set with many climbs and descents to determine
a~polynomial correction to the temperature via minimization of the
error between actual altitude changes and those predicted from
integration of the hydrostatic equation. The chi-square ($\chi^2$) to
be minimized was
\begin{equation}
\chi^2=\sum\limits_i\frac{1}{\sigma_z^2}(h_i-Z_i)^2\label{eq:ChiSqEquation-1}
\end{equation}
where $Z_i$ is the geometric altitude measured by GPS, $\sigma_z$
is the uncertainty in the height measurement, and the predicted height
$h_i$ was determined by integration of the hydrostatic equation
in the form
\begin{align}
&h_i=h_{i-1}-\frac{R_{\mathrm{a}}(f(T_i))}{g}\ln\frac{p_i}{p_{i-1}}\label{eq:AltitudeChangeEquation-1}\\
&f(T_i)=\frac{\left(c_0+(1+c_1)T_i+T_0\right)}{1+\alpha_T\,\frac{R_{\mathrm{a}}}{2C_v}M^2}\,.\label{eq:RecoveryCorrection}
\end{align}
where $c_0$ and $c_1$ are coefficients to be found by minimization of
Eq.~(\ref{eq:ChiSqEquation-1}). In these equations, $R_{\mathrm{a}}$
is the moist-air gas constant, $g$ the acceleration of gravity
(adjusted for latitude and altitude), and $\{p_i\}$ is the time
sequence of measured pressures. The function $f(T_i)$ allows the
adjustable coefficients $c_0$ and $c_1$ to be applied to the measured
temperature $T_i$, with conversion to ambient temperature on the basis
of the recovery factor ($\alpha_T$), the Mach number ($M$)
and the specific heat at constant volume ($c_v$). The resulting
temperature is converted to an absolute temperature by the addition of
$T_0=273.15$\,\unit{K}.

Because the climbs and descents were made en route and so spanned some
horizontal distance, the vertical integration will match the pressure
change only if the atmosphere is horizontally homogeneous.  If not,
the results will be biased as the fit attempts to compensate for
horizontal gradients, and this can introduce an error into the
minimization results. To consider how serious this problem is, it is
useful to assess how a~pressure gradient will affect the results.
Suppose the horizontal pressure gradient along the flight path is
$\mathrm{d}p/\mathrm{d}s=G_p$. Then there will be a~contribution to the
pressure change arising just from the pressure gradient over a~period
$\Delta t$, of magnitude $G_pv\Delta t$, where $v$ is the
airspeed., Therefore in Eq.~(\ref{eq:AltitudeChangeEquation-1}) the
pressure ratio in the logarithmic factor must be modified to be
$(p_i-G_pv_i\Delta t)/p_{i-1}$.

It is convenient to express this in terms of $d$~value, the difference
between geometric altitude and pressure altitude, because that is
measured routinely. Part of the change in $d$~value during a~climb
results from the horizontal pressure gradient, while another part
arises from the climb in an atmosphere that differs from the standard
atmosphere.
The expected change in $d_i$, the measurement of $d$~value, is then
\begin{equation}
d_i-d_{i-1}  = -\left(\frac{R_{\mathrm{a}}f(T_i)}{g}-\frac{R_{\mathrm{s}}T_{\mathrm{s}}(p)}{g_{\mathrm{s}}}\right)\ln\frac{p_i}{p_{i-1}}-\frac{G_pR_{\mathrm{a}}T_iv_i\Delta t}{gp_i}\label{eq:DvalueIteration-1}
\end{equation}
where $R_{\mathrm{s}}$ and $g_{\mathrm{s}}$ are the gas constant and
acceleration of gravity defined in the definition of the US standard
atmosphere and $T_{\mathrm{s}}(p)$ is the absolute temperature
corresponding in the standard atmosphere to pressure $p$. The first
term on the right side arises from the climb or descent, while the
last term is the contribution from the horizontal pressure
gradient. The horizontal pressure gradient $G_p$ can then
be deduced from the measurements of $d$~value by rearranging
Eq.~(\ref{eq:DvalueIteration-1}):
\begin{equation}
G_pv_i\Delta t = \frac{gp_i}{R_{\mathrm{a}}T_i}\left\{-\left(\frac{R_{\mathrm{a}}f(T_i)}{g}-\frac{R_{\mathrm{s}}T_{\mathrm{s}}(p_i)}{g_{\mathrm{s}}}\right)\ln\frac{p_i}{p_{i-1}}-(d_i-d_{i-1})\right\}\,\,.\label{eq:GpEquation-1}
\end{equation}
Then, the altitude-change equation,
Eq.~(\ref{eq:AltitudeChangeEquation-1}),
should be replaced by
\begin{equation}
h_i=h_{i-1}-\frac{R_{\mathrm{a}}(f(T_i))}{g}\ln\left(\frac{p_i-G_pv_i\Delta t}{p_{i-1}}\right)\label{eq:AltitudeChangeModifiedForG}
\end{equation}
with $G_pv_i\Delta t$ evaluated using
Eq.~(\ref{eq:GpEquation-1}).

The measurements used were from 10 flights that comprised the fifth
circuit of the HIAPER Pole to Pole (HIPPO) experiment
\citep{wofsy2011hia}, starting and ending in Colorado, USA, but
extending north of the Arctic Circle and south to beyond New
Zealand. The flight patterns featured repeated climbs and descents to
measure profiles through the atmosphere, so the 122 profiles measured
(many covering more than 8\,\unit{km} in altitude) provided a~good set
of measurements for this study. Several data-quality restrictions were
applied to avoid periods of problematic data, notably when ice
accumulation or frozen water affected the wind-sensing system and so
the measurement of attack angle (needed for the correction to ambient
pressure). Periods with climb or descent rates less than
2\,\unit{m\,s^{-1}} were excluded, as were periods of rapid climbs
or descents exceeding 7.5\,\unit{m\,s^{-1}}. Flight periods with
airspeed less than 130\,\unit{m\,s^{-1}} were also excluded to avoid
times when the flaps might have been deployed, potentially affecting
the pressure measurements. With these exclusions, the data set
consisted of about 26\,000 samples during climbs and descents.

For measurements made at a~rate of 1\,Hz, the uncertainty $\sigma_z$
in measurement of the height difference arises primarily from the
uncertainty in the pressure change, as discussed above. The best-fit
value of $\chi^2$ as defined by Eq.~(\ref{eq:ChiSqEquation-1}) was
consistent with a~value of about 1.6\,m for $\sigma_z$, and this would
be appropriate if the uncertainty in pressure (at a~representative
altitude of about 300 to 500\,hPa) is about 0.1\,hPa, so this
uncertainty in altitude is consistent with other estimates in this
paper. The minimization was done in various ways, including evaluating
results over matrices of values of the fit parameters $c_0$ and $c_1$,
conjugate-gradient stepping, and use of the ``R'' routine
\textit{optim} \citep{Rlanguage} which implements the
\citet{NelderMead1965} minimization algorithm. All produced consistent
results, with convergence to values of
$\{c_0,\,c_1\}=\{0.32\,\unit{{\degree}C},\,0.007\}$.  This adjustment
from the measurements would change the measured total temperature,
over the course of these flights, by $+0.29\pm 0.13$\,\unit{K}, so the
fit indicates that the error in the measured temperature is within
these limits. This result applies to the measurement of total
temperature, but the minimization of Eq.~(\ref{eq:ChiSqEquation-1})
depended on the accuracy of the ambient temperature after application
of the recovery factor (using $\alpha _T = 0.988+0.053
\log_{10}M+0.090\log_{10}M^2+0.091\log_{10}M^3$), so the
\mbox{constraint} on measurement uncertainty includes uncertainty in the
recovery factor as well as the calibration of the temperature sensor
and digitization system.

The uncertainty in the determination of the fit parameters
$\{c_0,\,c_1\}$ is about $\{0.02, 0.001\}$, but the uncertainty matrix
is highly correlated such that the range of values giving an increase
in $\chi^2$ equal to the mean contribution from each point spans from
$\{0.030, 0.006\}$ to $\{0.034, 0.008\}$. Within this range, the mean
change in temperature implied by the fit remains in the range 0.28 to
0.31\,\unit{K} and the standard deviation in the correction remains
smaller than 0.15\,\unit{K}, so there is low uncertainty in the
implied adjustment needed for temperature.

A~potentially more significant source of error, however, is the effect
of measurements that for some reason are questionable or erroneous.
As discussed above, such measurements were excluded where they were
identified, but some may remain. To check on the effects of variations
in the measurements entering the minimization, the sequence of
measurements was split into five segments and fit coefficients were
determined for each. The means of these fit coefficients were $\{0.37,
0.018\}$ and when used individually to evaluate the adjustment needed
in the full data set these fits indicated an adjustment of $0.30 \pm
0.30.$ These estimates of uncertainty then indicate a~required
adjustment in temperature of about $+0.3\pm 0.3$\,\unit{{\degree}C},
with the adjusted total absolute temperature $T^{\prime}$ given in
relation to measured total temperature $T$ by
$T^{\prime}=T_0+c_0+\left(1+c_1\right)\left(T-T_0\right)$ where
$c_0=0.32\,\unit{{\degree}C}$ and $c_1=0.007$. This estimated
correction and associated uncertainty, obtained because the LAMS
provides a~calibration of the pressure-sensing system and so with GPS
enables accurate integratio of the hydrostatic equation, is
obtained independent of reference standards or inter-comparisons with
other sensors and is the best available estimate of uncertainty in the
temperature measurement from the GV.

\section{Using the LAMS to measure temperature}

As discussed above, the LAMS provides a~direct measurement of the
longitudinal component of the relative wind, $v_{\mathrm{l}}$, and
also enables corrections that improve the measurements of the ambient
and dynamic pressure. Those two pressures are sufficient to determine
the Mach number $M=v_{\mathrm{l}}/v_{\mathrm{s}}$ where
$v_{\mathrm{s}}=\sqrt{\gamma R_{\mathrm{a}}T}$ is the speed of sound
in air. An equation for temperature can be obtained
from Eq.~(\ref{eq:qFromLAMS-1}) rewritten in the form
\begin{equation}
T=\frac{v_{\mathrm{l}}^2}{2c_p\left[\left(\frac{p_{\mathrm{t}}}{p}\right)^{R_{\mathrm{a}}/c_p}-1\right]}\,\,.\label{eq:TfromVL-1}
\end{equation}
Measurements of $v_{\mathrm{l}}$, $p$ and $p_{\mathrm{t}}$ thus
determine the temperature without any reference to temperature sensors
on the aircraft.

Figure~\ref{fig:ATL} shows the measurements obtained using
Eq.~(\ref{eq:TfromVL-1}) in comparison to the primary conventional
measurement of temperature.  The mean difference (LAMS temperature
minus conventional temperature) is 0.02\,\unit{{\degree}C} and the
standard deviation is 1.1\,\unit{{\degree}C}.  The fairly large
standard deviation arises mostly from areas of significant turbulence,
where larger errors can arise because the sample volumes represented
by the dynamic pressure $q$ and the sensed airspeed $v_{\mathrm{l}}$ are
different. A~histogram of the difference shows that the central peak
is characterized by a~standard deviation of about
0.5\,\unit{{\degree}C} and extremes account for the increase to
1.1\,\unit{{\degree}C} in the full sample.

Figure~\ref{fig:TLAMSvsATX} shows a~comparison of the temperature
determined from the LAMS and that measured directly during a~portion
of a~flight of the C-130. The variance is higher in the
LAMS-determined temperature for the flight segment in the boundary
layer (near 2100Z) because the flow conditions at the pitot tube and
in the air sampled by LAMS tend to have lower coherence at high rate,
leading to a~noisier estimate of the temperature.

This new measurement of temperature is valuable as a~check on the
temperature sensors, because miscalibration or changes in the sensors
will appear as a~\mbox{discrepancy} in comparison to this
measurement. However, temperature measurement by LAMS also has a~very
useful potential application in clouds, where backscatter from the
cloud particles makes the LAMS signal very strong and where this
measurement should continue to be valid. Measurement of temperature in
cloud has been challenging because immersion sensors can become wet
and, in the dynamically heated airflow, experience wet-bulb cooling to
a~variable extent dependent on the wetting (e.g.,
\citealp{heymsJAM79,WangGeertz2009}).  If the measurement of
temperature available from LAMS remains valid in cloud, it can provide
important information on the buoyancy of clouds and would support
studies of entrainment via mixing-diagram analysis of the type
undertaken by \citet{paluch1979} or \citet{betts:1983}, which can be
compromised when using conventional temperature sensors.

Figure~\ref{fig:ATLinCirrus} illustrates the capability of the LAMS to
measure temperature in cirrus cloud. These measurements were made
during a~descent through a~cirrus layer, where the backscattered
signal was dominated by the ice crystals that were present in
concentrations varying from about 0.1\,\unit{L^{-1}} to more than
100\,\unit{L^{-1}}.  This demonstrates that the LAMS is able to
continue to operate in such conditions and that it continues to
provide a~useful temperature independent of the immersion temperature
probes.

At this time, it is less certain how the system will perform in water
clouds because almost all water clouds encountered with the LAMS have
been supercooled and heating of the window is not adequate to prevent
accumulation of ice on the window. An example of measurements in water
cloud is shown in Fig.~\ref{fig:ATLinWaterCloud}. The gap in
measurements at about 22:13:15\,UTC was caused by loss of signal as
a~result of icing on the window, but even before that the measured
temperature from LAMS was erratic and often systematically about
1\,\unit{{\degree}C} too low. More heating or data from warmer clouds
will be needed to test the potential for measurements in water
cloud. It is not yet clear that this will be a~useful measurement
because the backscattered return in dense cloud might be dominated by
regions closer than the focal point of the system, where airflow
distortion could be important.  The change in the location of the
sample volume of a~laser system was recognized by \citet{Werner:84},
and the possible error in sensed airspeed was discussed by
\citet{KeelerEtAl1987} who recommended modified processing techniques
selecting the peak rather than the mean in the Doppler-shifted
wavelength spectrum, as used here, for measurements in cloud. Indeed,
too low a~measurement of airspeed would cause the deduced temperature
to be too low, as is the trend in this figure. The small extinction
lengths shown in Fig.~\ref{fig:ATLinWaterCloud}, from 10 to 20\,m in
the more dense parts of this cloud, support this explanation. However,
the regions with erroneous temperature do not correspond to those with
short extinction length as consistently as would be expected if this
is the cause of the error in temperature.  The performance in water
clouds therefore is not yet understood and will need further
investigation and additional measurements.

\conclusions[Summary and conclusions]

A~new laser air-motion sensor, capable of measuring airspeed via the
Doppler shift in a~laser beam focused about 15 to 30\,m ahead of the
aircraft, has been used to determine corrections to be applied to the
wind component along the axis of the aircraft. With these corrections,
the uncertainty in this component of the wind has been reduced to less
than 0.1\,\unit{m\,s^{-1}}. Fits to the corrections deduced from this
system, as functions of the measurements of ambient and dynamic
pressure as well as angle of attack, support this level of accuracy
even when the LAMS system is not available. Because the basis for the
measurement is the Doppler shift in the frequency of backscattered
light, the measurement is not dependent on calibration, and because
the measurement is made well ahead of the aircraft it is unaffected by
flow distortion around the aircraft.

Once an accurate measurement of airspeed is available, the expected
pressure excess above ambient produced by that airflow at the inlet of
a~pitot tube can be calculated. The pressure at flight level can then
be determined with low uncertainty by subtracting that excess pressure
from the measured total pressure at the pitot tube. The estimated
uncertainty in that measurement is less than 0.3\,hPa, and the
precision (relevant to pressure mapping while the aircraft remains in
steady flight conditions) is about 0.1\,hPa. Calibration to this level
of precision enables improved measurement of mesoscale pressure fields
in the atmosphere, following the methods developed by
\citet{ParishEtAl2007} and \citet{ParishLeon2013} based on GPS
technology and by earlier authors including \citet{BrownEtAl1981},
\citet{ShapiroKennedy1981} and \citet{LemoneTarleton1986} on the basis
of other measurements of geometric altitude.

With accurate measurement of pressure, combined with excellent
measurements of geometric altitude from modern GPS, it is possible to
deduce constraints on the temperature measurement from integrations of
the hydrostatic equation during climbs and descents. For the GV,
a~dataset consisting of 122 extended climbs and descents, typically
through $>8$\,km, was used to determine that the measured temperature
was within about 0.3\,\unit{{\degree}C} of the values required to
minimize differences between calculated and true altitude changes. The
correction required was a~function of temperature but typically was
$+0.3\pm 0.3$\,\unit{{\degree}C}. This correction included all effects
entering the measurement of ambient temperature at flight level,
including corrections dependent on the recovery factor of the
temperature probes, which are a~significant source of uncertainty
because of the large (often 25\,\unit{{\degree}C}) corrections
required for dynamic heating at GV flight speeds.

Finally, it was shown that the LAMS, combined with parametrized fits
to correction factors for the measured dynamic and ambient pressure,
can provide a~measurement of temperature that is independent of any
other temperature sensor. That measurement continues to be valid in
all-ice clouds, but the limited measurements available in water clouds
appear less satisfactory. The latter problem is not understood but is
worth further investigation because most immersion sensors are
affected by cloud water and produce erroneously low values in water
clouds.

A~three-dimensional version of the LAMS is now under development and
will be ready for flight testing soon. That will extend the
improvements available from LAMS to all three components of the
measured wind.

\appendix

\section{Properties of moist air}

For accurate calculation in humid air, the values used for the gas
constant and specific heats should be those for moist air, although
the characteristics of dry air at the same pressure and temperature
are often used. The density of moist air having vapour pressure $e$
(and so mixing ratio $r=\epsilon e/(p-e)$ where $\epsilon$ is the
ratio of the molecular weight of water to that of dry air) is
\begin{align}
\rho_{\mathrm{a}} & =
\frac{(p-e)}{R_{\mathrm{d}}T}+\frac{e}{R_{\mathrm{w}}T}=\frac{p}{R_{\mathrm{d}}T}\left(1-\frac{e}{p}+\frac{e\epsilon}{p}\right) = \frac{p}{R_{\mathrm{d}}T}\left(1+(\epsilon-1)\frac{e}{p}\right)\label{eq:MoistAirDensity-1}
\end{align}
where $R_{\mathrm{d}}$ is the gas constant for dry air and
$R_{\mathrm{w}}$ that for water vapour, so the gas constant that will
satisfy the perfect gas equation for moist air is
\begin{equation}
R_{\mathrm{a}}=R_{\mathrm{d}}/\left[1+(\epsilon-1)\frac{e}{p}\right]\,\,.\label{eq:moistR-1}
\end{equation}

For air the specific heats $c_{p\mathrm{d}}$ and
$c_{v\mathrm{d}}$ are very close to
those for a~diatomic molecule with five degrees of freedom, while for
water the values are approximately those for six degrees of freedom
(i.e., $c_v=3R_{\mathrm{w}})$, so similar results for $c_p$, $c_v$ and
$\gamma=c_p/c_v$ for humid air are averages weighted by the
mass fraction of each constituent, as follows:
\begin{align}
&c_v  =  \frac{(p-e)R_{\mathrm{a}}}{pR_{\mathrm{d}}}\frac{5R_{\mathrm{u}}}{2M_{\mathrm{d}}}+\frac{eR_{\mathrm{a}}}{pR_{\mathrm{w}}}\frac{3R_{\mathrm{u}}}{M_{\mathrm{w}}}
  = c_{v\mathrm{d}}\frac{R_{\mathrm{a}}}{R_{\mathrm{d}}}\left(1+\left(\frac{6}{5\epsilon}-1\right)\frac{e}{p}\right)\nonumber\\
&\quad \simeq c_{v\mathrm{d}}\frac{R_{\mathrm{a}}}{R_{\mathrm{d}}}\left(1+0.92926\,\frac{e}{p}\right)\label{eq:moistcv-1}
\end{align}
\begin{align}
&c_p  =
c_{p\mathrm{d}}\frac{R_{\mathrm{a}}}{R_{\mathrm{d}}}\left(1+\left(\frac{8}{7\epsilon}-1\right)\frac{e}{p}\right)
\simeq
c_{p\mathrm{d}}\frac{R_{\mathrm{a}}}{R_{\mathrm{d}}}\left(1+0.83739\,\frac{e}{p}\right)\label{eq:moistcp-1}
\end{align}
\begin{align}
&\gamma=\gamma_{\mathrm{d}}\frac{1+\left(\frac{8}{7\epsilon}-1\right)\frac{e}{p}}{1+\left(\frac{6}{5\epsilon}-1\right)\frac{e}{p}}\simeq\gamma_{\mathrm{d}}\frac{1+0.83739\frac{e}{p}}{1+0.92926\frac{e}{p}}\label{eq:moistgamma-1}
\end{align}

In this paper, the values used for $R_{\mathrm{d}}$ and $c_{p\mathrm{d}}$ are,
respectively, 287.0653\,\unit{J\,kg^{-1}\,K^{-1}} and
1004.73\,\unit{J\,kg^{-1}\,K^{-1}}.  The preceding moist-air
equations for $R_{\mathrm{a}}$, $c_p$ and $\gamma$ were then used,
adjusted for the humidity at each 1\,Hz measurement, in the equations
of this paper. These adjustments do not differ significantly from the
approximate formulas of \citet{KhelifEtAl1999} for the same
quantities, but the equations used here are exact (within the
approximations used for the specific heats).

\section{The effect of flow angle on measured total pressure from a~pitot
tube}

Pitot tubes are designed to be insensitive to airflow conditions at
small angles from their centrelines and are assumed to deliver the
correct total pressure under such conditions. To check this, a~flight
segment with LAMS operational included yaw manoeuvres in which the
aircraft was flown in conditions of small side-slip
($<3${\degree}) in cross-controlled conditions so that the
aircraft continued in approximately the same direction and at
approximately the same airspeed. Under those conditions, one would
expect that the total pressure measured by the pitot tube would not
show a~dependence on side-slip angle.

Because a~high-accuracy test is desired, small corrections are needed
for the observed departures from steady flight speed and in altitude.
Over the course of the manoeuvre, GPS measurements of altitude were
used with the hydrostatic equation to estimate and correct for changes
in the ambient pressure using $\delta p=-(p/R_{\mathrm{a}}T)g\delta z$
where $\delta z$ is the change in altitude from the start of the
flight segment, $p$ is the ambient pressure, $R_{\mathrm{a}}$ the gas
constant for air, $T$ the absolute temperature, and $g$ the
acceleration of gravity.  In addition, a~correction was made for the
expected change in total pressure arising from small changes in
airspeed, as measured by the LAMS. This is an independent
measurement of airspeed that does not rely on the aircraft
measurements of ambient and dynamic pressure, so the correction is not
affected by possible errors in the measurement of dynamic
pressure. The correction applied is given by the following equation:
\begin{equation}
\delta q=\left(\frac{p}{R_{\mathrm{a}}T}\left(\frac{v_{\mathrm{l}}^2}{2c_pT}+1\right)^{\frac{c_p}{R_{\mathrm{a}}}-1}v_{\mathrm{l}}\right)\delta v_{\mathrm{l}}\label{eq:CorrectionInSideslipFigure}
\end{equation}
which is obtained by differentiating Eq.~(\ref{eq:qFromLAMS-1}).  In
this equation, $v_{\mathrm{l}}$ is the airspeed measured by LAMS and
the increment is referenced to the arbitrary starting value in the
time series so that corrections are made for the non-steady flight
speed during the manoeuvres.

With these corrections, the average total pressure measurements as
a~function of the magnitude of the side-slip angle are as shown in
Fig.~\ref{fig:PtvsSSLIP}. Within a~limit of about 0.1\,hPa, there is
no dependence on side-slip angle out to about 3{\degree}, a~departure
in side-slip angles and also in attack angles from the mean that is
characteristic of normal flight of both NCAR aircraft. This is
justification for neglecting possible dependence of the total pressure
measurement on flow angles, at least for the small angles
characteristic of normal flight.

\begin{acknowledgements}
  The instrument development and data collection were supported by the
  NCAR Earth Observing Laboratory. Data used in this study were
  collected during field campaigns led by S.~Wofsy et~al.~(HIPPO),
  C.~Davis et~al.~(PREDICT), and J.~Stith (IDEAS), during which the
  Research Aviation Facility pilots, mechanics, technicians, and
  software engineers operated the Gulfstream~GV and Lockheed C-130
  research aircraft.  The authors also thank Jorgen Jensen and Jeff
  Stith for comments and advice on the manuscript. The National Center
  for Atmospheric Research is sponsored by the National Science
  Foundation.
\end{acknowledgements}

\begin{thebibliography}{30}

\bibitem[{\mbox{Balachandran}(2006)}]{balachandran2006fundamentals}
Balachandran,~P.: {Fundamentals of Compressible Fluid Dynamics},
Prentice-Hall of India Pvt-Ltd, available at: \url{http://books.google.com/books?id=KEzdXmXgaHkC} (last access: 6~March~2014), 2006.


\bibitem[{Bellamy(1945)}]{Bellamy1945}
Bellamy,~J.~C.: {The use of pressure altitude and altimeter corrections in meteorology},
{J.~Meteorol.}, 2, 1--79, \doi{10.1175/1520-0469(1945)002<0001:TUOPAA>2.0.CO;2}, 1945.


\bibitem[{Betts({1983})}]{betts:1983}
Betts,~A.~K.: {Thermodynamics of mixed stratocumulus layers -- saturation point budgets},
{J. Atmos. Sci.}, {40}, {2655--2670}, \doi{10.1175/1520-0469(1983)040<2655:TOMSLS>2.0.CO;2}, {1983}.


\bibitem[{Brown(1988)}]{BrownTN1988}
Brown,~E.~N.: Position Error Calibration of a~Pressure Survey Aircraft
Using a~Trailing Cone, NCAR technical note NCAR/TN-313+STR,
Atmospheric Technology Division, NCAR, Boulder, CO, USA, available at:
\url{http://nldr.library.ucar.edu/repository/collections/TECH-NOTE-000-000-000-579} (last access: 6~March~2014), 1988.


\bibitem[{Brown et~al.({1981})Brown, Shapiro, Kennedy, and Friehe}]{BrownEtAl1981}
Brown,~E.~N., Shapiro,~M.~A., Kennedy,~P.~J., and Friehe,~C.~A.: {The application
of airborne radar altimetry to the measurement of height and slope of isobaric surfaces},
{J. Appl. Meteorol.}, {20}, {1070--1075}, \doi{10.1175/1520-0450(1981)020<1070:TAOARA>2.0.CO;2}, {1981}.


\bibitem[{Brown et~al.({1983})Brown, Friehe, and Lenschow}]{BrownFrieheLenschow1983}
Brown,~E.~N., Friehe,~C.~A., and Lenschow,~D.~H.: {The use of pressure-fluctuations
on the nose of an aircraft for measuring air motion}, {J. Clim. Appl. Meteorol.},
{22}, {171--180}, \doi{10.1175/1520-0450(1983)022<0171:TUOPFO>2.0.CO;2}, {1983}.


\bibitem[{Cho et~al.({2011})Cho, Kim, Lee, and Kee}]{ISI:000286931800009}
Cho,~A., Kim,~J., Lee,~S., and Kee,~C.: {Wind estimation and airspeed calibration
using a~UAV with a~single-antenna GPS receiver and pitot tube}, {IEEE T. Aero. Elec. Sys.}, {47}, {109--117}, \doi{10.1109/TAES.2011.5705663}, {2011}.


\bibitem[{Foster and Cunningham(2010)}]{FosterCunningham2010}
Foster,~J. and Cunningham,~K.: {A~GPS-based pitot-static calibration
  method using global output error optimization}, {Aerospace Sciences
  Meetings}, American Institute of Aeronautics and Astronautics,
\doi{10.2514/6.2010-1350}, 2010.


\bibitem[{Gracey et~al.(1951)Gracey, Letko, and Russell}]{NACATN2331}
Gracey,~W., Letko,~W., and Russell,~W.~R.: {Wind Tunnel Investigation of a~Number
 of Total-Pressure Tubes at High Angles of Attack, Subsonic Speeds}, no. 2331
 in {NACA Technical Note}, National Advisory Committee for Aeronautics, 1951.


\bibitem[{Heymsfield et~al.(1979)Heymsfield, Dye, and Biter}]{heymsJAM79}
Heymsfield,~A.~J., Dye,~J.~E., and Biter,~C.~J.: {Overestimates of entrainment
from wetting of aircraft temperature sensors in cloud}, J. Appl. Meteorol., 18,
92--95, \doi{10.1175/1520-0450(1979)018\textless0092:OOEFWO\textgreater2.0.CO;2}, 1979.


\bibitem[{Ikhtiari and Marth(1964)}]{IkhtiariMarth1964}
Ikhtiari,~P.~A. and Marth,~V.~G.: {Trailing cone static pressure measurement device}, {J. Aircraft}, 1, 93--94, \doi{10.2514/3.43563}, 1964.


\bibitem[{Keeler et~al.(1987)Keeler, Serafin, Schwiesow, Lenschow, Vaughan, and Woodfield}]{KeelerEtAl1987}
Keeler,~R.~J., Serafin,~R.~J., Schwiesow,~R.~L., Lenschow,~D.~H., Vaughan,~J.~M.,
and Woodfield,~A.: {An airborne laser air motion sensing system, Part I: Concept
and preliminary experiment}, {J. Atmos. Ocean. Tech.}, 4, 113--127, \doi{10.1175/1520-0426(1987)004<0113:AALAMS>2.0.CO;2}, 1987.


\bibitem[{Khelif et~al.({1999})Khelif, Burns, and Friehe}]{KhelifEtAl1999}
Khelif,~D., Burns,~S.~P., and Friehe,~C.~A.: {Improved wind measurements on
research aircraft}, {J. Atmos. Ocean. Tech.}, {16}, {860--875}, \doi{10.1175/1520-0426(1999)016<0860:IWMORA>2.0.CO;2}, {1999}.


\bibitem[{Kristensen and Lenschow(1987)}]{KristensenLenschow1987}
Kristensen,~L. and Lenschow,~D.~H.: {An airborne laser air motion sensing system,
Part II: Design criteria and measurement possibilities}, {J. Atmos. Ocean. Tech.}, 4, 128--138, \doi{10.1175/1520-0426(1987)004<0128:AALAMS>2.0.CO;2}, 1987.


\bibitem[{LeMone and Tarleton({1986})}]{LemoneTarleton1986}
LeMone,~M.~A. and Tarleton,~L.~F.: {The use of inertial altitude in the determination
of the convective-scale pressure field over land}, {J. Atmos. Ocean. Tech.}, {3}, {650--661}, \doi{10.1175/1520-0426(1986)003<0650:TUOIAI>2.0.CO;2}, {1986}.


\bibitem[{Lenschow(1972)}]{NCAR_OpenSky_TECH-NOTE-000-000-000-064}
Lenschow,~D.~H.: {The measurement of air velocity and temperature
  using the NCAR Buffalo Aircraft Measuring System}, Tech. rep.,
available at: \url{http://nldr.library.ucar.edu/repository/collections/TECH-NOTE-000-000-000-064} (last access: 6~March~2014),
1972.


\bibitem[{Martos et~al.(2011)Martos, Kiszely, and Foster}]{MartosEtAl2011}
Martos,~B., Kiszely,~P., and Foster,~J.: {Flight test results of
  a~GPS-based pitot-static calibration method using output-error
  optimization for a~light twin-engine airplane}, in: {Guidance, Navigation, and
Control and Co-located Conferences}, American Institute of Aeronautics and Astronautics, \doi{10.2514/6.2011-6669}, 2011.


\bibitem[{Mayor et~al.({1997})Mayor, Lenschow, Schwiesow, Mann, Frush, and Simon}]{MayorEtAl1997}
Mayor,~S.~D., Lenschow,~D.~H., Schwiesow,~R.~L., Mann,~J., Frush,~C.~L., and Simon,~M.~K.:
{Validation of NCAR 10.6-\unit{{\mu}m} \chem{CO_2} Doppler lidar radial velocity measurements
and comparison with a~915-MHz profiler}, {J. Atmos. Ocean. Tech.}, {14}, {1110--1126}, \doi{10.1175/1520-0426(1997)014<1110:VONMCD>2.0.CO;2}, {1997}.


\bibitem[{Nelder and Mead({1965})}]{NelderMead1965}
Nelder,~J.~A. and Mead,~R.: {A~simplex-method for function minimization}, {Comput.~J.}, {7}, {308--313}, {1965}.


\bibitem[{Paluch({1979})}]{paluch1979}
Paluch,~I.~R.: {Entrainment mechanism in Colorado cumuli}, {J. Atmos. Sci.}, {36}, {2467--2478}, \doi{{10.1175/1520-0469(1979)036\textless2467:TEMICC\textgreater2.0.CO;2}}, {1979}.


\bibitem[{Parish and Leon({2013})}]{ParishLeon2013}
Parish,~T.~R. and Leon,~D.~C.: {Measurement of cloud perturbation pressures using
an instrumented aircraft}, {J. Atmos. Ocean. Tech.}, {30}, {215--229}, \doi{10.1175/JTECH-D-12-00011.1}, {2013}.


\bibitem[{Parish et~al.({2007})Parish, Burkhart, and Rodi}]{ParishEtAl2007}
Parish,~T.~R., Burkhart,~M.~D., and Rodi,~A.~R.: {Determination of the horizontal
pressure gradient force using global positioning system on board an instrumented aircraft},
{J. Atmos. Ocean. Tech.}, {24}, {521--528}, \doi{10.1175/JTECH1986.1}, {2007}.


\bibitem[{{{R Core Team}}(2013)}]{Rlanguage}
{{R Core Team}}: {R: a~language and environment for statistical
  computing}, R Foundation for Statistical Computing, Vienna, Austria,
available at: \url{http://www.R-project.org} (last access: 6~March~2014), 2013.


\bibitem[{Rodi and Leon({2012})}]{RodiLeon2012}
 Rodi,~A.~R. and Leon,~D.~C.: Correction of static pressure on a
 research aircraft in accelerated flight using differential pressure
 measurements, Atmos. Meas. Tech., 5, 2569--2579,
 \doi{10.5194/amt-5-2569-2012}, 2012.


\bibitem[{Shapiro and Kennedy({1981})}]{ShapiroKennedy1981}
Shapiro,~M.~A. and Kennedy,~P.~J.: {Research aircraft measurements of jet-streamgeostrophic
and ageostrophic winds}, {J. Atmos. Sci.}, {38}, {2642--2652}, \doi{10.1175/1520-0469(1981)038<2642:RAMOJS>2.0.CO;2}, {1981}.


\bibitem[{Spuler et~al.({2011})Spuler, Richter, Spowart, and Rieken}]{SpulerEtAl2011}
Spuler,~S.~M., Richter,~D., Spowart,~M.~P., and Rieken,~K.: {Optical fiber-based
laser remote sensor for airborne measurement of wind velocity and turbulence}, {Appl. Optics}, {50}, {842--851}, \doi{10.1364/AO.50.000842}, {2011}.


\bibitem[{Tropez et~al.(2007)Tropez, Yarin, and Foss}]{springerhdbk2007}
Tropez,~C., Yarin,~A.~L., and Foss,~J.~F. (Eds.): {Springer Handbook of Experimental
Fluid Mechanics}, Springer, Berlin Heidelberg, \doi{10.1007/978-3-540-30299-5}, 2007.


\bibitem[{Wang and Geerts({2009})}]{WangGeertz2009}
Wang,~Y. and Geerts,~B.: {Estimating the evaporative cooling bias of an airborne
reverse flow thermometer}, {J. Atmos. Ocean. Tech.}, {26}, {3--21}, \doi{10.1175/2008JTECHA1127.1}, {2009}.


\bibitem[{Werner et~al.(1984)Werner, K{\"o}pp, and Schwiesow}]{Werner:84}
Werner,~C., K{\"o}pp,~F., and Schwiesow,~R.~L.: {Influence of clouds and fog
on LDA wind measurements}, Appl. Optics, 23, 2482--2484, \doi{10.1364/AO.23.002482}, 1984.


\bibitem[{Wofsy et~al.(2011)Wofsy, Daube, Jimenez, and {The HIPPO Science Team and Cooperating Modellers and Satellite Teams}}]{wofsy2011hia}
Wofsy,~S.~C., Daube,~B.~C., Jimenez,~R., and {the HIPPO Science Team
  and Cooperating Modellers and Satellite Teams}: {HIAPER Pole-to-Pole
  Observations (HIPPO): fine-grained, global-scale measurements of
  climatically important atmospheric gases and aerosols},
Philos. T. Roy. Soc. A, 369, 2073--2086, \doi{10.1098/rsta.2010.0313}, 2011.

\end{thebibliography}

\begin{table}[t]
  \caption{Pairs of reverse-heading manoeuvres.
    Average values for altitude, heading, and the longitudinal component
    of the wind ($v_x$) are listed. Data from the GV flight of 6 August
    2010.}
\label{tab:Pairs-of-reverse-heading}
%\scalebox{.85}[.85]
{\begin{tabular}{lrrrr}
\tophline
Time (UTC) & Altitude (m) & Heading & $v_x$ (\unit{m\,s^{-1}}) & $\Delta v_x$\\
\middlehline
17:38:45--17:39:45$^{*}$ & 8835    & 240 & $-$12.305 &          \\
17:42:30--17:43:30$^{*}$ & 8832    & 57  & 14.678    & 2.373    \\
17:46:30--17:47:30       & 10\,060 & 240 & $-$23.930 &          \\
17:50:30--17:51:30       & 10\,070 & 57  & 23.433    & $-$0.497 \\
17:55:00--17:56:00       & 10\,980 & 241 & $-$18.904 &          \\
17:59:00--18:00:00       & 10\,980 & 57  & 18.520    & $-$0.384 \\
18:03:00--18:04:00       & 11\,900 & 240 & $-$26.330 &          \\
18:07:15--18:08:15       & 11\,900 & 57  & 26.309    & 0.022    \\
18:17:00--18:19:00       & 12\,830 & 59  & 19.993    &          \\
18:28:00--18:30:00       & 12\,820 & 239 & $-$19.542 & 0.451    \\
18:31:00--18:33:00$^{*}$ & 12\,810 & 239 & $-$19.948 &          \\
18:36:15--18:38:15$^{*}$ & 12\,810 & 59  & 18.823    & $-$1.125 \\
18:45:00--18:47:00       & 12\,810 & 239 & $-$19.294 & $-$0.471 \\
18:48:00--18:50:00       & 12\,800 & 240 & $-$19.477 &          \\
18:53:00--18:55:00       & 12\,800 & 59  & 18.711    & $-$0.766 \\
18:54:00--18:56:00       & 12\,800 & 59  & 19.015    &          \\
19:01:30--19:03:30       & 12\,800 & 239 & $-$19.147 & $-$0.131 \\
19:20:00--19:22:00       & 4227    & 10  & 1.522     &          \\
19:27:00--19:29:00       & 4228    & 190 & $-$1.827  & $-$0.305 \\
19:28:00--19:30:00       & 4228    & 189 & $-$1.341  &          \\
19:33:00--19:35:00       & 4242    & 10  & 1.678     & 0.337    \\
19:41:00--19:43:00       & 4242    & 190 & $-$2.542  & $-$0.864 \\
\cline{1-5}
MEAN &  &  &  & $-$0.113\\
\cline{1-5}
MEAN w/0 2 largest ($^*$) &  &  &  & $-$0.261\\
\bottomhline
\end{tabular}}
%\hack{
%\setlength\tabularwidth{0.9\tabularwidth}
%}
%\scalebox{.7}[.7]{
\belowtable{%
%\hack{\vspace*{2mm}}
}
%}
\end{table}

\begin{table}[t]
  \caption{Comparisons of predicted and measured temperatures
    from climbs and descents. of the C-130. The segments are from flights
    RF05, RF06, and RF08 flown respectively on 7, 15, and 17 November
    2011. $T_{\mathrm{p}}$ is the predicted temperature and $\overline{T}_{\mathrm{m}}$
    is the weighted mean of the measured values of temperature, as defined
    in the text.}
\label{tab:C-130-climbs}
%\scalebox{.85}[.85]
{\begin{tabular}{lrrr}
\tophline
Flight number, times UTC & $T_{\mathrm{p}}$ [\unit{{\degree}C}] & $\overline{T}_{\mathrm{m}}$ [\unit{{\degree}C}] & $T_{\mathrm{p}}-\overline{T}_{\mathrm{m}}$\\
\middlehline
RF05, 20:58:00--21:11:00 & $-$10.98 & $-$10.37 & $-$0.5  \\
RF07, 21:25:10--21:33:00 & $-$6.36  & $-$5.89  & $-$0.47 \\
RF07, 21:25:10--21:29:00 & 2.27     & 2.42     & $-$0.15 \\
RF07, 21:29:00--21:33:00 & $-$12.85 & $-$12.15 & $-$0.70 \\
RF08, 21:45:00--21:53:00 & $-$0.9   & $-$0.5   & $-$0.4  \\
RF08, 23:37:00--23:41:30 & $-$6.5   & $-$6.3   & $-$0.4  \\
RF08, 23:45:00--23:50:00 & $-$9.4   & $-$8.8   & $-$0.6  \\
RF08, 23:56:00--24:01:00 & $-$9.5   & $-$8.4   & $-$1.1  \\
\cline{1-4}
mean offset, $T_{\mathrm{p}}-\overline{T}_{\mathrm{m}}$ &  &  & $-$0.55\\
\bottomhline
\end{tabular}}
%\hack{
%\setlength\tabularwidth{0.9\tabularwidth}
%}
%\scalebox{.7}[.7]{
\belowtable{%
%\hack{\vspace*{2mm}}
}
%}
\end{table}

\begin{figure}
\includegraphics[width=120mm]{amtd-2014-0031-f01}
\caption{The direct measurement of dynamic pressure ($q_{\mathrm{m}}$)
  on the C-130 vs. that deduced using the LAMS measurement of
  airspeed, via Eqs.~(\ref{eq:PCOR1-1}) and (\ref{eq:PCOR2-1}). All
  one-second-average points from one C-130 research flight on which
  the LAMS was tested (17 November 2011) are shown.}
\label{fig:QvsQLAMS}
\end{figure}

\begin{figure}
\includegraphics[width=110mm]{amtd-2014-0031-f02}
\caption{Measurements made at 1\,Hz during the 17 November 2011 flight
  of the C-130. All measurements are included for times when the true
  airspeed exceeded 50\,\unit{m\,s^{-1}} (to exclude a~short period
  with flaps deployed at the end of the flight).  The measurements
  plotted are the total pressure $p_{\mathrm{t}}$ measured by two
  independent systems using two different pitot tubes and sets of
  static buttons. The root-mean-square deviation from this line is
  0.1\,hPa, and the similar deviation from a~best-fit line is less
  than 0.04\,hPa.}
\label{fig:Pt-comparison}
\end{figure}

\begin{figure}
\includegraphics[width=120mm]{amtd-2014-0031-f03}
\caption{$D$~value measurements as a~function of time for a~flight
  segment at about 450\,hPa, and the corresponding values of the Mach
  number (plotted relative to the right axis). It might be expected
  that the $d$~value would change smoothly, as suggested by the solid
  red line. GV flight of 12 August 2010, Colorado USA to St. Croix,
  Virgin Islands. Gaps in data show portions omitted because the LAMS
  signal was too weak to be reliable.}
\label{fig:D-value-measurements}
\end{figure}

\begin{figure}
\includegraphics[width=120mm]{amtd-2014-0031-f04}
\caption{Temperature determined from LAMS using
  Eq.~(\ref{eq:TfromVL-1}) plotted as a~function of the corresponding
  direct measurement of temperature for the ferry flight from Colorado
  USA to St.~Croix, Virgin Islands, on 10 August 2010. Each plotted
  point represents a~measurement over one second of flight.}
\label{fig:ATL}
\end{figure}

\begin{figure}
\includegraphics[width=120mm]{amtd-2014-0031-f05}
\caption{Temperature determined from LAMS plotted with the standard
  measurement of temperature for a~flight segment from the C-130
  flight of 17 November 2011. The bottom panel shows the 1\,Hz
  measurements from LAMS; in the top panel, these have been smoothed
  by an 11\,s box average.}
\label{fig:TLAMSvsATX}
\end{figure}

\begin{figure}
\includegraphics[width=120mm]{amtd-2014-0031-f06}
\caption{Top panel: temperature determined from LAMS measurements of
  airspeed using Eq.~(\ref{eq:TfromVL-1}), compared to the temperature
  measured by a~conventional immersion temperature sensor during
  a~descent through a~cirrus cloud layer. Bottom panel: the measured
  ice concentration from a~two-dimensional cloud (2-DC) imaging
  probe.}
\label{fig:ATLinCirrus}
\end{figure}

\begin{figure}
\includegraphics[width=120mm]{amtd-2014-0031-f07}
\caption{Top panel: temperature determined by the LAMS during a~C-130
  cloud pass on 15 November 2011. The temperature measured by
  a~conventional temperature probe is also shown. Middle panel:
  extinction length or distance corresponding to unity optical depth,
  determined from the measured droplet size distribution. Bottom
  panel: cloud droplet concentration measured by a~cloud droplet
  probe.}
\label{fig:ATLinWaterCloud}
\end{figure}

\appendixfigures
\appendixfigures

\begin{figure}
\includegraphics[width=120mm]{amtd-2014-0031-fB01}
\caption{The total pressure (from the sum of the ambient pressure
  measurement and the dynamic pressure measurement) on the C-130 as
  a~function of the magnitude of the side-slip angle during yaw
  manoeuvres in which sideslip angles were forced by rudder action
  while the aircraft continued on approximately a~straight-and-level
  course. The mean total pressure of 760.6\,hPa has been subtracted
  from the measurements. Error bars are standard deviations in the
  measurements for the total-pressure axis and are the range of the
  bin used in sideslip.  Corrections for deviations from a~level
  course and for small variations in airspeed have been applied, as
  discussed in the text.}
\label{fig:PtvsSSLIP}
\end{figure}

\end{document}
\endinput

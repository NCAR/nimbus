
\section{AIR CHEMISTRY MEASUREMENTS}
\begin{hangparagraphs}
\textbf{Raw Carbon Monoxide Concentration (ppb): }\textbf{\uline{CO}}\index{CO}\\
CO is the uncorrected output of the TECO model 48 CO analyzer. This
instrument measures the concentration of CO by gas filter correlation.
The optics of the version operated by the RAF have been modified to
increase the light through the absorption cell, and a zero trap has
been added that periodically removes CO from the sample air stream
to obtain an accurate zero. This permits correction for the significant
temperature-dependent drift of the zero level of the meaasurement.


\textbf{}%
\begin{minipage}[t]{1\columnwidth}%
\textbf{Carbon Monoxide Analyzer Status (V): }\textbf{\uline{CMODE}}\index{CMODE},\\
\textbf{Carbon Monoxide Baseline Zero Signal (V): }\textbf{\uline{COZRO}}\textbf{\index{COZRO}}\\
\textbf{Raw Carbon Monoxide, Baseline Corrected (V): }\textbf{\uline{COCOR\index{COCOR}}}%
\end{minipage}\\
CMODE records if the CO analyzer is supplied with air from which CO
has been removed and so is recording its zero level. When CMODE is
less than 0.2 V, the instrument is in the normal operational mode,
and when CMODE is greater than 8.0 V the instrument is in the ``zero''
mode. When measurements are processed, the zero-mode signals are represented
by a cubic spline to obtain a reference baseline for the signal (COZRO),
and this baseline is subtracted from the measured value (CO) to obtain
COCOR. This variable still jumps to zero periodically and does not
include the calibration that enters the following variable, COCAL. 

\textbf{Corrected Carbon Monoxide Concentration (ppmv): COCAL}\index{COCAL}\\
COCAL is the calibrated signal after correction for drift of the baseline
and after application of the appropriate calibration coefficients
to produce units of ppmv. The quality of the baseline fit can be judged
by examining the offset at the zero points. If there are relatively
small changes in the baseline, the zero offset will be only a few
ppbv. If there have been rapid changes in the baseline, the zero offset
can be up to 50 ppbv. The magnitude of the offset at the zero values
gives a good measure of uncertainty in the data set. The detection
limit is 10 ppbv, with an uncertainty of $\pm15\%$. At 1 Hz, data
will have considerable variability, so 10-s averaging is often useful
when the measurements are used for analysis. 

\textbf{Raw TECO Ozone Output (ppb): }\textbf{\uline{TEO3}}\index{TEO3}\\
TEO3 is the uncorrected output of the TECO 49 UV ozone analyzer. This
commercial instrument has been modified to record the temperature
and pressure inside the ozone absorption cell.

\begin{minipage}[t]{1\columnwidth}%
\textbf{Internal TECO Ozone Sampling Pressure (mb): }\textbf{\uline{TEP}}\textbf{\index{TEP},
}\textbf{\uline{TEO3P}}\index{TEO3P}\\
\textbf{Internal TECO Ozone Sampling Temperature ($^{\circ}C$): }\textbf{\uline{TET\index{TET}}}%
\end{minipage}

~~~~~~~~~TEP (or TEO3P) is the pressure inside the detection
cell of the TECO 49 UV ozone analyzer, and TET is the cell temperature.
These are used to convert the measurements from the instrument to
units of ppbv.

\textbf{Corrected TECO Ozone Concentration (ppbv): }\textbf{\uline{TEO3C}}\index{TEO3C}\\
TEO3C is the measurement from the TECO 49 UV ozone analyzer after
correction for the pressure and temperature in the cell. The instrument
provides output only each ten seconds, and measurements are collected
in the 3 s preceding the update. The measurements may be artificially
high or low when rapid changes in humidity are present, as may occur
when crossing the top of the boundary layer or when going through
clouds. In operation on the ground prior to takeoff or immediately
after landing, a high concentration of hydrocarbons can cause spuriously
high measurements. The detection limit is 1 ppbv with an uncertainty
of $\pm$5\%.

\textbf{}%
\begin{minipage}[t]{1\columnwidth}%
\textbf{NO Raw Counts (counts per sample interval): }\textbf{\uline{XNO}}\textbf{\index{XNO}}\\
\textbf{NOy Raw Counts (counts per sample interval): }\textbf{\uline{XNOY\index{XNOY}}}\\
\textbf{NO Calibration Flow (slpm): }\textbf{\uline{XNOCF}}\textbf{\index{XNOCF}}\\
\textbf{NOy Calibration Flow (slpm): }\textbf{\uline{XNCLF}}\textbf{\index{XNCLF}}\\
\textbf{NO, NOy Measurement Status (dimensionless): }\textbf{\uline{XNST}}\textbf{\index{XNST}}\\
\textbf{NO Zero Air Flow (slpm): }\textbf{\uline{XNOZA}}\textbf{\index{XNOZA}}\\
\textbf{NOy Zero Air Flow (slpm): }\textbf{\uline{XNZAF}}\textbf{\index{XNZAF}}\\
\textbf{NO Sample Flow (slpm): }\textbf{\uline{XNOSF}}\textbf{\index{XNOSF}}\\
\textbf{NOy Sample Flow (slpm): }\textbf{\uline{XNSAF}}\textbf{\index{XNSAF}}\\
\textbf{NOy Reaction Chamber Pressure (mb): }\textbf{\uline{XNOYP}}\textbf{\index{XNOYP}}\\
\textbf{Gold NOy Converter Temperature ($\text{�}$C): }\textbf{\uline{XNMBT\index{XNMBT}}}%
\end{minipage}\\
XNO and XNOY are the raw data counts from the NO and NO$_{y}$ instruments,
respectively, and XNCLF and XNOCF are the respective calibration flows
for these instruments.  XNST records the status for both instruments:
In measurement mode, XNST is 0, while XNST is 5 when the instruments
are in zero mode and 10 when the instruments are in calibraation mode.
the NOy and NO instruments. The instrument is in the measure mode
for XNST of 0. For a XNST reading of 5 the instruments are in the
zero mode. XNST value of 10 is the calibration mode. XNOZA and XNZAF
are flow rates for zero air used to back flush inlets, typically at
takeoff and landing, and for calibration using ``zero'' air. Even
if the status, XNST, is 0, indicating the instrument is in the measurement
mode, when XNOZA and XNZAF are approximately 1 slpm the instrument
is measuring zero air and not ambient air.  XNOSF and XNSAF are the
sample flow rates through the NO and NO$_{y}$ instruments respectively.
These values are typically about 1 slpm. XNMBT is the temperature
of the gold NO$_{y}$ converter.

\begin{minipage}[t]{1\columnwidth}%
C\textbf{orrected NO Concentration (ppbv): }\textbf{\uline{XNOCAL}}\textbf{\index{XNOCAL}}\\
\textbf{Corrected NOy Concentration (ppbv): }\textbf{\uline{XNYCAL\index{XNYCAL}}}%
\end{minipage}\\
XNOCAL and XNYCAL are the calibrated NO and NO$_{y}$ concentrations,
respectively, with units of ppbv. The NO and NOy data are represented
by a cubic spline for baseline subtraction, and then the calibration
coefficients are applied and the measurements are converted to units
of ppbv. The quality of the data can be assessed by examining the
accuracy of the zero correction. This instrument  adds water vapor
to the sample stream to reduce the effect of ambient water on the
final signal. The water vapor addition is not sufficient to saturate
the sample stream, but enough to remove much of the interference.
The detection limits of the NO,NO$_{y}$ instruments are 50 ppbv for
a one-second averaging time. The uncertainty is $\pm$ 5\%.

\textbf{Raw Chemiluminescent Ozone Signal (V): }\textbf{\uline{O3FS}}\index{O3FS}\\
Raw output from the reverse chemiluminescence ozone instrument, which
operates on the basis of reacting nitric oxide with ozone and detecting
the resulting chemiluminescence.

\begin{minipage}[t]{1\columnwidth}%
\textbf{Chemiluminescent Ozone Sample Flow Rate (sccm): }\textbf{\uline{O3FF}}\textbf{\index{O3FF}}\\
\textbf{Chemiluminescent Ozone Nitric Oxide Flow Rate (sccm): }\textbf{\uline{O3FN}}\textbf{\index{O3FN}}\\
\textbf{Chemiluminescent Ozone Sample Pressure (mb): }\textbf{\uline{O3FP\index{O3FP}}}%
\end{minipage}

~~~~~~~~~These variables characterize conditions within the
chemiluminescence ozone sensor. The sample rate, in standard $cm^{3}/s$,
is O3FF, while O3FN gives the NO flow rate in the same units and O3FP
is the pressure in the ozone sample cell.

\textbf{Chemiluminescent Ozone Concentration (ppbv): }\textbf{\uline{O3FC}}\index{O3FC}\\
This is the corrected ozone concentration, with units of ppbv. This
instrument is calibrated both on the ground and in flight by comparison
with the TECO 49 UV instrument. The final data are corrected for
the influence of water vapor on the signal. The detection limit is
0.1 ppbv and the uncertainty is about 10\% for a one-second sample.\\
\end{hangparagraphs}



\section{THE STATE OF THE AIRCRAFT\label{sec:INS}}

The primary sources of information on the location and motion of the
aircraft are from inertial reference systems\index{IRU}\index{inertial reference unit|see {IRU}}\index{inertial reference system|see {IRU}}\index{INS|see {IRU}}
and from global positioning systems\index{GPS}. Both are described
in this section, as well as combined results that merge the best features
of each into composite variables for location and motion.


\subsection{Inertial Reference Systems}

An Inertial Reference System (IRS) or Inertial Reference Unit (IRU)
provides measurements of aircraft position, velocity relative to the
Earth, acceleration and attitude or orientation. For the GV, the system
is a\index{IRU!GV} Honeywell Laseref III/IV Inertial Reference System.
Data from the IRS come via a serial digital bit stream (the ARINC
digital bus) to the ADS\index{ADS=aircraft data system} (Aircraft
Data System). Because there is some delay in transmission and recording
of these variables, adjustments for this delay are made when the measurements
are merged into the processed data files, as documented in the NetCDF\index{NetCDF!header}
header files. Some of the variables are recorded only on the original
``raw'' data tapes and are not usually included on final archived
data files; these are discussed at the end of this section.  See
also Section XXX for information on results from inertial systems
that were used prior to installation of the present Honeywell systems.

An Inertial Reference System\index{IRU!alignment} ``aligns'' while
the aircraft is stationary by measurement of the variations in its
reference frame caused by the rotation of the Earth. Small inaccuracy
in that alignment lead to a ``Schuler oscillation''\index{Schuler oscillation}
that produces oscillatory errors in position and other measurements,
with a period $\tau_{Sch}$ of about 84 minutes ($\tau_{Sch}=2\pi\sqrt{R_{e}/g}$).
Position errors of less than 1.0 n mi/hr are within normal operating
specifications. See Section \ref{sub:IRS/GPS} for discussion of additional
variables, similar to the following, for which corrections are made
for these errors via reference to data from a Global Positioning System.
\begin{hangparagraphs}
\textbf{Latitude (}$\text{�}$\textbf{): }\textbf{\uline{LAT}}\sindex[var]{LAT}\index{LAT}\\
This is the aircraft latitude\index{latitude}, output from the IRS
at 10 Hz. Positive values are north of the equator; negative values
are south. The resolution is 0.00017$^{\circ}$ and the accuracy is
reported by the manufacturer to be 0.164$^{\circ}$ after 6 h of flight.

\textbf{Longitude (}$\text{�}$\textbf{): }\textbf{\uline{LON}}\sindex[var]{LON}\textbf{\uline{\index{LON}}}\\
This is the aircraft longitude\index{longitude}, output from the
IRS at 10 Hz. Positive values are east of the prime meridian; negative
values are west. The resolution is 0.00017$^{\circ}$ and the accuracy
is reported by the manufacturer to be 0.164$^{\circ}$ after 6 h of
flight.

\textbf{Aircraft True Heading (}$^{\circ}$\textbf{): }\textbf{\uline{THDG}}\index{THDG}\\
This is the aircraft's true heading\index{heading!true}, output from
the IRS at 25 Hz. The resolution is 0.00017$\text{�}$ and the accuracy
is quoted by the manufacturer as 0.2$\text{�}$ after 6 h of flight.
``True'' distinguishes the heading from the magnetic heading\index{heading!magnetic},
the heading that would be measured by a magnetic-compass needle.

\textbf{Aircraft Pitch Attitude Angle (}$\text{�}$\textbf{): }\textbf{\uline{PITCH}}\sindex[var]{PITCH}\index{PITCH}\\
This is the aircraft's pitch\index{pitch} angle, provided from the
IRS at 50 Hz. Positive values correspond to the nose of the aircraft
pointing above the horizon. The resolution is 0.00017$\text{�}$ and
the accuracy is quoted by the manufacturer as 0.05$\text{�}$ after
6 h of flight.

\textbf{Aircraft Roll Attitude Angle (}$\text{�}$\textbf{): }\textbf{\uline{ROLL}}\sindex[var]{ROLL}\index{ROLL}\\
This is the aircraft's roll\index{roll} angle, output from the IRS
at 50 Hz. Positive angles occur with the starboard (right) wing down
((i.e., a clockwise rotation from level when facing forward in the
aircraft). The resolution is 0.00017$\text{�}$ and the accuracy is
quoted by the manufacturer as 0.05$\text{�}$ after 6 h of flight..

\textbf{Aircraft Vertical Acceleration (m\,s$^{-2}$): }\textbf{\uline{ACINS}}\sindex[var]{ACINS}\index{ACINS}\\
This\index{acceleration!vertical} is the output from the IRS vertical
accelerometer with its internal drift removed via pressure damping\index{pressure damping}.%
\footnote{For earlier projects using the Litton LTN-51 INS, this is a direct
measurement without adjustment for changes in gravity during flight
and without pressure-damping. Previous use employed a baro-inertial
loop to compensate for drift in the integrated measurement. See the
discussion of WP3 below.%
} Positive values are upward. The sample rate is 50 Hz and the resolution
is 0.0024 m\,s$^{-2}$. 

\textbf{Computed Aircraft Vertical Velocity (m/s): }\textbf{\uline{VSPD}}\sindex[var]{VSPD}\index{VSPD}\\
This is the upward velocity\index{velocity!aircraft vertical} of
the aircraft provided directly by the IRS. It is determined in the
IRS by integration of the vertical acceleration, with damping based
on measured pressure to correct for accumulated errors in the integration
of acceleration. The sample rate is 50 Hz with a resolution of 0.00016
m/s. The Honeywell Laseref IRS employs a baro-inertial loop\index{baro-inertial loop},
similar to that described below for WP3\index{WP3} and the Litton
LTN-51\index{IRU!Litton LTN-51}, to update the value of the acceleration.
It is also filtered within the IRS so that there is little variance
with frequency higher than 0.1 Hz. 

\textbf{Pressure-Damped Aircraft Vertical Velocity (m/s): }\textbf{\uline{WP3}}\sindex[var]{WP3}\index{WP3}
(obsolete)\\
\label{WP3 algorithm}This was a derived output incorporating a third-order
damping feedback loop to remove the drift from the inertial system's
vertical accelerometer (ACINS or VZI) using pressure altitude (PALT)
as a long-term, stable reference. Positive values are up. The Honeywell
IRS now in use provides its own version of this measurement, VSPD\index{VSPD},
and WP3 is now considered obsolete (and in any case should not be
calculated from ACINS as provided by the Honeywell Laseref IRS\index{IRU!Honeywell}
because that ACINS already incorporates pressure damping). Documentation
is included here because many old data files include this variable.
Note that ``pressure altitude\index{altitude!pressure}'' is not
a true altitude but an altitude equivalent to the ambient pressure
in a standard atmosphere, so updating a variable integrated from inertial
measurements to this value can introduce errors vs.~the true altitude..
WP3 was calculated by the data-processing software as follows (with
coefficients in historical use and not updated to the recommendations
elsewhere in this technical note):%
\footnote{Regarding signs, note that ACINS is a number near zero, not near g,
and so already has the estimated acceleration of gravity removed.
The assumption made in the following is that the INS will report values
adjusted for the gravitational acceleration \emph{at the point of
alignment}, which would be $G_{L}$. If $g_{F}$, the estimate for
gravity at the flight altitude (palt) and latitude (lat), is \emph{smaller}
than $G_{L}$ then the difference ($G_{L}-g_{f}$) will be positive;
this will correct for the reference value for ACINS being the gravity
measured at alignment ($G_{L}$) when it should actually be the sensed
gravity ($g_{f}$) at the measurement point, so to obtain (sensed
acceleration - $g_{f}$) it is necessary to add ($G_{L}-g_{f}$) to
ACINS, \emph{increasing }``acz'' in this case. However, the situation
with ``vcorac'' is reversed: ``vcorac'' is a positive term for
all eastward flight, for example, but in that case the motion of the
aircraft makes objects seem lighter (i.e., they experience less acceleration
of gravity) than without such flight. ACINS is positive upward so
it represents a net acceleration of the aircraft upward (as imposed
by the combination of gravity and the lift force of the aircraft).
To accomplish level flight in these circumstances, the aircraft must
actually accelerate downward so the accelerometer will experience
a negative excursion relative to slower flight. To compensate, ``vcorac''
must make a positive contribution to remove that negative excursion
from ``acz''. In the conceptual extreme that the aircraft flies
fast enough for the interior to appear weightless, ACINS would reduce
to -1{*}$G_{L}$ and vcorac would increase to +$G_{L}$, leaving acz
near zero as required if the aircraft were to remain in level flight
in the rotating frame%
}\\
\framebox{\begin{minipage}[t]{0.95\textwidth}%
$g_{1}$ = 9.780356 m\,s$^{-2}$ ~~~%
\footnote{this and several other values are those used in past processing; this
is not the current value%
}\\
$a_{1}$ = 0.31391116$\times$10$^{-6}$ s$^{-2}$\\
$a_{2}$ = .0052885 (dimensionless)\\
VEW = eastward groundspeed of the aircraft (see below)\\
VNS = northward groundspeed of the aircraft (see below)\\
LAT = latitude measured by the IRS {[}$\text{�}${]}\\
$C_{dr}=\pi/180^{\circ}$ = conversion factor, degrees to radians\\
PALT = pressure altitude of the aircraft\\
$\Omega$ = angular rotation of the earth = 7.292116$\times10^{6}$
radians/s\\
$R_{E}$ = radius of the Earth = 6.371229$\times10^{6}$ m\\
$g_{f}$ = local gravity corrected for latitude and altitude\\
$V_{c}$ = correction to gravity for the motion of the aircraft\\
$G_{L}$ = local gravity at the location of IRU alignment, corrected
to zero altitude\\
$\{C[0],C[1],C[2]\}$ = feedback coefficients, \{0.15, 0.0075, 0.000125\}
for 125-s response

\noindent \rule[0.5ex]{1\linewidth}{1pt}
\begin{enumerate}
\item From the pressure altitude PALT (in m) and the latitude LAT (converted
to radians), estimate the acceleration of gravity:\\
\[
g_{f}=g_{1}\left(1+a_{2}\sin^{2}(\mathrm{C_{dr}\{LAT\})}+a_{1}\mathrm{\{PALT\}}\right)
\]

\item Determine corrections for Coriolis acceleration\index{Coriolis acceleration}
and centrifugal acceleration\index{centrifugal acceleration}, based
on motion of the aircraft:\\
\[
a_{c}=2\Omega\mathrm{\{VEW\}}\cos(C_{r}\mathrm{\{LAT\}})+\frac{\mathrm{\{VEW\}}^{2}+\mathrm{\{VNS\}}^{2}}{R_{E}}
\]

\item Estimate the acceleration experienced by the aircraft as follows:\\
\[
\mathrm{acz}=a_{z}=\mathrm{\{ACINS\}}+G_{L}-g_{f}+a_{c}
\]
 
\item Use a feedback loop to update the integrated value of the acceleration
and obtain a new value of the vertical motion of the aircraft. The
following code segment uses \emph{acz}=$a_{z}$, \emph{deltaT{[}{]}}
to represent the time between updates, and \emph{hi3, hx,} and \emph{hxx}
to store the feedback terms:\\
\begin{minipage}[t]{1\columnwidth}%
\begin{lyxcode}
wp3{[}FeedBack{]}~+=~(acz~-~C{[}1{]}~{*}~hx{[}FeedBack{]}~~\\
~~~-~C{[}2{]}~{*}~hxx{[}FeedBack{]})~{*}~deltaT{[}FeedBack{]};\end{lyxcode}
%
\end{minipage}
\item Update the feedback terms:\\
\begin{minipage}[t]{1\columnwidth}%
\begin{lyxcode}
hi3{[}FeedBack{]}~=~hi3{[}FeedBack{]}~+~(wp3{[}FeedBack{]}~~\\
~~~~-~C{[}0{]}~{*}~hx{[}FeedBack{]})~{*}~deltaT{[}FeedBack{]};~\\
hx{[}FeedBack{]}~~=~hi3{[}FeedBack{]}~-~palt;~\\
hxx{[}FeedBack{]}~=~hxx{[}FeedBack{]}~~\\
~~~~~~~~~~~~~~+~hx{[}FeedBack{]}~{*}~deltaT{[}FeedBack{]};~\end{lyxcode}
%
\end{minipage}
\item Set WP3 to the average of the last wp3 result and the current wp3
result.\end{enumerate}
%
\end{minipage}}

\textbf{Inertial Altitude (m): }\textbf{\uline{ALT}}\sindex[var]{ALT}\index{ALT}\\
This is the altitude of the aircraft as provided by the IRS, with
pressure damping applied within the IRU to the integrated aircraft
vertical velocity to avoid the accumulation of errors. The value therefore
is updated to the pressure altitude, not the geometric altitude, and
should be regarded as a measurement of pressure altitude where short-term
variations are provided by the IRU in a way that uses the pressure
altitude as a baseline for longer-term updating.The sample rate is
25 Hz with a resolution of 0.038 m.

\textbf{Aircraft Ground Speed (m/s): }\textbf{\uline{GSF}}\sindex[var]{GSF}\index{GSF}\\
This is the ground speed of the aircraft as provided by the IRU (at
10 Hz). The resolution is 0.0020 m/s.

\textbf{Aircraft Ground Speed East Component (m/s): }\textbf{\uline{VEW}}\sindex[var]{VEW}\index{VEW}\\
The IRU provides this measurement of the east-directed component of
ground speed, at a rate of 10 Hz. The resolution is 0.0020 m/s.

\textbf{Aircraft Ground Speed North Component (m/s) }\textbf{\uline{VNS}}\sindex[var]{VNS}\index{VNS}\\
The IRU provides this measurement of the north-directed component
of ground speed, at a rate of 10 Hz. The resolution is 0.0020 m/s.

\textbf{Distance East/North of a Reference (km): }\textbf{\uline{DEI}}\textbf{\sindex[var]{DEI}\index{DEI@DE\textbf{I}}/}\textbf{\uline{DNI}}\sindex[var]{DNI}\index{DNI}\\
These are derived outputs obtained by subtracting a fixed reference
position from the current position. The values are determined from
measurements of latitude and longitude and converted from degrees
to distance in a rectilinear coordinate system. The reference position
can be either the starting location of the flight or a user-defined
reference point (e.g., the location of a project radar). The accuracy
of these values is dependent on the accuracy of the source of latitude
and longitude measurements (see LAT and LON), and the calculations
are only appropriate for short distances because they do not take
into account the spherical geometry of the Earth..\\
\framebox{\begin{minipage}[t]{0.9\textwidth}%
LON$_{ref}$ = reference longitude ($\text{�}$)\\
LAT$_{ref}$ = reference latitude ($\text{�}$)\\
SF$_{ref}$ = reference scale factor = 111.12 km /$\text{�}$\\
\\
\rule[0.5ex]{1\linewidth}{1pt}

\[
\mathrm{DEI}=(\mathrm{\{LON\}}-\{\mathrm{LON}_{ref}\})\mathrm{SF}_{ref}\cos(\mathrm{\{LAT\}})
\]
\[
\mathrm{DNI=(\mathrm{\{LAT\}}=\{\mathrm{LAT}_{ref}\})\mathrm{SF}_{ref}}
\]
%
\end{minipage}}

\textbf{Radial Azimuth/Distance from Fixed Reference: }\textbf{\uline{FXAZIM}}\textbf{\sindex[var]{FXAZIM}\index{FXAZIM},
}\textbf{\uline{FXDIST}}\sindex[var]{FXDIST}\index{FXDIST}\\
Derived calculations of the radial azimuth ($\text{�})$ and distance
(km) from a fixed reference position (a user-specified reference latitude
and longitude) to the aircraft's location, as discussed in the preceding
paragraph. These are calculated by rectangular-to-polar conversion
of DEI/DNI. 
\end{hangparagraphs}

\subsubsection*{\noun{--Raw IRS Variables Not Included In Normal Data Files}:--}

The following IRS variables\index{IRU variables!raw}\index{IRU measurements!not in normal data files}
are not normally included in archived data files, but their values
are recorded by the ADS and can be obtained from the original ``raw''
data files: \\

\begin{hangparagraphs}
\textbf{Raw Lateral Body Acceleration (m/s$^{2}$): }\textbf{\uline{BLATA}}\sindex[var]{BLATA}\index{BLATA}\\
The raw output from the IRU lateral accelerometer. Positive values
are toward the starboard, normal to the aircraft center line. The
sample rate is 50 Hz with a resolution of 0.0024 m\,s$^{-2}$.

\textbf{Raw Longitudinal Body Acceleration (m/s$^{2}$): }\textbf{\uline{BLONA}}\sindex[var]{BLONA}\index{BLONA}\\
The raw output from the IRU longitudinal accelerometer. Positive values
are in the direction of the nose of the aircraft and parallel to the
aircraft center line. The sample rate is 50 Hz with a resolution of
0.0024 m\,s$^{-2}$.

\textbf{Raw Normal Body Acceleration (m/s$^{2}$): }\textbf{\uline{BNORMA}}\sindex[var]{BNORMA}\index{BNORMA}\\
The raw output from the IRU vertical accelerometer. Positive values
are upward in the reference frame of the aircraft, normal to the aircraft
center line and lateral axis. The sample rate is 50 Hz with a resolution
of 0.0024 m\,s$^{-2}$.

\textbf{Raw Body Pitch Rate (}$\text{�}$\textbf{/s): }\textbf{\uline{BPITCHR}}\sindex[var]{BPITCHR}\index{BPITCHR}\\
The raw output of the IRU pitch rate gyro. Positive values indicate
the nose moving upward and refer to rotation about the aircraft's
lateral axis. The sample rate is 50 Hz with a resolution of 0.0039$\text{�}$/s.

\textbf{Raw Body Roll Rate (}$\text{�}$\textbf{/s): }\textbf{\uline{BROLLR}}\sindex[var]{BROLLR}\index{BROLLR}\\
The raw output of the IRU roll rate gyro. Positive values indicate
starboard wing moving down and refer to rotation about the aircraft
center line. The sample rate is 50 Hz with a resolution of 0.0039$\text{�}$/s.

\textbf{Raw Body Yaw Rate (}$\text{�}$\textbf{/s): }\textbf{\uline{BYAWR}}\sindex[var]{BYAWR}\index{BYAWR}\\
The raw output of the IRU yaw rate. Positive values represent the
nose turning to the starboard and refer to rotation about the aircraft's
vertical axis. The sample rate is 50 Hz with a resolution of 0.0039$\text{�}$/s. 
\end{hangparagraphs}

\subsection{Global Positioning Systems}

GPS\index{global positioning system|see{GPS}}\index{GPS} variables
aboard RAF aircraft are provided by a Trimble TANS-III GPS\index{GPS!Trimble TANS-III}
receiver. It has the ability to track up to 6 satellites at a time
but needs only 4 to provide 3-dimensional position and velocity data
(3 satellites for 2-dimensions). The accuracy of the position measurements
is stated to be 25 meters (horizontal) and 35 meters (vertical) under
``steady-state conditions.''%
\footnote{Note: The GPS signals at one time suffered from ``selective availability,''
a US DOD term for a dithered signal that degrades GPS absolute accuracy
to 100 meters. This was especially noticeable in the altitude measurement,
so GALT normally was not useful. As of 1 May 2000, selective availability
was deactivated to allow everyone to obtain better position measurements.
(See the Interagency GPS Executive Board web site for more information.) %
} Likewise, velocity measurements are within 0.2 m/s for all axes.
Measurement resolution is that of 4-byte IEEE format (about 6 significant
digits). All variables are output at a sample rate of 1 Hz.

Some of the following variables are also available from a second unit
and are qualified by the name of that unit; e.g., GGLAT\_GMN for GGLAT
as measured by a Garmin GPS\index{GPS!Garmin} unit. In addition,
some of the measurements from the GPS\index{GPS!aircraft avionics unit}
units that are part of the aircraft avionics systems are recorded;
these are denoted by a suffix ``\_G'' or ``\_A''.\\
 
\begin{hangparagraphs}
\textbf{GPS Latitude (}$\text{�}$\textbf{): }\textbf{\uline{GGLAT}}\sindex[var]{GGLAT}\index{GGLAT},
\textbf{\uline{LAT\_G\sindex[var]{G}\index{LAT_G@\textbf{LAT\_G}}}};
also formerly \textbf{\uline{GLAT\sindex[var]{GLAT} \index{GLAT}}}\\
The aircraft latitude output from the GPS. Positive values are north
of the equator; negative values are south. GGLAT is provided by the
data-system GPS; LAT\_G and LATF\_G are from the avionics system GPS.
LATF\_G is a fine-resolution measurement that requires special processing. 

\textbf{GPS Longitude (}$\text{�}$\textbf{): }\textbf{\uline{GGLON}}\sindex[var]{GGLON}\index{GGLON},
\textbf{\uline{LON\_G\sindex[var]{G}\index{LON_G@\textbf{LON\_G}}}};
also formerly \textbf{\uline{GLON\sindex[var]{GLON}\index{GLON@\textbf{GLON}}}}\\
The aircraft longitude output from the GPS. Positive values are east
of the prime meridian; negative are west. GGLON is provided by the
(or a) data-system GPS; LON\_G and LONF\_G are from the avionics system
GPS. LONF\_G is a fine-resolution measuremen that requires special
processingt. 

\textbf{GPS Ground Speed (m/s): }\textbf{\uline{GSF}}\textbf{\_}\textbf{\uline{G}}\textbf{\sindex[var]{G}\index{GSF_G@GSF\_G},
}\textbf{\uline{GGSPD}}\sindex[var]{GGSPD}\index{GGSPD}\\
The aircraft ground speed output from the GPS. GSF\_G originates from
an avionics-system GPS; GGSPD originates from a data-system GPS.

\textbf{GPS Ground Speed Vector East Component (m/s): }\textbf{\uline{GGVEW}}\sindex[var]{GGVEW}\index{GGVEW},
\textbf{\uline{VEW\_G}}\sindex[var]{G}\index{VEW_G@VEW\_G}\\
The east component of ground speed as measured by the GPS . VEW\_G
originates from an avionics-system GPS; GGVEW originates from a data-system
GPS.

\textbf{GPS Ground Speed Vector North Component (m/s): }\textbf{\uline{GGVNS}}\sindex[var]{GGVNS}\index{GGVNS},
\textbf{\uline{VNS\_G}}\uline{\sindex[var]{G}\index{VNS_G@VNS\_G}}\\
The northward component of ground speed as measured by a GPS unit.
VNS\_G originates from an avionics-system GPS; GGVNS originates from
a data-system GPS.

\textbf{GPS-Computed Aircraft Vertical Velocity (m/s): }\textbf{\uline{VSPD\_G\sindex[var]{G}\index{VSPD_G@\textbf{VSPD\_G}}}}\textbf{,
}(obsolete) \textbf{\uline{GVZI}}\sindex[var]{GVZI (obsolete)}\index{GVZI}\\
The aircraft vertical velocity provided by a GPS unit. Positive values
are upward.

\textbf{GPS Altitude (m): }\textbf{\uline{GGALT}}\sindex[var]{GGALT}\index{GGALT},
\textbf{\uline{ALT\_G\sindex[var]{G}\index{ALT_G@ALT\_G}}}\\
The aircraft altitude\index{altitude!aircraft!GPS} output from the
GPS. The measurement is with respect to a geopotential surface (MSL)
defined by the GPS's built-in earth model. Positive values are above
the reference surface. GGALT originates from a data-system GPS; ALT\_G
originates from an avionics-system GPS.

\textbf{GPS Aircraft Track Angle ($\text{�}$): }\textbf{\uline{GGTRK\sindex[var]{GGTRK}\index{GGTRK@\textbf{GGTRK}}}}\textbf{,}\textbf{\uline{
TKAT\_G\sindex[var]{G}\index{TKAT_G@\textbf{TKAT\_G}}}}\\
The direction of the aircraft track (degrees clockwise from true north)
as measured by a data-system GPS (GGTRK) or an avionics-system GPS
(TKAT\_G).

\textbf{GPS Height of the Geoid (m): }\textbf{\uline{GGEOIDHT\sindex[var]{GGEOIDHT}\index{GGEOIDHT@\textbf{GGEOIDHT}}}}\\
Height of geoid (MSL) above the WGS84 ellipsoid. 

\textbf{GPS Satellites Tracked: }\textbf{\uline{GGNSAT}}\textbf{\sindex[var]{GGNSAT}\index{GGNSAT@\textbf{GGNSAT}}}\\
The number of satellites tracked by the GPS unit.

\textbf{GPS Mode: }\textbf{\uline{GGQUAL}}\sindex[var]{GGQUAL}\index{GGQUAL}\\
Quality flag: 0, 1, 2 for invalid, normal GPS, and differential GPS,
respectively.

\textbf{GPS Mode: }\textbf{\uline{GMODE\sindex[var]{GMODE (obsolete)}\index{GMODE@\textbf{GMODE}}
}}(obsolete)\\
This is the former output from the GPS indicating the mode of operation.
The normal value is 4, indicating automatic (not manual) mode and
that the receiver is operating in 4-satellite (as opposed to fewer)
mode.

\textbf{GPS Status: }\textbf{\uline{GGSTATUS}}\sindex[var]{GGSTATUS}\index{GGSTATUS},
\textbf{\uline{GSTAT\_G}}\textbf{\sindex[var]{G}}\textbf{\uline{\index{GSTAT_G@\textbf{GSTAT\_G}}}},
(obsolete) \textbf{\uline{GSTAT}}\sindex[var]{GSTAT}\index{GSTAT}\\
The status of the GPS receiver. A value of 1 indicates that the receiver
is operating normally; a value of 0 indicates a warning regarding
data quality. GGSTATUS indicates the status of the data-system GPS;
GSTAT\_G indicates the status of the avionics-system GPS. The obsolete
variable GSTAT, formerly used for the same purpose, has the reverse
meaning: A value of 0 indicates normal operation and any other code
indicates a malfunction or warning regarding poor data accuracy. 


\end{hangparagraphs}

\subsection{Other Measurements of Aircraft Altitude}
\begin{hangparagraphs}
\textbf{Altitude, Reference (MSL): }\textbf{\uline{ALTX}}\textbf{\sindex[var]{ALTX}\index{ALTX},
}\textbf{\uline{GALTC}}\sindex[var]{GALTC}\index{GALTC}\\
\marginpar{revise this as described in the GGALTC algorithm note of 10 Dec}This
is the derived altitude above the geopotential surface obtained primarily
from GALT, the GPS\index{GPS} altitude, with help from another reference
altitude, typically ALT, the inertial altitude. The GPS signals at
times are degraded, most often during aircraft turns. Two GPS status
measurements are used to detect this, but sometimes the information
comes too late. A 10-second running average is calculated of the difference
between the GPS altitude and the reference altitude. When the sample-to-sample
altitude difference changes more than 50 meters or when the GPS status
detects a degraded signal, the altitude from this measurement changes
from GPS altitude to an adjusted reference altitude (defined as reference
altitude plus the running difference average). When the GPS altitude
is again ``good'' and to avoid a sudden altitude transition, the
output linearly steps back to the GPS altitude over the next 10 seconds.
{*}{*}{*} \emph{REVISE THIS, NEW ALGORITHM}

\textbf{ISA Pressure Altitude (m): }\textbf{\uline{PALT}}\sindex[var]{PALT}\index{PALT}\\
\marginpar{Revised Nov 2010}This derived altitude\index{altitude!pressure}
is obtained from the reference barometric (static or ambient) pressure
measurement using the International Standard Atmosphere\index{International Standard Atmosphere}
(\index{ISA|see {International Standard Atmosphere}}ISA), equivalent
to the reference atmosphere for aviation operations worldwide.%
\footnote{See ``U.S. Standard Atmosphere, 1976'', NASA-TM-A-74335, available
for download at \href{http://ntrs.nasa.gov/archive/nasa/casi.ntrs.nasa.gov/19770009539_1977009539.pdf}{this URL}.%
} The pressure altitude is best interpreted as a variable equivalent
to the measured pressure, not as a geometric altitude. In the following
description of the algorithm, some constants (identified by the symbol
$\dagger$) are specified as part of the ISA and so should not be
``improved'' to more modern values such as those given in the table
in section \ref{ConstantsBox} (e.g., $R_{0}^{\prime}).$%
\footnote{Prior to and including some projects in 2010, processing used slightly
different coefficients: for aircraft other than the GV, $T_{0}/\lambda$
was represented by -43308.83, the reference pressure $p_{0}$ was
taken to be 1013.246, and the exponent $x$ was represented numerically
by 0.190284. For the GV, the value of $T_{0}/\lambda$ was taken to
be 44308.0, the transition pressure $p_{T}$ was 226.1551 hPa, $x$
= 0.190284, and coefficient $\frac{R_{0}^{\prime}T_{T}}{gM_{d}}$
was taken to be 6340.70 m instead of 6341.620 m as obtained below.
The difference between these older values and the ones recommended
below is everywhere less than 10 m and so is small compared to the
expected uncertainty in pressure measurements, because 1 hPa change
in pressure leads to a change in pressure altitude that varies from
about 8--40 m over the altitude range of the GV .)%
} A note at \href{http://wiki.eol.ucar.edu/rafscience/ProcessingAlgorithms?action=AttachFile&do=view&target=PressureAltitude.pdf}{this URL}
describes the pressure altitude in more detail and documents the change
that was implemented in November 2010.\\
\\
\framebox{\begin{minipage}[t]{0.9\textwidth}%
$T_{0}$= 288.15 K, reference temperature$\dagger$\\
$\lambda$ \sindex[lis]{lambda@$\lambda$= tropospheric lapse rate, standard atmosphere}=
-0.0065 K/m = the lapse rate\index{International Standard Atmosphere!lapse rate}
for the troposphere$\dagger$\\
$p$ = measured static (ambient) pressure, hPa, usually from PSXC\\
$p_{a}$\sindex[lis]{Pa@$p_{a}$= reference pressure for zero altitude, ISA}
= 1013.25 hPa, reference pressure for PALT=0$\dagger$\\
$M_{d}^{\prime}$ = 28.9644 $kg/kmol$= molecular weight of dry air,
ISA definition$\dagger$\\
$g^{\prime}$ = 9.80665 m\,s$^{-2}$, acceleration of gravity$\dagger$\\
$R_{0}^{\prime}$ = universal gas constant, defined$\dagger$ as 8.31432$\times10^{3}$
$J\,\mathrm{kmol}^{-1}\,\mathrm{K}^{-1}$\\
$z_{T}$ = altitude of the ISA tropopause = 11,000 m$\dagger$ \\
$x=R_{0}^{\prime}\lambda/(M_{d}^{\prime}g^{\prime})$ $\approx$ 0.1902632
(dimensionless)%
\footnote{This is the value, rounded to seven significant figures, that is used
for data processing.%
}\\
\\
\rule[0.5ex]{1\linewidth}{1pt}

For pressure > 226.3206 hPa (equivalent to a pressure altitude < $z_{T}$):\\
\[
\mathrm{PALT}=-\left(\frac{T_{0}}{\lambda}\right)\left(1-\left(\frac{p}{p_{a}}\right)^{x}\right)
\]
otherwise, if $T_{T}$\sindex[lis]{Tt@$T_{T}$= temperature at the ISA tropopause}
and $p_{T}$\sindex[lis]{pT@$p_{T}$= pressure at the ISA tropopause}
are respectively the temperature and pressure at the altitude\index{International Standard Atmosphere!tropopause}
$z_{T}$:\\
\[
T_{T}=T_{0}+\lambda z_{T}=216.65\,\mathrm{K}
\]
\[
p_{T}=p_{a}\Bigl(\frac{T_{0}}{T_{T}}\Bigr)^{\frac{g^{\prime}M_{d}^{\prime}}{\lambda R_{0}^{\prime}}}=226.3206\,\mathrm{hPa}
\]
\[
\mathrm{PALT}=z_{T}+\frac{R_{0}^{\prime}T_{T}}{g^{\prime}M_{d}^{\prime}}\ln\left(\frac{p{}_{T}}{p}\right)
\]
which, after conversion from natural to base-10 logarithm, is coded
to be equivalent to the following:\\

\begin{lyxcode}
~~~~~~~~~~//~transition~pressure~at~the~assumed~ISA~tropopause:

\#define~ISAP1~226.3206

~~~~~~~~~~//~reference~pressure~for~standard~atmosphere:

\#define~ISAP0~1013.25

if~(psxc~>~ISAP1)
\begin{lyxcode}
palt~=~44330.77~{*}~(1.0~-~pow(psxc/ISAP0,~0.1902632));
\end{lyxcode}
else
\begin{lyxcode}
palt~=~11000.0~+~14602.12~{*}~log10(ISAP1/psxc);\end{lyxcode}
\end{lyxcode}
%
\end{minipage}}

\textbf{Geometric Radio Altitude (m): }\textbf{\uline{HGM}}\sindex[var]{HGM}\index{HGM}\\
This is the distance to the surface below the aircraft, measured by
a radio altimeter\index{altimeter!radio}. The maximum range is 762m
(2,500 ft). The instrument changes in accuracy at an altitude of 152
m: The estimated error from 152 m to 762 m is 7\%, while the estimated
error for altitudes below 152 m is 1.5 m or 5\%, whichever is greater.

\textbf{Geometric Radar Altitude (Extended Range) (APN-159) (m): }\textbf{\uline{HGME}}\sindex[var]{HGME}\index{HGME}\\
There are two outputs from an APN-159 radar altimeter\index{altimeter!radar}\index{APN-159},
one with coarse resolution (CHGME) and one with fine resolution (HGME).
Both raw outputs cycle through the range 0-360 degrees, where one
cycle corresponds to 4,000 feet for HGME and to 100,000 feet for CHGME.
To resolve the ambiguity arising from these cycles, 4,000-foot increments
are added to HGME to maintain agreement with CHGME. This preserves
the fine resolution of HGME (1.86 m) throughout the altitude range
of the APN-159.

\textbf{Geometric Radar Altitude (Extended Range) (APN-232) (m): }\textbf{\uline{HGM232}}\sindex[var]{HGM232}\index{HGM232}\\
Altitude above ground as measured by an APN-232 radar altimeter.

\textbf{Pressure-Damped Inertial Altitude (m): }\textbf{\uline{HI3}}\sindex[var]{HI3 (obsolete)}\index{HI3}
(obsolete)\\
The aircraft altitude obtained from the twice-integrated IRU acceleration
(ACINS), pressure-adjusted to obtain long-term agreement with PALT.
Note that this variable has mixed character, producing short-term
variations that accurately track the inertial system changes but with
adjustment to the pressure altitude, which is not a true altitude.
The variable is not appropriate for estimates of true altitude, but
proves useful in the updating algorithm used with the LTN-51 IRU for
vertical wind. See the discussion of WP3 \vpageref{WP3 algorithm}.
This variable is now obsolete.
\end{hangparagraphs}

\subsection{\label{sub:IRS/GPS}Combining IRS and GPS Measurements}

Measurements from the global positioning\index{GPS} and inertial
reference systems\index{IRU} are combined to produce new variables
that take advantage of the strengths of each, so that the resulting
variables have the long-term stability of the GPS and the short-term
resolution of the IRU. This section describes some variables that
result from this blending of variables. These corrected variables
are usually the best available when the GPS and IRS are both functioning. 

One can determine if the GPS is functioning by examining the GPS status
variables described in the previous section or by looking for spikes
or ``flat-lines'' in the data. If the GPS data are missing for a
short time (a few seconds to a minute), accuracy is not affected.
However, longer dropouts will result in accuracies degrading toward
those of the IRU. Without the GPS or another ground reference, the
IRS error cannot be determined empirically, and one should assume
that it is within the manufacturer's specification (1 nautical mile
of error per hour of flight, 90\% CEP). When the GPS is active, RAF
estimates that the correction algorithm produces a position with an
error less than 100 meters. Due to the nature of the algorithm, the
error will increase fromabout 100 meters to the IRU specification
in about one-half hour after GPS information is lost.\medskip{}

\begin{hangparagraphs}
\textbf{GPS-Corrected Inertial Ground Speed Vector, (m/s): }\textbf{\uline{VEWC}}\textbf{\sindex[var]{VEWC}\index{VEWC},
}\textbf{\uline{VNSC}}\sindex[var]{VNSC}\index{VNSC}\\
These variables result from combining GPS and IRU output of the east
and north components of ground speed from a complementary-filter\index{filter, complementary (for wind)}
algorithm. Positive values are toward the east and north, respectively.
The smooth, high-resolution, continuous measurements from the inertial
navigation system, \{VNS, VEW\}, which can slowly accumulate errors
over time, are combined with the measurements from the GPS, \{GVNS,
GVEW\}, which have good long-term stability, via an approach based
on a complementary filter. A low-pass filter, $F_{L}(\{\mathrm{GVNS,GVEW\}})$\sindex[lis]{FL@$F_{L}$= digital low-pass filter},
is applied to the GPS measurements of groundspeed, which are assumed
to be valid for frequencies at or lower than the cutoff frequency
$f_{c}$\sindex[lis]{fc@$f_{c}$= cutoff frequency fo the filter $F_{L}$}
of the filter. Then the complementary high-pass filter, denoted ($1-F_{L}$)($\{\mathrm{VNS,VEW\}}$),
is applied to the IRS measurements of groundspeed, which are assumed
valid for frequencies at or higher than $f_{c}$. Ideally, the transition
frequency would be selected where the GPS errors (increasing with
frequency) are equal to the IRS errors (decreasing with frequency).
The filter used is a three-pole Butterworth lowpass filter\index{filter!Butterworth},
coded following the algorithm described in Bosic, S.~M., 1980: \emph{Digital
and Kalman filtering : An Introduction to Discrete-Time Filtering
and Optimum Linear Estimation, }p. 49. The digital filter used is
recursive, not centered, to permit calculation during a single pass
through the data. If the cutoff frequency lies where both the GPS
and IRU measurements are almost the same, then the detailed characteristics
of the filter (e.g., phase shift) in the transition region do not
matter because the complementary filters have cancelling effects when
applied to the same signal. The transition frequency $f_{c}$ was
chosen to be (1/600) Hz. The Butterworth filter was chosen because
it provides flat response away from the transition. The net result
then is the sum of these two filtered signals, calculated as described
in the following box:\\
\framebox{\begin{minipage}[t]{0.95\textwidth}%
VEW\index{VEW} = IRS-measured east component of the aircraft ground
speed\\
VNS\index{VNS} = IRS-measured north component of the aircraft ground
speed\\
GGVEW\index{GGVEW} = GPS-measured east component of the aircraft
ground speed\\
GGVNS\index{GGVNS} = GPS-measured north component of the aircraft
ground speed\\
$F_{L}()$ = three-pole Butterworth lowpass recursive digital filter\\
\\
\rule[0.5ex]{1\linewidth}{1pt}

\[
\{\mathrm{VNSC}\}=F_{L}(\mathrm{\{GGVNS\})}+(1-F_{L})(\{\mathrm{VNS\}})
\]
\[
\{\mathrm{VEWC}\}=F_{L}(\mathrm{\{GGVEW\})}+(1-F_{L})(\{\mathrm{VEW\}})
\]
%
\end{minipage}}\\
\\
This result is used as long as the GPS signals are continuous and
flagged as being valid. When that is not the case, some means is needed
to avoid sudden discontinuities in velocity (and hence windspeed),
which would introduce spurious effects into variance spectra and other
properties dependent on a continuously valid measurement of wind.
To extrapolate measurements through periods when the GPS signals are
lost (as sometimes occurs, for example, in turns) a fit is determined
to the difference between the best-estimate variables \{VNSC,VEWC\}
and the IRS variables \{VNS,VEW\} for the period before GPS reception
was lost, and that fit is used to extrapolate through periods when
GPS reception is not available. The procedure is as follows:\\
\begin{minipage}[t]{1\columnwidth}%
\begin{enumerate}
\item If GPS reception has never been valid earlier in the flight, use the
IRU values without correction. 
\item If GPS reception is lost after a valid complementary-filter\index{complementary filter (for wind)}
correction has been obtained earlier using the procedure described
above, but no valid Schuler-oscillation\index{Schuler oscillation}
fit has been accumulated as described in {[}3.{]} below, use the correction
factors from the complementary filter reduced by a factor of 0.997
each second, producing an exponential decay back toward the IRS values
with decay time constant of about 5.6 min.
\item When GPS\index{GPS} reception is good, update a least-squares fit
to the difference between the GPS and IRU groundspeeds, for each component.
The errors are assumed to result primarily from a Schuler oscillation\index{Schuler oscillation},
so the three-term fit is of the form $\Delta=a_{1}+a_{2}\sin(\Omega_{Sch}t)+a_{3}\cos(\Omega_{Sch}t)$
, where $\Omega_{Sch}$ is the angular frequency of the Schuler oscillation
(taken to be $2\pi/(5067\, s))$ and $t$ is the time since the start
of the flight. A separate fit is used for each component of the velocity
and each component of the position (discussed below under LATC and
LONC). The fit matrix\index{fit matrix, wind correction} used to
determine these coefficients is updated each time step but the accumulated
fit factors decay exponentially with a 30-min decay constant, so the
terms used to determine the fit are exponentially weighted over the
period of valid data with a time constant that decays exponentially
into the past with a characteristic time of 30 min. This is long enough
to determine a significant portion of the Schuler oscillation but
short enough to emphasize recent measurements of the correction.
\item When GPS data become invalid, if sufficient data (spanning 30 min)
have been accumulated, invert the accumulated fit matrices to determine
the coefficients $\{a_{1},a_{2},a_{3}\}$ and then use the formula
for $\Delta$ in the preceding step to extrapolate the correction
to the IRS measurements while the GPS measurements remain invalid.
Doing so immediately would introduce a discontinuity in \{VNSC,VEWC\},
however, so the correction $\Delta$ is introduced smoothly by adjusting
\{VNSC,VEWC\} smoothly as follows: $\Delta^{\prime}$=$(1-\eta)$(VNSC$_{0}$-GVNS$_{0}$)$+\eta\Delta$
where $\Delta^{\prime}$ is the sequentially adjusted correction,
(VNSC$_{0}$-GVNS$_{0}$) is the difference preserved from the last
time the GPS groundspeed was valid, and\sindex[lis]{eta@$\eta=$update constant for exponential updating}
$\eta=0.997\, s^{-1}$ is chosen to give a decaying transition with
a time constant of about 5.5 min. This has the potential to introduce
some artificial variance at this scale and so should be considered
in cases where variance spectra are analyzed in detail, but it has
much less influence on such spectra than would introducing a discontinuous
transition. Ideally, the current fit and the last filtered discrepancy
(VNSC$_{0}$-GVNS$_{0}$) should be about equal, so this should not
introduce a significant change.\end{enumerate}
%
\end{minipage}

\medskip{}


\textbf{GPS-Corrected Inertial Latitude and Longitude (}$\text{�}$\textbf{):
}\textbf{\uline{LATC}}\textbf{\sindex[var]{LATC}\index{LATC},
}\textbf{\uline{LONC}}\sindex[var]{LONC}\index{LONC}\\
Combined GPS and IRS output of latitude and longitude. Positive values
are north and east, respectively. These variables are the best estimate
of position, obtained by the following approach:\\
\framebox{\begin{minipage}[t]{0.95\textwidth}%
LAT\index{LAT} = latitude measured by the IRS\\
LON\index{LON} = longitude measured by the IRS\\
GGLAT\index{GGLAT} = latitude measured by the GPS\\
GGLON\index{GGLON} = longitude measured by the GPS\\
VNSC\index{VNSC} = aircraft ground speed, north component, corrected
\\
VEWC\index{VEWC} = aircraft ground speed, east component, corrected\\
\\
\rule[0.5ex]{1\linewidth}{1pt}
\begin{enumerate}
\item Initialize the corrected position at the IRS position at the start
of the flight or after any large change (>5$\text{�}$) in the IRS
position.
\item Integrate forward from that position using the aircraft groundspeed
with components \{VNSC,VEWC\}. Note that in the absence of GPS information
this will introduce long-term errors because it does not account for
the Earth's spherical geometry. It provides good short-term accuracy,
but the GPS updating in the next step is needed to compensate for
the difference between a rectilinear frame and the Earth's spherical
coordinate frame and provides a smooth yet accurate track.
\item Use an exponential adjustment to the GPS position, with time constant
that is typically about 100 s.%
\footnote{specifically, LATC += $\eta$(GLAT-LATC) with $\eta=2\pi/(600\,\mathrm{s})$%
}
\item To handle periods when the GPS becomes invalid, use an approach analogous
to that for groundspeed, whereby a Schuler-oscillation fit to the
difference between the GPS and IRS measurements is developed and used
to extrapolate through periods when the GPS is invalid. \end{enumerate}
%
\end{minipage}}\end{hangparagraphs}


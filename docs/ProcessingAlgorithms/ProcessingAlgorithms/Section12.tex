
\section{AEROSOL PARTICLE MEASUREMENTS}

RAF uses a modified TSI, Inc. Model 3760 condensation nucleus counter\index{condensation nucleus counter|see {CN counter}}\index{CN counter}
to measure the concentration of particulates in the atmosphere larger
than about 0.01 $\mu m$ diameter. Individual inlets have been designed
for each research aircraft that provide approximately isokinetic flow
at research airspeeds. The CN counter is often used as a stand-alone
instrument, but it also can be placed downstream of various instruments,
such as a counterflow virtual impactor or differential mobility analyzer.
It is useful at altitudes up to about 11km. It operates by condensing
n-butyl alcohol on the particles as they pass through a cooling/condenser
tube where supersaturation of a few hundred percent is produced. The
particles grow large enough to be seen by a laser-diode optical detector,
and the pulses from tat detector are counted to obtain an estimate
of the total concentration of aerosol particles. The counter does
not resolve particle concentration by size; the lower size limit of
the TSI 3760 is about XXX $\mu m$, and all particles above that size
enter the measurement of the total concentration.

If large concentrations are encountered, two or more particles may
be present in the viewing volume at once and will produce only a single
pulse from the photodetector. This ``coincidence'' error, which
increases from about 0.6\% at a total concentration of $10^{3}/cm^{3}$to
about 6\% at $10^{4}$cm$^{-3}$; a correction for these effects of
coincidence is applied, but for concentrations above about $2\times10^{4}$\,cm$^{-3}$
effects of coincidence become large enough that the correction introduces
significant uncertainty in the measurements. 

The variables associated with these measurements of condensation-nucleus
concentrations are discussed in the remainder of this section.
\begin{hangparagraphs}
\textbf{CN Counter Inlet Pressure (mbar): }\textbf{\uline{PCN}}\sindex[var]{PCN}\index{PCN}\\
PCN is the absolute pressure inside the inlet tube of the instrument,
as measured by a Heise Model 623 pressure sensor. The measurement
is used to correct the sample flow rates (FCN and XICN) that are used
to obtain measurements of concentration.

\textbf{CN Counter Inlet Temperature ($^{\circ}C$): }\textbf{\uline{TEMP1}}\textbf{\sindex[var]{TEMP1}\index{TEMP1},
}\textbf{\uline{TEMP2}}\sindex[var]{TEMP2}\textbf{\index{TEMP2},
}\textbf{\uline{CNTEMP}}\sindex[var]{CNTEMP}\index{CNTEMP}\\
TEMP1, TEMP2 or CNTEMP is the output from a temperature sensor mounted
on the outside of the sampling tube immediately ahead of the counter.
The measurement, an approximation to the temperature of the air passing
through the tube, is used to correct the sample flow rates (FCN and
XICN).

\textbf{}%
\begin{minipage}[t]{1\columnwidth}%
\textbf{Raw CN Counter Sample Flow Rate (slpm): }\textbf{\uline{FCN}}\sindex[var]{FCN}\index{FCN}\\
\textbf{Corrected CN Counter Sample Flow Rate (vlpm): }\textbf{\uline{FCNC\sindex[var]{FCNC}\index{FCNC}}}%
\end{minipage}

~~~~~~~~~FCN is the raw sample flow rate in standard liters
per minute (slpm) measured with a Sierra 830 Mass Flow meter. The
flow meter measures the flow rate that would apply under ``standard''
conditions; i.e., pressure of 1013.25 and temperature of $0^{\circ}C$.
FCNC is the sample flow rate in vlpm (volumetric liters per minute)
corrected for pressure and temperature.\\
\framebox{\begin{minipage}[t]{0.9\textwidth}%
PCN = pressure at the inlet to the CN counter (mb)\\
TEMP1 = temperature at the inlet of the sample tube ($^{\circ}C$)\\
$P_{ref}$ = standard reference pressure, 1013.25 mb\\
$T_{ref}$ = standard reference temperature, 293.26 K \\
\\
\rule[0.5ex]{1\linewidth}{1pt}
\[
\mathrm{FCNC=\{FCN\}}\frac{P_{ref}}{\mathrm{\{PCN\}}}\frac{(\{\mathrm{TEMP1\}}+T_{0})}{T_{ref}}
\]
%
\end{minipage}}\\


\begin{minipage}[t]{1\columnwidth}%
\textbf{Raw CN Isokinetic Side Flow Rate (slpm): }\textbf{\uline{XICN}}\sindex[var]{XICN}\index{XICN}\\
\textbf{Corrected CN Isokinetic Side Flow Rate (vlpm): }\textbf{\uline{XICNC\sindex[var]{XICNC}\index{XICNC}}}%
\end{minipage}\\
XICN is the raw isokinetic side flow rate in standard liters per minute
(slpm) measured with a Sierra 830 Mass Flow meter, and XICNC is that
flow corrected for pressure and temperature to be the true volumetric
flow. For isokinetic sampling, the flow rate at the inlet entrance
needs to equal the true airspeed, and for proper operation the flow
rate through the CN counter should be at least 1.2 vlpm. A side flow
of filtered air is added so both of these conditions can be met.\\
\framebox{\begin{minipage}[t]{0.9\textwidth}%
PCN = pressure at the inlet to the CN counter (mb)\\
TEMP1 = temperature at the inlet of the sample tube ($^{\circ}C$)\\
$P_{r}$ = standard reference pressure, 1013.25 mb\\
$T_{r}$\sindex[lis]{Tr@$T_{r}$= absolute reference temperature, STP}
= 293.26 K \\
\\
\rule[0.5ex]{1\linewidth}{1pt}
\[
\mathrm{XICNC=\{XICN\}}\frac{P_{r}}{\mathrm{\{PCN\}}}\frac{(\{\mathrm{TEMP1\}}+T_{0})}{T_{r}}
\]
%
\end{minipage}}\\


\textbf{TSI CN Counter Output (counts per sample interval); }\textbf{\uline{CNTS}}\sindex[var]{CNTS}\index{CNTS}\\
CNTS is the raw output count from the TSI,~Inc. 3760 condensation
cncleus counter. The project-dependent sample rate may be chosen in
the range from 1--50 Hz. In some unusual cases the counts are divided
by a selected power of two to keep the counter from overflowing; see
the project documentation. 

.

\textbf{Condensation Nucleus (CN) Concentration (cm$^{-3}$): }\textbf{\uline{CONCN}}\sindex[var]{CONCN}\index{CONCN}\\
CONCN is the corrected concentration of condensation nuclei, calculated
with consideration of the sample rate and corrected for losses caused
by coincidence: \\
\framebox{\begin{minipage}[t]{0.9\textwidth}%
CNTS = counts per second from the CN counter\\
$\Delta T$ = interval between recorded samples\\
$D$ = scale factor (normally 1)\\
FCNC = corrected sample flow rate ($cm^{3}/s)$\\
$T_{vv}$ = time each particle is in the view volume = 4.167$\times10-6$
s\\
\\
\rule[0.5ex]{1\linewidth}{1pt}
\[
\mathrm{A=\frac{\{CNTS\}}{\mathrm{\{FCNC\}}\Delta T}\, D}
\]
\begin{equation}
\mathrm{CONCN=A}\, e^{AT_{vv}\mathrm{\{FCNC\}}}\label{eq:12.1}
\end{equation}
%
\end{minipage}} \\
\\
See the introduction to this section for comments regarding the range
of validity of the coincidence correction in Eq.~ (\ref{eq:12.1}).

\end{hangparagraphs}


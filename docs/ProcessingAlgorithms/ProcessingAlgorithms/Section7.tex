
\section{THE STATE OF THE ATMOSPHERE\label{sec:State Variables}}


\subsection{General Information On Variable Names\index{Variable Names}}

Measurements of meteorological state variables like pressure, temperature,
water vapor pressure often are made at various locations on an aircraft.
To distinguish among similar measurements, many variable names incorporate
an indication of where the measurement was made. In this document,
locations in variable names are represented by ``x'', where ``x''
may represent the following:

\begin{center}
\begin{tabular}{|c|c|}
\hline 
Character & Location\tabularnewline
\hline 
\hline 
B & bottom (or bottom-most)\tabularnewline
\hline 
\emph{B} & (obsolete) boom\tabularnewline
\hline 
F & fuselage\tabularnewline
\hline 
G & (obsolete) gust probe\tabularnewline
\hline 
R & radome\tabularnewline
\hline 
T & top (or top-most)\tabularnewline
\hline 
W & wing\tabularnewline
\hline 
\end{tabular} 
\par\end{center}

In addition, a true letter 'X' (not replaced by the above letters)
may be appended to a measurement to indicate that it is the preferred
choice used for derived calculations. Other suffixes sometimes used
to distinguish among measurements are these: 'D' for a digital sensor;
'H' for a heated (usually, deiced) sensor, 'L' for port side, and
'R' for starboard side.

RAF uses the radome\label{radome gust-sensing system} gust-sensing\index{radome gust-sensing}
technique (Brown, E.~N, C.~A.~Friehe, and D.~H.~Lenschow, 1983:\emph{
Journal of Climate and Applied Meteorology,} \textbf{22, }171--180)
as the reference system to measure incidence angles (angles of attack
and sideslip). The pressure difference between sensing ports above
and below the center line of the radome is used, along with the dynamic
pressure\index{dynamic pressure} measured at a port on the centerline
and referenced to the static pressure, to determine the angle of attack,
and similarly for the sideslip angle. A Rosemount Model 858AJ gust
probe\index{858AJ gust probe} is occasionally used for specialized
measurements. The radome measurements are made by differential pressure
sensors located in the nose area of the aircraft and connected to
the radome by semi-rigid tubing.


\subsection{\label{sub:PTq}Pressure\index{Pressure}, Temperature\index{Temperature},
and Humidity\index{Humidity}}
\begin{hangparagraphs}
\textbf{Static Pressure (mb): }\textbf{\uline{PSx}}\textbf{,\sindex[var]{PSx@\textbf{PSx}}\index{PSx}
}\textbf{\uline{PSxC}}\sindex[var]{PSxC}\index{PSxC}\\
The atmospheric pressure at the flight level of the aircraft, measured
by a calibrated absolute (barometric) transducer at location x. PSx
is the measured static or ambient pressure\index{pressure!ambient},
and may be affected by local flow-field distortion\index{flow-field distortion}.
PSxC is corrected for local flow-field distortion. (See \href{http://www.eol.ucar.edu/raf/Bulletins/bulletin21.html}{RAF Bulletin \#{}21}.)
For output variables PSFD\index{PSFD} and PSFDC\index{PSFDC}, the
letter ``D'' indicates that the transducer used was a Rosemount
Model 1501 digital absolute pressure transducer\index{pressure!transducer}.
For output variable PSFRD\index{PSFRD}, the letters ``RD'' indicate
that the pressure transducer used was a Ruska Model 7885-1B digital
absolute pressure transducer. Other static pressure measurements are
made by a Rosemount Model 1201 absolute pressure transducer, an analog
transducer used with appropriate calibration coefficients.

\textbf{Total Temperature ($\text{�}$C): }\textbf{\uline{TTx}}\textbf{\sindex[var]{TTx}\index{TTx},
}\textbf{\uline{TTxH}}\sindex[var]{TTxH}\index{TTxH}\\
The recovery temperature\index{temperature!recovery} is the temperature
sensed by a temperature probe that is exposed to the atmosphere. In
flight, the temperature is heated above the ambient temperature\index{temperature!ambient}
because it senses the temperature of air near the sensor that has
been heated adiabatically during compression as it is brought near
the airspeed of the aircraft. These variables are the measurements
of that recovery temperature from calibrated temperature sensors at
location x. (The name is misleading, but is kept to comply with long-standing
practice. These are not true total temperature\index{temperature!total}
measurements, for which the air would be at the same speed as the
aircraft.) For Rosemount temperature probes, the recovery temperature
is near the total temperature, but all probes must be corrected to
obtain either true total temperature or true ambient temperature.
In the standard output, the variable name also conveys the sensor
type: TTx is a measurement from a Rosemount Model 102 non-deiced temperature
sensor, while TTxH is the measurement from a Rosemount Model 102 deiced
(heated) temperature sensor. Some past experiments also used a reverse-flow
temperature housing; the associated variable name for this probe was
TTRF.

\textbf{Ambient Temperature ($\lyxmathsym{�}$C): }\textbf{\uline{ATx}}\textbf{\sindex[var]{ATx}\index{ATx},
}\textbf{\uline{ATxH}}\sindex[var]{ATxH}\index{ATxH}\\
The 'x' in the name of the variable used for ambient temperature,
ATx, conveys the same information regarding sensor type and location
as the variable name used with total (recovery) temperature. See the
discussion above regarding TTx. The ambient temperature (also known
as the static air temperature\index{temperature!static air}) is calculated
from the measured recovery temperature, which includes dynamic heating
effects caused by the airspeed of the aircraft. The calculated temperature
therefore depends on the recovery temperature TTx\index{TTx} as well
as the dynamic\index{pressure!dynamic} and static pressure\index{pressure!ambient},
usually respectively QCXC\index{QCXC} and PSXC\index{PSXC}. The
pressures are first corrected from the raw measurements\index{measurements!raw}
QCX\index{QCX} and PSX\index{PSX} to obtain variables that account
for deviations caused by airflow around the aircraft and/or position-dependent
systematic errors. The basic equations use conservation of energy
for a perfect gas undergoing an adiabatic compression.\\
\\
The following section combines discussion of the calculations of temperature
and airspeed, to reflect the linkage between these derived measurements.
\\
\\
\label{ambient temperature and TAS calculation}Aircraft temperature\index{temperature!sensor}
sensors do not measure the total temperature\index{temperature!total},
but measure the temperature of the air immediately in contact with
the sensing element. This air will not have undergone an adiabatic
deceleration to zero velocity and hence will have a temperature $T_{r}$
somewhat less than $T_{t}$. This temperature is the measured or ``recovery''
temperature\index{temperature}. The ratio of the actual temperature
difference attained to the temperature difference relative to the
total temperature is defined to be the recovery factor\index{recovery factor}
$\alpha:$\sindex[lis]{a@$\alpha$= recovery factor, temperature probe}\\
\begin{equation}
\alpha=\frac{T_{r}-T_{a}}{T_{t}-T_{a}}\label{eq:8.2-1}
\end{equation}
From conservation of energy\index{conservation of energy}:\\
\begin{equation}
\frac{U_{a}^{2}}{2}+c_{p}T_{a}=\frac{U_{r}^{2}}{2}+c_{p}T_{r}=\frac{U_{t}^{2}}{2}+c_{p}T_{t}\label{eq:8.1-1}
\end{equation}
\\
where $\{U_{a},\, U_{r},\, U_{t}\}$ \sindex[lis]{Ua@$U_{a}$= true airspeed (sometimes $U$)}are
respectively the aircraft true airspeed\index{true airspeed}, the
airspeed relative to the aircraft of the air in thermal contact with
the sensor, and the airspeed of air relative to the aircraft when
fully brought to the motion of the sensor (i.e., zero). The corresponding
absolute temperatures (expressed in kelvin) for the same conditions
are $\{T_{a},\, T_{r},\, T_{t}\}$\sindex[lis]{Ta@$T_{a}$= ambient air temperature in absolute units; sometimes,
$T_{K}$}\sindex[lis]{Tr@$T_{r}$= recovery temperature}\sindex[lis]{Tt@$T_{t}$= total air temperature},
the ambient, recovery, and total temperatures. \\
\\
Then, from (\ref{eq:8.1-1}), 
\begin{equation}
T_{a}=T_{r}-\alpha\frac{U_{a}^{2}}{2c_{p}}\label{eq:8.3-1}
\end{equation}
The temperature sensors used on RAF aircraft are designed to decelerate
the air adiabatically to near zero velocity. Recovery factors determined
from wind tunnel testing for the Rosemount sensors are 0.95 (non-deiced
model) and 0.98 (deiced model). %
\footnote{ The recovery factor determined for the now-obsolete NCAR reverse-flow
sensor was 0.6. The recovery factor for the now retired NCAR fast-response
(K-probe) temperature sensor was 0.8. %
}Recovery factors\index{recovery factor} have also been determined
from flight maneuvers, often from ``speed runs'' where the aircraft
is flown level through its speed range and the variation of recovery
temperature with airspeed is used with (\ref{eq:8.3-1}), with the
assumption that $T_{a}$ remains constant, to determine the recovery
factor. Data files and project summaries normally document what recovery
factor was used for processing a particular project.\\
\\
As can be seen in the above equation, the true airspeed $U_{a}$ is
used to calculate the ambient temperature $T_{a}$. However, the ambient
temperature is also needed to calculate the true airspeed. Therefore
the constraints imposed on ambient temperature and true airspeed by
the measurements of recovery temperature, dynamic pressure (the pressure
measured by a pitot tube pointed into the airstream and assumed to
be that obtained when the incoming air is brought to rest relative
to the aircraft), and ambient (or ``static'') pressure must be used
to solve simultaneously for the two unknowns, temperature and airspeed.
\\
\\
The relationship is conveniently derived by first calculating the
dimensionless Mach number\index{Mach number}\sindex[lis]{M@$M$= Mach number, ratio of airspeed to the speed of sound}
($M$), which is the ratio of the airspeed to the speed of sound\index{speed of sound}
($U_{s}=\sqrt{\gamma R_{d}T_{a}}$ \sindex[lis]{Us@$U_{s}$= speed of sound}where
$\gamma$ is the ratio of specific heats of air, $c_{p}/c_{v}$).
The Mach number is a function of air temperature only and can be determined
as follows: \\
a). Express energy conservation, as in (\ref{eq:8.1-1}), in the form\\
\begin{equation}
d\left(\frac{U_{a}^{2}}{2}\right)+c_{p}dT=0\,\,\,\,.\label{eq:8.1a-1}
\end{equation}
\\
b). Use the perfect gas law to replace $dT$ with $\frac{pV}{nR}(\frac{dV}{V}+\frac{dp}{p})$
where $V$\sindex[lis]{V@$V$= volume} and $p$\sindex[lis]{p@$p$= pressure}
are the volume and pressure of a parcel of air. Then use the expression
for adiabatic compression\index{adiabatic compression} in the form
$pV^{\gamma}=constant$ to replace the derivative $dV$ with $-\frac{-1}{\gamma}\frac{dp}{p}$,
leading to the equation:\\
\begin{equation}
\frac{U_{a}^{2}}{2}+c_{p}T_{a}=c_{p}T_{a}\left(\frac{p_{t}}{p_{a}}\right)^{\frac{R}{c_{p}}}\label{eq:8.4-1}
\end{equation}
where $p_{t}$ is the total pressure\index{pressure!total} (i.e.,
PSXC\index{PSXC}+QCXC\index{QCXC}) and $p_{a}$ is the ambient pressure\index{pressure!ambient}\sindex[lis]{pa@$p_{a}$= ambient air pressure}
(PSXC\index{PSXC}).\\
\\
c). Use the above definition of the Mach number\index{Mach number}
$M$ ($M=U_{a}/U_{s}$) in the form $U_{a}^{2}=\gamma M^{2}R_{d}T_{a}$
to obtain:\\
\begin{equation}
M^{2}=\left(\frac{2c_{v}}{R_{d}}\right)\left[\left(\frac{p_{t}}{p_{a}}\right)^{\frac{R}{c_{p}}}-1\right]\label{eq:8.5-1}
\end{equation}
which can be found from $p_{t}$ and $p_{a}$ alone.\\
\\
d). Use the expression for ambient temperature in terms of recovery
temperature and airspeed, (\ref{eq:8.3-1}), to obtain the temperature
in terms of the Mach number and the recovery temperature:\\
\begin{eqnarray}
T_{a} & = & T_{r}-\alpha\frac{U_{a}^{2}}{2c_{p}}=T_{r}-\alpha\frac{M^{2}\gamma R_{d}T_{a}}{2c_{p}}\nonumber \\
 & = & \frac{T_{r}}{1+\dfrac{\alpha M^{2}R_{d}}{2c_{v}}}\label{eq:8.6-1}
\end{eqnarray}
\\
e). Express the true airspeed\index{true airspeed} ($U_{a}$) as\\
\begin{equation}
U_{a}=M\sqrt{\gamma R_{d}T_{a}}\label{eq:8.7-1}
\end{equation}
\\
Then the temperature is found as described in the following box:\\
\\
 %
\framebox{\begin{minipage}[t]{0.9\textwidth}%
TTX\index{TTX} = measured recovery temperature ($T_{r})$\\
QCxC\index{QCxC} = measured difference between the dynamic and static
pressures\\
PSXC\index{PSXC} = measured ambient pressure, after airflow/location
correction ($p_{a}$)\\
MACHx\index{MACHx}\sindex[var]{MACH = Mach number} = Mach number
based on QCxC and PSXC (there may be several of these)\\
MACHX = Mach number based on QCXC and PSXC (this is the selected best
of the MACHx options)\\
$\alpha$ = recovery factor for the particular temperature sensor\\
\\
\rule[0.5ex]{1\columnwidth}{1pt}

From (\ref{eq:8.5-1}),

\begin{equation}
\mathrm{MACHx}=\left\{ \left(\frac{2c_{v}}{R_{d}}\right)\left[\left(\frac{\mathrm{\{PSXC\}+\{QCxC\}}}{\mathrm{\{PSXC\}}}\right)^{\frac{R}{c_{p}}}-1\right]\right\} ^{1/2}\label{eq:8.8-1}
\end{equation}
\\
From (\ref{eq:8.6-1})

\begin{equation}
\mathrm{ATx}=\frac{\mathrm{\left(\{TTx\}+T_{0}\right)}}{\left(1+\dfrac{\alpha\mathrm{(\{MACHX\})}^{2}R_{d}}{2c_{v}}\right)}-T_{0}\label{eq:8.9-1}
\end{equation}
%
\end{minipage}}

\textbf{Dew/Frost Point ($\text{�}$C): }\textbf{\uline{DPx}}\sindex[var]{DPx}\index{DPx}\\
The dew point\index{dew point} or frost point\index{frost point}
is measured by either an EG\&G Model 137, a General Eastern Model
1011B or a Buck Model 1011C dew-point hygrometer\index{hygrometer!dewpoint}.
Below 0\textbf{$\text{�}$C} the instrument is assumed to be measuring
the frost point, although occasionally in climbs there is a short
transition near the freezing level before the condensate on the mirror
of the instrument freezes. These are the uncorrected values, so they
represent directly what the instrument senses. The dew point is a
measure of water vapor pressure and so is dependent on the vapor pressure
in the housing being equal to that in the ambient air. This is achieved
by appropriate orientation of the sensor housing and, recently, by
measurement of the pressure and correction of the measurements, as
discussed in regard to DPxC.

\textbf{Corrected Dew Point (C): DPxC}\sindex[var]{DPxC}\index{DPxC}\sindex[var]{}\index{PSDPx}\\
The corrected dew point\index{dew point!corrected} is the dew point
estimated from the original measurement, referenced to the equilibrium
vapor pressure\index{pressure!water vapor!equilibrium} over a plane
water surface in the absence of other gases. Dew/frost point hygrometers\index{hygrometer}
measure the equilibrium point in the presence of air\index{enhancement factor, equilibrium water vapor},
and this affects the measurement in a minor way. Calculation of this
variable removes this dependence, so the vapor pressure obtained from
the corrected dew or frost point is that vapor pressure corresponding
to equilibrium in the absence of air. In addition, if the measurement
is below 0$\text{�}$C, it is assumed to be a measurement of frost
point and a corresponding dew point is calculated from the measurement
(with correction for the influence of the total pressure on the measurement.\marginpar{changed 2011}\\
\\
An additional correction is needed in those cases where the pressure\index{pressure!dew point housing}
in the housing of the instrument (measured as PSDPx) differs from
the ambient pressure, because the changed pressure affects the partial
pressure\index{pressure!partial, water vapor} of water vapor in proportion
to the change in total pressure and so changes the measured dew point
from the desired quantity (that in the ambient air) to that in the
housing\index{hygrometer!housing}. This is especially important in
the case of the GV because the potential effect increases with airspeed.
If the pressure in the housing is measured or otherwise known (e.g.,
from correlations with other measurements), then this correction can
be introduced into the processing algorithm at the same time that
the correction for the presence of dry air is introduced. This correction,
based on the pressure PSDPx, is included in the following discussion.\\
\\
The relationship between water vapor pressure and dew or frost point
is based on the Murphy and Koop%
\footnote{Q. J. R. Meteorol. Soc. (2005), 131, pp. 1539\textendash{}1565%
}\index{Murphy and Koop, 2005}\index{pressure!water vapor!equilibrium}
(2005) equations.%
\footnote{Prior to 2010, the vapor pressure relationship used was the Goff-Gratch
formula as given in the Smithsonian Tables (List, 1980).%
} They express the equilibrium vapor pressure as a function of frost
point or dew point \emph{and at a total air pressure $p$} in relationships
that are equivalent to the following equations:\\
\begin{eqnarray}
e=e_{s,i}(T_{FP})= & b_{0}^{\prime}\exp(b_{1}\frac{(T_{0}-T_{FP})}{T_{0}T_{FP}}+b_{2}\ln(\frac{T_{FP}}{T_{0}})+b_{3}(T_{FP}-T_{0}))\label{eq:MK1-1}
\end{eqnarray}
\begin{equation}
e=e_{s,w}(T_{DP})=e_{0}\exp\left((\alpha-1)e_{6}+d_{2}(\frac{T_{0}-T_{DP}}{T_{DP}T_{0}}\right)+d_{3}\ln(\frac{T_{DP}}{T_{0}})+d_{4}(T_{DP}-T_{0})\label{eq:proposedNewWater}
\end{equation}
\begin{equation}
f(p,T_{P})=1+p(f_{1}+f_{2}T_{p}+f_{3}T_{P}^{2}))\label{eq:EnhancementFactor}
\end{equation}
where $e$\sindex[lis]{e@$e$= water vapor pressure} is the water
vapor pressure, $T_{FP}$\sindex[lis]{Tdp@$T_{DP}$= temperature at the dew point}\sindex[lis]{Tfp@$T_{FP}$= temperature at the front point}
or $T_{DP}$ is the frost or dew point, respectively, expressed in
kelvin, $T_{0}$\sindex[lis]{T0@$T_{0}$= 273.15 K}=273.15\,K, $e_{s,i}(T_{FP})$\sindex[lis]{esi@$e_{s,i}$= equilibrium vapor pressure over a plane ice surface}
is the equilibrium vapor pressure over a plane ice surface at the
temperature $T_{FP}$, $e_{s,w}(T_{DP})$\sindex[lis]{esl@$e_{s,l}$= equilibrium vapor pressure over a plane water surface}
is the equilibrium vapor pressure over a plane water surface at the
temperature $T_{DP}$ (above or below 0$\text{�}$C), and $f(p,T_{P})$\sindex[lis]{fpT@$f(p,T_{p})$= enhancement factor for equilibrium water vapor
pressure}is the enhancement factor at total air pressure $p$ and temperature
$T_{p}$\sindex[lis]{Tp@$T_{p}=$dew point temperature if above 0$^{\circ}C$, frost point
temperature otherwise}, with $T_{P}$ equal to $T_{DP}$ when above 0$\text{�}$C and $T_{FP}$
when below 0$\text{�}$C . The enhancement factor is defined so that
the ambient vapor pressure\sindex[lis]{ea@$e_{a}$= ambient water vapor pressure}
$e_{a}$ is related to the \emph{measured }dew or frost point (not
the dew or frost point equivalent to the saturation vapor pressure
at the dew or frost point) by $e_{a}=f(p,T_{P})\, e$. If the effect
of the enhancement factor is to be removed in order to report a dew
point in correspondence with the saturation vapor pressure, then the
effect of the enhancement factor as represented by $f(p,T_{P})$ must
be removed from the measurement. The coefficients used in the above
formulas are given in the following tables, with the additional definitions
that \sindex[lis]{alpha@$\alpha$= tanh($e_{s}(T-T_{x})$, Murphy/Koop equations}$\alpha=\tanh(e_{5}(T-T_{x}))$,
\sindex[lis]{Tx@$T_{x}$= 218.8~k, Murphy/Koop equations}$T_{X}$
= 218.8~K, and $d_{i}=e_{i}+\alpha e_{i+5}$ for i = \{2,3,4\}:\\
\framebox{\begin{minipage}[t]{0.95\textwidth}%
\textbf{~~~~~}%
\begin{tabular}{|c|c|}
\hline 
\textbf{Coefficient} & \textbf{Value}\tabularnewline
\hline 
\hline 
$b_{0}^{\prime}$ & 6.11536\,hPa\tabularnewline
\hline 
$b_{1}$ & $-5723.265\, K,$\tabularnewline
\hline 
$b_{2}$ & 3.53068\tabularnewline
\hline 
$b_{3}$ & -0.00728332\,K$^{-1}$\tabularnewline
\hline 
$f_{1}$ & 4.923$\times10^{-5}$ hPa$^{-1}$\tabularnewline
\hline 
$f_{2}$ & -3.25$\times10^{-7}$hPa$^{-1}$K$^{-1}$\tabularnewline
\hline 
$f_{3}$ & 5.84$\times10^{-10}$hPa$^{-1}$K$^{-2}$\tabularnewline
\hline 
\end{tabular}\textbf{~~~~~}%
\begin{tabular}{|c|c|}
\hline 
\textbf{coefficient} & \textbf{value}\tabularnewline
\hline 
\hline 
$e_{0}$ & 6.091886 hPa\tabularnewline
\hline 
$e_{1}$ & 6.564725\tabularnewline
\hline 
$e_{2}$ & -6763.22\,K\tabularnewline
\hline 
$e_{3}$ & -4.210\tabularnewline
\hline 
$e_{4}$ & 0.000367\,K$^{-1}$\tabularnewline
\hline 
$e_{5}$ & 0.0415\,K$^{-1}$\tabularnewline
\hline 
$e_{6}$ & -0.1525967\tabularnewline
\hline 
$e_{7}$ & -1331.22\,K\tabularnewline
\hline 
$e_{8}$ & -9.44523\tabularnewline
\hline 
$e_{9}$ & 0.014025\,K$^{-1}$\tabularnewline
\hline 
\end{tabular}%
\end{minipage}}\\
\\
The vapor pressure in the instrument housing, \sindex[lis]{eh@$e_{h}$= water vapor pressure in an instrument housing}$e_{h}$,
is related to the sensed dew or frost point according to equation
(\ref{eq:MK1-1}) or (\ref{eq:proposedNewWater}), but further corrections
must also be made for the enhancement factor and to account for possible
difference between the pressure in the sensor housing\sindex[lis]{ph@$p_{h}$= pressure in a sensor housing}
$p_{h}$ and the ambient pressure $p$: \\
\begin{equation}
e_{a}=f(p,T_{p})e_{h}\frac{p}{p_{h}}\label{eq:HousingPressureCorrection}
\end{equation}
 \\
Because processing to obtain the corrected dew point DPxC\index{DPxC}
from the ambient vapor pressure\index{pressure!water vapor} $e_{a}$
would require difficult inversion of the above formulas, interpolation
is used instead. A table constructed from (\ref{eq:MK1-1}) and another
constructed from (\ref{eq:proposedNewWater}), giving vapor pressure
as a function of frost point or dew point temperature in 1$\text{�}$C
increments from -100 to +50$\text{�}$C, is then used for three-point
Lagrange interpolation (via a function described below as $T_{DP}=F_{D}(e)$)\sindex[lis]{Fd@$F_{d}$= interpolation formula, dew point temperature from water
vapor pressure} to find dew point temperature from vapor pressure.%
\footnote{prior to 2011 the conversion was made using the formula $\mathrm{DPxC=0.009109+DPx(1.134055+0.001038DPx)}$.
For instruments producing measurements of vapor density (RHO), the
previous Bulletin 9 section incorrectly gave the conversion formula
as $DPxC=273.0Z/(22.51-Z)$, a conversion that would apply to frost
point, not dew point. However, the code in use shows that the conversion
was instead $237.3Z/(17.27-Z)$, where Z in both cases is $Z=\ln((\mathrm{ATX}+273.15)\mathrm{RHO/1322.3)}$. %
} Tests of these interpolation formulas against high-accuracy numerical
inversion of formulas (\ref{eq:MK1-1}) and (\ref{eq:proposedNewWater})
showed that the maximum error introduced by the interpolation formula
was about 0.004$\text{�}$C and the standard error about 0.001$\text{�}$C.
This inversion then provides a corrected dew point\index{dew point!corrected}
that incorporates the effects of the enhancement factor as well as
differences between the ambient pressure and that in the housing.
\\
\\
For other instruments that measure vapor density, such as Lyman-alpha\index{Lyman-alpha hygrometer}
or tunable diode laser hygrometers\index{hygrometer!tunable diode laser},
a similar conversion is made from vapor density to dew point, as documented
below:\\
\framebox{\begin{minipage}[t]{0.95\columnwidth}%
DPx\index{DPx} = uncorrected dew point measurement directly from
an instrument, $^{\circ}C$\\
RHO = water vapor density\index{water vapor!density} measurement
directly from the instrument, $g/m^{3}$\\
ATX\index{ATX} = reference ambient temperature\index{temperature!ambient}
($^{\circ}C$)\\
$T_{K}$ \sindex[lis]{Tk@$T_{K}$= absolute temperature (in kelvin)}=
temperature in kelvin = ATX+$T_{0}$\\
PSXC\index{PSXC} = reference ambient pressure (hPa) = $p$

$e_{t}$ = intermediate vapor pressure used for calculation only\\
$M_{w}$= molecular weight of water\\
$R_{0}$ = universal gas constant

\noindent \rule[0.5ex]{1\linewidth}{1pt}

for dew/frost point hygrometers, producing the measurement DPx:~~~~if
DPx < 0$\text{�}$C:

~~~~~~~~obtain $e_{t}$ from (\ref{eq:MK1-1}) using $T_{FP}$=DPx
+ $T_{0}$

~~~~else (i.e., DPx $\geq$ 0$\text{�}$C):

~~~~~~~~obtain $e_{t}$ from (\ref{eq:proposedNewWater})
using $T_{DP}=DPx+T_{0}$

~~~~correct $e_{t}$ for enhancement factor to get actual vapor
pressure $e$:

~~~~~~~~$e=f(p,T_{P})\, e_{t}$

~~~~obtain DPxC by finding the dew point corresponding to the
vapor pressure $e$:

~~~~~~~~DPxC=$F_{DP}(e)$

---- ---- ---- ---- ---- ---- ---- ---- ---- ---- ---- ---- ---- ----
----

for other instruments producing measurements of vapor density (RHO,
g/m$^{3}$):%
\footnote{prior to 2011 the following formula was used: 
\[
Z=\frac{\ln((\mathrm{ATX}+273.15)\,\mathrm{RHO}}{1322.3}
\]


\[
\mathrm{DPxC}=\frac{273.0\, Z}{(22.51-Z)}
\]
%
}

~~~~find the water vapor pressure in units of hPa:

~~~~~~~~$e=$(\{RHO\}~$R_{0}T_{K}$/$M_{w}$)$\times10^{-5}$

~~~~find the equivalent dew point:

~~~~~~~~DPxC=$F_{DP}(e)$%
\end{minipage}}\\


\textbf{Water Vapor Pressure (mb): }\textbf{\uline{EDPC}}\sindex[var]{EDPC}\index{EDPC}\\
EDPC is the ambient vapor pressure of water, used in the calculation
of several derived variables. It is usually obtained from a measurement
of dew point, DPXC\index{DPXC}, which includes correction for the
enhancement factor\index{enhancement factor} that influences dew
point or frost point measurements.%
\footnote{prior to 2011, this variable was calculated using the Goff-Gratch
formula. See the discussion of DPXC for more information on previous
calculations.%
} The formula for obtaining the ambient water vapor pressure as a function
of dew point is given in the discussion of DPxC in Section \ref{sec:State Variables},
Eqs.~(\ref{eq:proposedNewWater}) and (\ref{eq:EnhancementFactor}).\\
\begin{comment}
this is a description of the old Bulletin-9 section, saved here for
reference. It did not correspond to the code in use prior to the 2011
change, however; only the >0 form of $f$ was used, and there was
a small error in coefficients, as described in a 2010 Note that documented
the change made in 2011 and the reasons for it.%
\begin{minipage}[t]{1\columnwidth}%
DPX = measured dew point ($\geq0^{\circ}$) or frost point ($<0^{\circ})$
in $^{\circ}C$\\
$D_{K}$ = DPX + 273.15\,K = measured dew point in kelvin\\
$f$= enhancement factor (Appendix C)\\
\\
\rule[[0.5ex]]{1.0\linewidth}{1pt}if DPX$\geq0^{\circ}C:$\\
\[
A=23.832241-5.02808\,\log\left(D_{K}\right)-1.3816\times10^{-7}(10^{11.334-0.0303998(D_{K}})
\]
\[
f=1.0007+(3.46\times10^{-6}\mathrm{PSXC})
\]
if DPX $<0^{\circ}C$:
\[
A=3.56654\log_{10}(D_{K})-0.0032098(D_{K})-\frac{2484.956}{D_{K}}+2.0702294
\]
\[
f=1.0003+(4.18\times10^{-6}\mathrm{PSXC})
\]
vapor pressure:

\begin{equation}
\mathrm{EDPC}=f\,10^{A}\label{eq:8.11GoffG-1}
\end{equation}
%
\end{minipage}
\end{comment}
\\
\\


\textbf{Relative Humidity (per cent or Pa/hPa): }\textbf{\uline{RHUM}}\sindex[var]{RHUM}\index{RHUM}\\
The relative humidity\index{humidity!relative}\index{relative humidity|see humidity, relative}
is the ratio of the water vapor pressure to the saturation water vapor
pressure in equilibrium over a plane \emph{liquid}-water surface,
scaled to express the result in units of per cent:\\
 %
\framebox{\begin{minipage}[t]{0.95\columnwidth}%
EDPX\index{EDPX} = atmospheric water vapor pressure (mb)\\
ATX\index{ATX} = ambient air temperature ($^{\circ}C$)\\
$e_{s}(\mathrm{ATX})$ = saturation water vapor pressure\index{pressure!water vapor!saturation}
at \emph{dewpoint} ATX (mb)\\
~~~~~~~~~~(see eq. \ref{eq:proposedNewWater}) for the formula
used.)\\
\\
\\
\rule[0.5ex]{1\columnwidth}{1pt}

\begin{equation}
\mathrm{RHUM}=100\%\,\times\,\frac{\mathrm{\{EDPX\}}}{e_{s}(\mathrm{\{ATX\}})}\label{eq:8.16RHUM-1}
\end{equation}
 %
\end{minipage}}\\
To follow normal conventions, the change in saturation vapor pressure
that arises from the enhancement factor\index{enhancement factor}
is not included in the calculated relative humidity, even though the
true relative humidity should include the enhancement factor as specified
in (\ref{eq:EnhancementFactor}) in the denominator of (\ref{eq:8.16RHUM-1}).
\\


\textbf{Absolute Humidity, Water Vapor Density (g/m$^{3}$): }\textbf{\uline{RHOx}}\sindex[var]{RHOx}\index{RHOx}\\
The absolute humidity\index{humidity!absolute} is the water vapor
density\index{density!water vapor}\index{water vapor density|see density, water vapor}
computed from various measurements of humidity as indicated by the
'x' suffix. The calculation proceeds in different ways for different
sensors. For sensors that measure a \index{chilled-mirror}\index{hygrometer!chilled-mirror}chilled-mirror
temperature, the calculation is based on the equation of state for
a perfect gas and uses the water vapor pressure determined by the
instrument, as in the following box.\\
\\
\framebox{\begin{minipage}[t]{0.95\textwidth}%
ATX\index{ATX} = ambient temperature ($^{\circ}C$)\\
EDPX\index{EDPX} = water vapor pressure, converted to consistent
units\\
$C_{mb2Pa}$\sindex[lis]{Cmb2@$C_{mb2Pa}$= conversion factor, mb to Pa}
= 100 Pa\,mb$^{-1}$ (conversion factor to MKS units) \\
$C_{kg2g}=$\sindex[lis]{Ckg2@conversion factor, kg to g}$10^{3}$\,g\,kg$^{-1}$
(conversion factor to give final units of g\,m$^{-3}$)\\
$T_{0}$ = 273.15\,K

\noindent \rule[0.5ex]{1\columnwidth}{1pt}
\begin{eqnarray}
\mathrm{RHOx} & = & C_{kg2g}\frac{C_{mb2Pa}\mathrm{\{EDPx\}}}{R_{w}\mathrm{(\{ATX\}+\mbox{\ensuremath{T_{0}}})}}\label{eq:8.17RHO-1}\\
 & = & 216.674\frac{\{\mathrm{EDPx}\}}{(\{\mathrm{ATX\}}+T_{0})}
\end{eqnarray}
%
\end{minipage}}\\
For instruments measuring the vapor pressure density (including the
Lyman-alpha\index{hygrometer!Lyman-alpha} probes and the newer version
called the UV hygrometer\index{hygrometer!UV}), the basic measurement
from the instrument is the water vapor density, \textbf{RHOUV}\sindex[var]{RHOUV}\index{RHOUV}
or\textbf{ RHOLA\sindex[var]{RHOLA}\index{RHOLA}}, determined by
applying calibration coefficients to the measured signals (XUVI\sindex[var]{XUVI}
or VLA\sindex[var]{VLA}). In addition, a slow update to a dew-point
measurement is used to compensate for drift in the calibration. The
algorithm for the UV Hygrometer is as described in the following box;
the processing used for early projects with the Lyman-alpha instruments
is similar but more involved and won't be documented here because
the instruments are obsolete. See \href{http://www.eol.ucar.edu/raf/Bulletins/bulletin9.html}{RAF Bulletin 9}
for the processing used for archived measurements from the Lyman-alpha
hygrometers.\\
\framebox{\begin{minipage}[t]{0.95\columnwidth}%
XUVI = output from the UV Hygrometer, after application of calibration
coefficients\\
DPXC\index{DPXC} = corrected dewpoint from some preferred source,
$^{\circ}$C\\
ATX\index{ATX} = preferred temperature, $^{\circ}$C\\
RHODT\index{RHODT} =water vapor density determined by a chilled-mirror
sensor\\
Tau = time constant for the exponential update (typically 300 s)\\
\rule[0.5ex]{1\columnwidth}{1pt}
\begin{lyxcode}
For~valid~measurements:%
\footnote{defined by DPXC<ATX and valid numbers for XUVI and RHODT%
}~

~~~~Offset~+=~(RHODT-XUVI-Offset)/Tau~\\
RHOUV~=~XUVI~+~Offset\end{lyxcode}
%
\end{minipage}} \\


\textbf{Specific Humidity (g/kg): }\textbf{\uline{SPHUM}}\sindex[var]{SPHUM}\index{SPHUM}\\
The specific humidity\index{humidity!specific} is the mass of water
vapor per unit mass of (moist) air, conventionally measured in g/kg.
\\
\\
\framebox{\begin{minipage}[t]{0.95\textwidth}%
PSXC\index{PSXC} = ambient pressure \\
EDPX\index{EDPX} = ambient water vapor pressure\\
$C_{kg2g}=$$10^{3}$\,g\,kg$^{-1}$ (conversion factor to give
final units of g\,kg$^{-1}$)\\
\\
\rule[0.5ex]{1\columnwidth}{1pt}
\begin{eqnarray}
\mathrm{SPHUM} & = & C_{kg2g}\frac{M_{w}}{M_{d}}(\mathrm{\frac{\{EDPX\}}{\mathrm{\{PSXC\}-(1-\frac{M_{w}}{M_{d}})\{\mathrm{EDPX}\}}}})\label{eq:8.18SPHUM-1}
\end{eqnarray}
%
\end{minipage}}\\
\\


\textbf{Mixing Ratio (g/kg): }\textbf{\uline{MR}}\sindex[var]{MR}\textbf{\index{MR},
}\textbf{\uline{MRCR}}\textbf{\sindex[var]{MRCR}\index{MRCR},
}\textbf{\uline{MRLA}}\textbf{\sindex[var]{MRLA}\index{MRLA},
}\textbf{\uline{MRLA1}}\textbf{\index{MRLA1}, }\textbf{\uline{MRLH}}\sindex[var]{MRLH}\index{MRLH}\\
The mixing ratio\index{water vapor!mixing ratio} is the ratio of
the mass of water to the mass of dry air, conventionally expressed
in units of g/kg. Mixing ratios may be calculated for the various
instruments measuring humidity on the aircraft, and the variable names
reflect the source: MR from the dewpoint \index{hygrometer!dew point}hygrometers,
MRCR from the cryogenic hygrometer\index{hygrometer!cryogenic}, MRLA
from the Lyman-alpha\index{hygrometer!Lyman-alpha} sensor, MRLA1
if there is a second Lyman-alpha sensor, and MRLH from a laser hygrometer\index{hygrometer!TDL}.
The example in the box below is for the case of the dewpoint hygrometers;
others are analogous.\\
\\
\framebox{\begin{minipage}[t]{0.95\textwidth}%
EDPX\index{EDPX} = water vapor pressure (mb)\\
PSXC\index{PSXC} = ambient total pressure\\
$C_{2kg2g}=$$10^{3}$\,g\,kg$^{-1}$ (conversion factor to give
final units of g\,kg$^{-1}$)\\
\\
\rule[0.5ex]{1\columnwidth}{1pt}
\begin{equation}
\mathrm{MR=C_{kg2g}\frac{M_{w}}{M_{d}}\frac{\mathrm{\{EDPX\}}}{(\mathrm{\{PSXC\}-\{EDPX\})}}}\label{eq:8.19MR-1}
\end{equation}
%
\end{minipage}}\\


\textbf{Potential Temperature (K): }\textbf{\uline{THETA}}\sindex[var]{THETA}\index{THETA}\\
The potential temperature\index{temperature!potential}\index{potential temperature|see {temperature, potential}}
in the output data files is the temperature reached if a dry parcel
at the measured pressure and temperature would be compressed or expanded
adiabatically to a pressure of 1000 mb. It does not take into account
the difference in specific heats caused by the presence of water vapor,
and water vapor can change the exponent in the formula below enough
to produce errors of 1\,K or more.\\
\framebox{\begin{minipage}[t]{0.95\columnwidth}%
ATX\index{ATX} = ambient temperature, $^{\circ}C$\\
PSXC\index{PSXC} = ambient pressure (mb)\\
$p_{0}$\sindex[lis]{p0@$p_{0}$= reference pressure equal to 1000 mb}
= reference pressure = 1000 mb\\
$R_{d}$ = gas constant\index{gas constant!dry air} for dry air\\
$c_{pd}$ = specific heat\index{specific heat!constant pressure}
at constant pressure for dry air\\
\\
\rule[0.5ex]{1\columnwidth}{1pt}
\begin{equation}
\mathrm{THETA}=\left(\mathrm{\{ATX\}}+T_{0}\right)\left(\frac{p_{0}}{\mathrm{\{PSXC\}}}\right)^{R_{d}/c_{pd}}\label{eq:8.13THETA-1}
\end{equation}
 %
\end{minipage}}\\
\\


\textbf{Pseudo-Adiabatic Equivalent Potential Temperature (K): }\textbf{\uline{THETAE}}\sindex[var]{THETAE}\index{THETAE}\marginpar{changed 2011}\\
Beginning in 2011, pesudo-adiabatic equivalent potential temperature\index{temperature, pseudo-adiabatic equivalent potential}
is calculated using the method developed by Davies-Jones\index{Davies-Jones}
(2009).%
\footnote{Davies-Jones, R., 2009: On formulas for equivalent potential temperature.
\emph{Mon. Wea. Review, }\textbf{137, }3137--3148.%
} His formula is\\
\begin{equation}
\Theta_{E}=\Theta_{DL}\exp\{\frac{r(L_{0}^{*}-L_{1}^{*}(T_{L}-T_{0})+K_{2}r)}{c_{pd}T_{L}}\}\label{eq:DaviesJonesThetaP-1}
\end{equation}
and\\
\begin{equation}
\Theta_{DL}=T_{K}(\frac{p_{0}}{p_{d}})^{0.2854}\,(\frac{T_{k}}{T_{L}})^{0.28\times10^{-3}r}\label{eq:ThetaDL-1}
\end{equation}
\\
where $T_{K}$ is the absolute temperature (in kelvin) at the measurement
level, $p_{d}$\sindex[lis]{pd@$p_{d}$= partial pressure of dry air}
is the partial pressure of dry air at that level, $p_{0}$ is the
reference pressure (conventionally 1000 hPa), $r$\sindex[lis]{r@$r$= mixing ratio of water vapor}
is the (dimensionless) water vapor mixing ratio, $c_{pd}$ the specific
heat of dry air, $T_{L}$\sindex[lis]{TL@$T_{L}$= temperature!lifted condensation level}
the temperature at the lifted condensation level\index{lifted condensation level}
(in kelvin), and $T_{0}=273.15\,$K. The coefficients in this formula
are: $L_{0}=2.56313\times10^{6}\mathrm{{J\, kg^{-1}}}$, $L_{1}=1754$
J\,kg$^{-1}\,\mathrm{{K}^{-1}}$, and $K_{2}=1.137\times10^{6}$\,J\,kg$^{-1}$.
The asterisks on $L_{0}^{*}$ and $L_{1}^{*}$ indicate that these
coefficients depart from the best estimate of the coefficients that
give the latent heat of vaporization\index{latent heat of vaporization}
of water, but they have been adjusted to optimize the fit to values
obtained by exact integration. Note that, unlike the formula discussed
below that was used prior to 2011, the mixing ratio must be used in
dimensionless form (i.e., kg/kg), \emph{not} with units of g/kg. The
following empirical formula, developed by Bolton\index{Bolton} (1980),%
\footnote{Bolton, D., 1980: The computation of equivalent potential tem- perature.
\emph{Mon. Wea. Rev.,} \textbf{108,} 1046\textendash{}1053. %
} is used to determine $T_{L}$:\\
\begin{equation}
T_{L}=\frac{\beta_{1}}{3.5\ln(T_{K}/\beta_{3})-\ln(\mathrm{e/\beta_{4}})+\beta_{5}}+\beta_{2}\label{eq:TLCL-1}
\end{equation}
where $e$ is the water vapor pressure, $\beta_{1}=2840\, K$, $\beta_{2}=55\,$K,
$\beta_{3}=1$\,K, $\beta_{4}=1$\,hPa, $\beta_{5}=-4.805$. (Coefficients
$\beta_{3}$ and $\beta_{4}$ have been introduced into (\ref{eq:TLCL-1})
only to ensure that arguments to logarithms are dimensionless and
to specify the units that must be used to achieve that.)\\
\\
Prior to 2011, the variable called the equivalent potential temperature%
\footnote{The AMS glossary defines equivalent potential temperature as applying
to the adiabatic process, not the pseudo-adiabatic process; the name
of this variable has therefore been changed.%
}\index{temperature!equivalent potential} in the output data files
was that obtained using the method of Bolton (1980), which used the
same formula to obtain the temperature at the lifted condensation
level ($T_{L}$) and then used that temperature to find the value
of potential temperature\index{temperature!potential} of dry air
that would result if the parcel were lifted from that point until
all water vapor condensed and was removed from the air parcel.  The
formulas used were as follows:\\
\\
\framebox{\begin{minipage}[t]{0.95\textwidth}%
$T_{L}$= temperature at the lifted condensation level, K\\
ATX\index{ATX} = ambient temperature ($^{\circ}C$)\\
EDPX\index{EDPX} = water vapor pressure (mb)\\
MR\index{MR} = mixing ratio (g/kg)\\
THETA\index{THETA} = potential temperature (K)\\
\\
\rule[0.5ex]{1\columnwidth}{1pt}
\[
T_{L}=\frac{2840.}{3.5\ln(\mathrm{\{ATX\}+T_{0}})-\ln(\mathrm{\{EDPX\}})-4.805}+55.
\]
\begin{equation}
\mathrm{THETAE}=\mathrm{\{THETA\}\left(\frac{3.376}{T_{L}}-0.00254\right)(\{MR\})(1+0.00081(\{MR\}))}\label{eq:8.13THETAE-1}
\end{equation}
%
\end{minipage}}\textbf{}\\
Differences vs the new formula are usually minor but can be of order
0.5\,K.\textbf{}\\


\textbf{Virtual Temperature ($\text{�}$C): }\textbf{\uline{TVIR}}\sindex[var]{TVIR}\index{TVIR}\\
The virtual temperature\index{temperature!virtual} is the temperature
of dry air having the same pressure and density as the air being sampled,
and so adjusts for the buoyancy added by water vapor.\marginpar{change 2011}
\\
\framebox{\begin{minipage}[t]{0.95\columnwidth}%
ATX\index{ATX} = ambient temperature, $^{\circ}C$\\
$r$ = mixing ratio, dimensionless {[}kg/kg{]} = \{MR\}/(1000 g/kg)\\
$T_{0}=273.15$\,K\\
\\
\\
\rule[0.5ex]{1\columnwidth}{1pt}

\begin{equation}
\mathrm{TVIR}=(\mathrm{\{ATX\}}+T_{0})\left(\frac{1+\frac{M_{d}}{M_{w}}r}{1+r}\right)-T_{0}\label{eq:814TVIR-1}
\end{equation}
 %
\end{minipage}}\\
\\


\textbf{Virtual Potential Temperature (K): }\textbf{\uline{THETAV}}\sindex[var]{THETAV}\index{THETAV}\\
The virtual potential temperature\index{temperature!virtual potential}
is analogous to the potential temperature except that it is based
on the virtual temperature (TVIR) instead of the ambient temperature
(ATX).\marginpar{change 2011} Dry-adiabatic expansion or compression
to the reference level (1000 hPa) is assumed. As for THETA, use of
dry-air values for the gas constant and specific heat at constant
pressure can lead to significant errors in humid conditions. For further
information, see this note. (XXX reference needed)\\
\framebox{\begin{minipage}[t]{0.95\columnwidth}%
TVIR\index{TVIR} = virtual temperature, $^{\circ}C$\\
PSXC\index{PSXC} = ambient pressure, hPa\\
$R_{d}=$gas constant for dry air\\
$c_{pd}=$specific heat at constant pressure for dry air (taken to
be exactly $7R_{d}/2)$\\
$T_{0}=273.15$\,K\\
$p_{0}$ = reference pressure, conventionally 1000 hPa

\noindent \rule[0.5ex]{1\columnwidth}{1pt}
\begin{equation}
\mathrm{THETAV}=\left(\mathrm{\{TVIR\}}+T_{0}\right)\left(\frac{p_{o}}{\mathrm{\{PSXC\}}}\right)^{R_{d}/c_{pd}}\label{eq:8.15THETAV-1}
\end{equation}
 %
\end{minipage}}\\


\textbf{Cryogenic Hygrometer Inlet Pressure (mb) and Frost Point Temperature
($^{\circ}C$): }\textbf{\uline{CRHP\sindex[var]{CRHP}\index{CRHP}}}
\textbf{and }\textbf{\uline{VCRH}}\sindex[var]{VCRH}\index{VCRH}
(obsolete)\\
These are measurements made directly in the chamber of the cryogenic
hygrometer\index{hygrometer!cryogenic}, a cabin-mounted instrument
connected to outside air by an inlet line. CRHP is the pressure and
VCRH is the frost-point temperature measured inside that chamber.
VCRH is determined from a third-order calibration equation applied
to the voltage measured by the instrument. 

\textbf{Corrected Cryogenic Frost Point Temperature and Dew Point
Temperature ($\text{�}$C): }\textbf{\uline{FPCRC\sindex[var]{FPCRC}\index{FPCRC}}}\textbf{
and }\textbf{\uline{DPCRC}}\sindex[var]{DPCRC}\index{DPCRC} (obsolete)\\
This is an obsolete instrument but its description is included here
because these variables appear in some old data files. To obtain estimates
of the ambient frost point and dew point, the measurements made inside
the chamber of the cryogenic hygrometer (CVRH and CRHP) must be corrected
for the difference in water vapor pressure between that chamber and
ambient conditions. The ratio of the chamber pressure to the ambient
pressure is assumed to be the same as the ratio of the chamber vapor
pressure to the ambient vapor pressure. The vapor pressure in the
chamber is determined from the Goff-Gratch (1946) equation%
\footnote{Goff, J. A., and S. Gratch (1946) Low-pressure properties of water
from \textminus{}160 to 212 \textdegree{}F, referenced and used in
the Smithsonian Tables (List, 1980).%
} for saturation vapor pressure with respect to a plane ice surface.
This vapor pressure is then used with CRHP and a measure of the ambient
pressure (PSXC) to determine the vapor pressure in the outside air,
and this is converted to an equivalent dew-point in the same manner
as for the standard variables DPxC. The instrument is only used for
measurements of frost poin\index{frost poin}t less than -15$^{\circ}C$
because it does not function well above that frost point. The steps
are documented below:\\
\framebox{\begin{minipage}[t]{0.9\textwidth}%
VCRH = frost point inside the cryogenic hygrometer ($^{\circ}C$)\\
CRHP = pressure inside the chamber of the cryogenic hygrometer (mb)\\
PSXC = reference ambient pressure (mb)\\
f$_{i}$ = enhancement factor (see Appendix C of Bulletin 9)\\
$F_{1}$($T_{d}$) =Goff-Gratch formula for vapor pressure at dew
point $T_{d}$\\
$F_{2}(T_{f})$ = Goff-Gratch formula for vapor pressure at frost
point $T_{f}$ \\
$T_{3}$ = temperature at the triple point of water = 273.16 K\\
\\
\\
\rule[0.5ex]{1\linewidth}{1pt}

chamber vapor pressure $e_{ic}$ (mb):

\[
e_{ic}=(6.1071\,\mathrm{mb})\times10^{A}
\]


\begin{eqnarray*}
\mathrm{where}\,\,\, A & = & -9.09718\left(\frac{T_{3}}{\mathrm{VCRH}+T_{3}}-1\right)\\
 & + & 3.56654\log_{10}\left(\frac{T_{3}}{\mathrm{VCRH}+T_{3}}\right)\\
 & + & 0.876793\left(1-\frac{\mathrm{VCRH}+T_{3}}{T_{3}}\right)
\end{eqnarray*}


ambient vapor pressure $e_{a}$ (mb):

\[
e_{a}=e_{ic}\left(\frac{\mathrm{PSXC}}{\mathrm{CRHP}}\right)f_{i}
\]


ambient dew and frost point DPCRC and FPCRC: (iterative solution)

\begin{eqnarray*}
e_{a} & = & F_{1}\left(\mathrm{DPCRC}\right)\\
 & = & F_{2}\left(\mathrm{FPCRC}\right)
\end{eqnarray*}

\begin{lyxcode}
\end{lyxcode}
%
\end{minipage}}

\textbf{Voltage Output From the Lyman-alpha Sensor (V): }\textbf{\uline{VLA}}\textbf{\sindex[var]{VLA}\index{VLA},
}\textbf{\uline{VLA1}}\index{VLA1} (obsolete)\\
This variable is the voltage output from the Lyman-alpha absorption
hygrometer\index{hygrometer!Lyman-alpha}, which provides fast-response,
high-resolution measurements of water vapor density. (If a second
sensor is used, a 1 is added to the variable name associated with
the second sensor.) 

\textbf{Voltage Output from the UV Hygrometer (V): XUVI\sindex[var]{XUVI}\index{XUVI}}\\
The UV Hygrometer\index{hygrometer!UV} is a modern (as of 2009) version
of the Lyman-alpha hygrometer, which provides a signal that represents
water vapor density. The instrument also provides measurements of
pressure and temperature inside the sensing cavity; they are, respectively,
\textbf{XUVP\sindex[var]{XUVP}\index{XUVP}} and \textbf{XUVT\sindex[var]{XUVT}\index{XUVT}}.

\textbf{Calculated Surface Pressure (mb): }\textbf{\uline{PSURF}}\sindex[var]{PSURF}\index{PSURF}\\
The estimated surface pressure is calculated from HGM\index{HGM}
(a radar-altimeter measurement of height), TVIR\index{TVIR}, PSXC\index{PSXC},
and MR\index{MR} using the thickness equation\index{equation!thickness}.
The average temperature for the layer is obtained by using HGM and
assuming a dry-adiabatic lapse rate from the flight level to the surface.
Because of this assumption, the result is only valid for flight in
a well-mixed surface layer or in other conditions in which the temperature
lapse rate matches the dry-adiabatic lapse rate.\\
\framebox{\begin{minipage}[t]{0.95\columnwidth}%
PSXC\index{PSXC} = ambient pressure (hPa)\\
HGM\index{HGM} = (radar) altitude above the surface (m)\\
TVIR\index{TVIR} = virtual temperature ($^{\circ}C$)\\
PSURF\index{PSURF} = estimated surface pressure (hPa)\\
$g$ = acceleration of gravity\\
$R_{d}$ = gas constant for dry air\\
$c_{pd}$ = specific heat of dry air at constant pressure\\
\\
\rule[0.5ex]{1\columnwidth}{1pt}

\[
T_{m}=(\mathrm{\{TVIR\}}+T_{0})+0.5\mathrm{\{HGM\}}\frac{g}{c_{pd}}
\]
\begin{equation}
\mathrm{PSURF}=\mathrm{\{PSXC\}}\,\exp\left\{ \frac{g\,\{\mathrm{HGM}\}}{R_{d}T_{m}}\right\} \label{eq:PSURF-1-1}
\end{equation}
%
\end{minipage}}\end{hangparagraphs}



\section{OBSOLETE VARIABLES\label{sec:OBSOLETE-VARIABLES}}

RAF retired the ``GENPRO'' processor, the software program previously
used to produce data sets, in 1993, but data files produced by that
processor are still retained and available for use. Also, there are
some instruments that are now retired but provided measurements in
some archived data files. Obsolete variable names that are associated
only with GENPRO or a retired instrument are discussed below, for
reference and to facilitate use of old data files. 
\begin{hangparagraphs}
\textbf{Unaltered Tape Time (s): TPTIME}\sindex[var]{TPTIME (obsolete)}\index{TPTIME}\\
This variable is derived by converting the HOUR, MINUTE and SECOND
to elapsed seconds after midnight of the current day. If time increments
to the next day, its value is not reset to zero, but 86400 seconds
are added to produce ever-increasing values for the data set.

\textbf{Processor Time (s): PTIME}\sindex[var]{PTIME (obsolete)}\index{PTIME}\\
This is an internal time variable created by the GENPRO processor.
It represents elapsed seconds after midnight. It differs from TPTIME
in that, after it has been set at the beginning of the data set, it
is incremented internally for each second of data processed. If duplicate
or missing raw data records exist, it can differ from TPTIME. It is
guaranteed to be a monotonically increasing and continuous series
of values.

\textbf{INS: Data System Time Lag (s): TMLAG}\sindex[var]{TMLAG (obsolete)}\index{TMLAG}\\
TMLAG is the amount of time between the reference time of a Litton
LTN-5l Inertial Navigation System (INS) and the data system clock,
in seconds. TMLAG will always be greater than zero and less than 2.

\textbf{}%
\begin{minipage}[t]{1\columnwidth}%
\textbf{LORAN-C Latitude (}$\text{�}$\textbf{): }\textbf{\uline{CLAT}}\textbf{\sindex[var]{CLAT (obsolete)}\index{CLAT}}\\
\textbf{LORAN-C Longitude (}$\text{�}$\textbf{): }\textbf{\uline{CLON}}\textbf{\sindex[var]{CLON (obsolete)}\index{CLON}}\\
\textbf{LORAN-C Circular Error of Probability (n mi): }\textbf{\uline{CCEP}}\textbf{\sindex[var]{CCEP (obsolete)}\index{CCEP}}\\
\textbf{LORAN-C Ground Speed (m/s): }\textbf{\uline{CGS}}\textbf{\sindex[var]{CGS (obsolete)}\index{CGS}}\\
\textbf{LORAN-C Time (s): }\textbf{\uline{CSEC}}\textbf{\sindex[var]{CSEC (obsolete)}\index{CSEC}}\\
\textbf{LORAN-C Fractional Time (s): }\textbf{\uline{CFSEC\sindex[var]{CFSEC (obsolete)}\index{CFSEC}}}%
\end{minipage}\\
Before the advent of GPS, NCAR/RAF operated a LORAN-C receiver that
provided information on the position and groundspeed of the aircraft.
The measurements of latitude and longitude from this system are CLAT
and CLON, measured at 1 Hz and with positive values of longitude to
the east and positive values of latitude to the north. and CCEP provides
an estimate of the uncertainty in those measurements (in units of
nautical miles). A status word, CSTAT, was used to record a value
of 15 when the system was operational. The ground speed and reference
times were also recorded in the above corresponding variables. The
sum of CSEC and CFSEC represented the time of the measurement, which
was not always the time in the data file when the measurements were
recorded, 

\begin{minipage}[t]{1\columnwidth}%
\textbf{INS Latitude (}$\text{�}$\textbf{): }\textbf{\uline{ALAT}}\textbf{\sindex[var]{ALAT (obsolete)}\index{ALAT}}\\
\textbf{INS Longitude (}$\text{�}$\textbf{): }\textbf{\uline{ALON}}\textbf{\sindex[var]{ALON (obsolete)}\index{ALON}}\\
\textbf{Raw INS Ground Speed X Component (m/s): }\textbf{\uline{XVI}}\sindex[var]{XVI (obsolete)}\textbf{\index{XVI}}\\
\textbf{Raw INS Ground Speed Y Component (m/s): }\textbf{\uline{YVI}}\textbf{\sindex[var]{YVI (obsolete)}\index{YVI}}\\
\textbf{Raw INS True Heading (}$\text{�}$\textbf{): }\textbf{\uline{THI}}\textbf{\sindex[var]{THI (obsolete)}\index{THI}}\\
\textbf{INS Wander Angle (}$\text{�}$\textbf{): }\textbf{\uline{ALPHA}}\textbf{\sindex[var]{ALPHA (obsolete)}\index{ALPHA}}\\
\textbf{INS Platform Heading (}$\text{�}$\textbf{): }\textbf{\uline{PHDG\sindex[var]{PHDG (obsolete)}\index{PHDG}}}%
\end{minipage}

~~~~~~~~~These variables from the Litton LTN-51 Inertial
Navigation System\index{Litton LTN-51 INS} (INS) are analogous to
the modern variables discussed in section \ref{sec:INS}. The measurements
of latitude and longitude were provided with 1-Hz frequency and had
a resolution of 0.0014$\text{�}$, while the ground speed components
were provided at 10 Hz and had resolution equal to 0.012 m/s. The
X component of the ground speed was along the longitudinal axis of
the aircraft \emph{at the time of alignment,}  and the Y axis was
in the starboard direction at the time of alignment. PHDG recorded
the orientation of the platform relative to true north, with resolution
0.0028$\text{�}$. THI was the true heading of the aircraft, produced
at 5 Hz with resolution of 0.0014$\text{�}$. The ``wander angle''
is an INS-only variable that recorded the angle of the INS platform
x-axis relative to its original orientation; it ``wandered'' in
response to east-west motion of the aircraft on a spherical Earth. 

\textbf{Raw Aircraft Vertical Velocity (m/s): }\textbf{\uline{VZI}}\sindex[var]{VZI (obsolete)}\index{VZI}\\
This is an integrated output from an up/down binary counter connected
to the INS vertical accelerometer. Resolution is 0.012 m/s. Due to
changes in local gravity and accumulated errors, this often develops
a significant offset during flight. 

\textbf{Aircraft True Heading (}$\text{�}$\textbf{): }\textbf{\uline{THF}}\sindex[var]{THF (obsolete)}\index{THF}\\
This measurement of aircraft heading was derived from the angle between
the horizontal projection of the aircraft center and true north: THF
= PHDG + ALPHA. Resolution is 0.0028$\text{�}$. 

\begin{minipage}[t]{1\columnwidth}%
\textbf{Aircraft Ground Speed (m/s): }\textbf{\uline{GSF}}\textbf{\sindex[var]{GSF (obsolete)}\index{GSF@\textbf{GSF}}}\\
\textbf{Aircraft Ground Speed East Component (m/s): }\textbf{\uline{VEW}}\textbf{\sindex[var]{VEW (obsolete)}\index{VEW}}\\
\textbf{Aircraft Ground Speed North Component (m/s): }\textbf{\uline{VNS\sindex[var]{VNS (obsolete)}\index{VNS}}}%
\end{minipage}

~~~~~~~~~These variables have the same names as the modern
variables for ground speed. (Cf.~section \ref{sec:INS}.) GSF is
the magnitude of the ground speed determined by the INS, as derived
from XVI and YVI: \\
\[
\mathrm{GSF=\sqrt{\{XVI\}^{2}+\{YVI\}^{2}}}
\]
\\
VEW and VNS are the east and north projections of this ground speed,
derived using THF for the aircraft heading.

\begin{minipage}[t]{1\columnwidth}%
\textbf{Wind Speed (m/s): }\textbf{\uline{WSPD}}\textbf{\sindex[var]{WSPD (obsolete)}\index{WSPD}}\\
\textbf{Wind Direction (}$\text{�}$\textbf{): }\textbf{\uline{WDRCTN\sindex[var]{WDRCTN (obsolete)}\index{WDRCTN}}}%
\end{minipage}

~~~~~~~~~These variables are calculated from UI and VI, the
east and north components of the wind determined as described in RAF
Bulletin No.~23 and summarized in section \ref{sec:WIND}:

\begin{eqnarray*}
\mathrm{WS} & = & \sqrt{\mathrm{\{UI\}^{2}+\{VI\}^{2}}}\\
\mathrm{WD} & = & \mathrm{\frac{180^{\circ}}{\pi}atan2(-\{UI\},}-\{VI\})+180^{\circ}
\end{eqnarray*}
\\


\textbf{Raw Attack Force (Fixed Vane) (g): }\textbf{\uline{AFIXx}}\sindex[var]{AFIXx (obsolete)}\index{AFIXx}\\
AFIXx is an amplified output from a strain-gage, fixed-vane sensor
mounted in the horizontal plane of the aircraft at the end of a gust
boom. The ``force'' on the vane (calibrated in ``equivalent grams''
at Jefferson County Airport gravity) varies as a function of the aircraft
attack angle and dynamic pressure. Here x refers to left or right.

\textbf{Raw Sideslip Force(Fixed Vane) (g): }\textbf{\uline{BFIXx}}\sindex[var]{BFIXx (obsolete)}\index{BFIXx}\\
BFIXx is an amplified output from a strain-gage, fixed-vane sensor
mounted in the vertical plane of the aircraft at the end of a gust
boom. The ``force'' on the vane (calibrated in ``equivalent grams''
at Jefferson County Airport gravity) varies as a function of the aircraft
sideslip angle and dynamic pressure. Here x refers to top or bottom.

\textbf{Attack Angle (Fixed Vane) (}$\text{�}$\textbf{): }\textbf{\uline{AKFXx}}\sindex[var]{AKFXx (obsolete)}\index{AKFXx}\\
AKFXx is the angle of attack, computed from AFIXx and QCx (either
boom or gust dynamic pressure). An empirically derived function, HSSATK,
is used to determine the attack angle based upon wind tunnel test
data.

\textbf{Sideslip Angle (Fixed Vane) (}$\text{�}$\textbf{): }\textbf{\uline{SSFXx}}\sindex[var]{SSFXx (obsolete)}\index{SSFXx}\\
SSFXx is the sideslip angle, computed from BFIXx, and QCx (either
boom or gust dynamic pressure). An empirically derived function, HSSATK,
is used to determine the sideslip angle based upon wind tunnel test
data.

\begin{minipage}[t]{1\columnwidth}%
\textbf{Dynamic Pressure (Boom) (mb): }\textbf{\uline{QCB}}\textbf{\sindex[var]{QCB (obsolete)}\index{QCB},
}\textbf{\uline{QCBC}}\textbf{\sindex[var]{QCBC (obsolete)}\index{QCBC}}\\
\textbf{Dynamic Pressure (Gust Probe) (mb): }\textbf{\uline{QCG}}\textbf{\sindex[var]{QCG (obsolete)}\index{QCG},
}\textbf{\uline{QCGC\sindex[var]{QCGC (obsolete)}\index{QCGC}}}%
\end{minipage}

~~~~~~~~~These variables, measured by a differential pressure
gauge, record the difference between a pitot (total) pressure and
a static pressure. The QCBC and QCGC values are corrected for local
flow-field distortion. The boom and gust probe measurements referred
to the same aircraft structure. The different designations used for
those measurements specified the transducer used and its location.
In the gust probe dynamic pressure measurement (QCG), a Rosemount
Model 1332 differential pressure transducer was located closer to
the sensor in the gust probe itself, whereas in the boom measurement
(QCB), a Rosemount Model 1221 pressure transducer was typically located
in the aircraft nose.

\textbf{Total Temperature, Reverse Flow ($^{\circ}C$): }\textbf{\uline{TTRF}}\sindex[var]{TTRF (obsolete)}\index{TTRF}\\
TTRF is the recovery temperature from a calibrated NCAR reverse-flow
temperature sensor\index{reverse-flow temperature sensor}, for which
the housing was designed to separate water droplets and protect the
element from wetting in cloud. 

\textbf{Total Temperature (Fast Response) ($^{\circ}C$): }\textbf{\uline{TTKP}}\sindex[var]{TTKP (obsolete)}\index{TTKP}\\
This is the output of recovery temperature from the NCAR fast-response
temperature probe, originally designed by Karl Danninger. (See discussion
of total temperature in section \ref{sub:PTq}.)

\textbf{Ambient Temperature ($^{\circ}C$): }\textbf{\uline{ATRF}}\sindex[var]{ATRF (obsolete)}\index{ATRF}\\
The ambient temperature computed using the NCAR reverse-flow temperature
sensor. (See discussion in Section \ref{sub:PTq} above.)

\textbf{Ambient Temperature (Fast Response) ($^{\circ}C$): }\textbf{\uline{ATKP}}\sindex[var]{ATKP (obsolete)}\index{ATKP}\\
The ambient temperature computed using the fast-response temperature
probe. (See discussion of ambient temperature in section \ref{sub:PTq}.)

\textbf{Raw Cloud Technology (Johnson-Williams) }\\
\textbf{Liquid Water Content ($g/m^{3}$): }\textbf{\uline{LWC}}\sindex[var]{LWC (obsolete)}\index{LWC}\\
This is the raw output of a Johnson-Williams\index{Johnson-Williams}
liquid water content sensor converted to units of grams per cubic
meter. The Johnson-Williams indicator measures the evaporative cooling
caused by the latent heat of vaporization of droplets contacting the
heated sensing element by sensing changes in its resistance as it
cools. Through calibration this resistance is converted to a liquid
water content. A ``compensation'' wire is also mounted in the J-W
sensor, parallel to the droplet stream, to compensate for cooling
effects of the airstream. Typically the instrument is set for a true
airspeed of 200 knots. The instrument must be zeroed in ``cloud-free
air.'' The Johnson-Williams liquid water content sensor is designed
for the cloud droplet spectrum. There is some evidence to indicate
that droplets larger than 30 $\mu m$ are shed before completely vaporizing
on the sensor element. This tends to underestimate the liquid water
content.

\textbf{Corrected Cloud Technology (Johnson-Williams) }\\
\textbf{Liquid Water Content (g/M3): }\textbf{\uline{LWCC}}\sindex[var]{LWCC (obsolete)}\index{LWCC}\\
This is the corrected liquid water content obtained by using the aircraft's
true airspeed after removing the zero offset: LWCC=LWC$U_{a}/U_{ref}$
where $U_{a}$ is the true airspeed of the aircraft and $U_{ref}$
is the true airspeed set on the dial of the instrument. $U_{ref}$
was normally 200 kts = 102.88889 m/s.

\textbf{Water Vapor Pressure (mb): }\textbf{\uline{EDPC}}\sindex[var]{EDPC (obsolete)}\index{EDPC}\\
This is a derived intermediate variable used in the calculation of
several derived thermodynamic variables. The vapor pressure over a
plane water surface is obtained by the method of Paul R. Lowe (1977),
a derived, sixth-order, Chebyshev polynomial fit to the Goff-Gratch
Formulation (1946) as a function of temperature expressed in $^{\circ}C$.
The error is much less than 1\% over the range -50$\text{�}$C to
+50$\text{�}$C. EDPC was calculated using this method for most RAF
research projects between 1993 and 1996. This variable does not have
the enhancement factor applied that is discussed in Appendix C. \\
\\
\framebox{\begin{minipage}[t]{0.9\textwidth}%
A. T $<$ -50 C:\\
\begin{eqnarray*}
\mathrm{EDPC} & = & 4.4685+T(0.27347+T\{6.83811\times10^{-3}\\
 & + & T[8.7094x10^{-5}+T(5.63513x10^{-7}+T\,1.47796\times10^{\mbox{-9}})]\})
\end{eqnarray*}
\\
B. T $>$= -50$\text{�}$C:\\
\begin{eqnarray*}
\mathrm{EDPC} & = & 6.107799961+T\,[0.4436518521+T(0.01428945805\\
 & + & T\{2.650648471\times10^{-4}+T\,[3.031240396\times10^{-6}\\
 & + & T(2.034080948\times10^{-8}+T\,6.136820929\times10^{-11})]\})]
\end{eqnarray*}
%
\end{minipage}}
\end{hangparagraphs}



\section{CLOUD PHYSICS VARIABLES}


\subsection{Measurements of Liquid Water Content}
\begin{hangparagraphs}
\textbf{Raw Output PMS/CSIRO (KING) Liquid Water Content (W): }\textbf{\uline{PLWC}}\textbf{\sindex[var]{PLWC}\index{PLWC},
}\textbf{\uline{PLWC1}}\index{PLWC1}\\
This variable\index{King probe} is the output of a PMS/CSIRO (King)
liquid water probe (in watts). PLWC is the power required to maintain
constant temperature in a heated element as that element is cooled
by convection and evaporation of impinging liquid water. The convective
heat losses are determined by calibration in dry air over a range
of airspeeds and temperatures, so that the remaining power can be
related to the liquid water content. The instrument is described in
\href{http://www.eol.ucar.edu/raf/Bulletins/bulletin24.html}{RAF Bulletin 24}\index{Bulletin 24}.
See PLWCC (which follows) for processing.\\


\textbf{Corrected PMS/CSIRO (KING) Liquid Water Content (g/m$^{3}$):
}\textbf{\uline{PLWCC}}\textbf{\sindex[var]{PLWCC}\index{PLWCC},
}\textbf{\uline{PLWCC1}}\index{PLWCC1}\\
\marginpar{change 2011}This is the corrected liquid water content\index{liquid water content!King probe}
obtained from relating the power consumption required to maintain
a constant temperature to the liquid water content, taking into account
the effect of convective heat losses. The instrument and processing
are described by King et al. (1978)%
\footnote{King, W. D., D. A. Parkin and R. J. Handsworth, 1978 A hot-wire liquid
water device having fully calculable response characteristics. J.
Appl. Meteorol., 17, 1809--1813. See also Bradley, S. G., and W. D.
King, 1979 Frequency response of the CSIRO Liquid Water Probe. J.
Appl. Meteorol., 18, 361--366.%
} and in a note available at this URL: XXX. Because the temperature
of the sensing wire is typically well above the boiling point of water\index{water!boiling point},
the assumption made in processing is that the water collected on the
sensing wire is vaporized at the boiling point $T_{b}$. The boiling
point is represented as a function of pressure as described below.
 \\
\\
\framebox{\begin{minipage}[t]{0.95\textwidth}%
PLWC = total power dissipated by the probe (W)\\
$P_{D}$ = power dissipated\index{King probe!power dissipated} by
the cooling effect of dry air alone\\
$P_{W}$ = power needed to heat and vaporize the liquid water that
hits the probe element\\
$L$ = length of the probe sensitive element\index{King Probe!element dimensions},
typically 0.021\,m\\
$d$= diameter of the probe sensitive element, typically 1.805$\times10^{-3}$m\\
$T_{s}$= sensor temperature\index{King probe!sensor temperature}
(K)\\
$T_{a}$= ambient temperature (K) = ATX+273.15\\
$T_{b}$\sindex[lis]{Tb@$T_{b}$= boiling temperature of water} =
boiling temperature of water (dependent on pressure): 

~~~~~~~~~with $x=\log_{10}(p/(1$hPa)) and $B=1^{\circ}C$,
$T_{b}=B\times10^{(b_{0}+b_{1}x+b_{2}x^{2}+b_{3}x^{3})}$\\
$T_{m}=(T_{a}+T_{s})/2$ = mean temperature for air properties\\
$L_{v}(T_{b})$ = latent heat of vaporization of water\sindex[lis]{Lv@$L_{V}$= latent heat of vaporization of water}
= (2.501+0.00237($T_{a}-T_{0})$)$\times10^{6}$J\,kg$^{-1}$\\
$c_{w}$\sindex[lis]{cw@$c_{w}$= specific heat of liquid water} =
specific heat of water=1875\,J\,kg$^{-1}$K$^{-1}$\\
$U_{a}$ = true airspeed (m/s) = TASX\\
$\lambda_{c}$\sindex[lis]{lambda_{c}
= thermal conductivity, dry air@$\lambda_{c}$= thermal conductivity, dry air} = thermal conductivity\index{thermal conductivity} of dry air (2.38+0.0071($T_{m}-T_{0}$))$\times10^{-2}$J\,m$^{-1}$s$^{-1}$K$^{-1}$\\
$\mu$\sindex[lis]{mu_{a}
= dynamic viscosity of air@$\mu_{a}$= dynamic viscosity of air} =viscosity of air =\index{viscosity} (1.718+0.0049($T_{m}-T_{0}))\times10^{-5}$
kg\,~m$^{-1}$s$^{-1}$\\
$\rho_{a}$\sindex[lis]{rhoa@$\rho_{a}$= density of air} = density
of air = $p/(R_{d}T_{a})$\\
Re\sindex[lis]{Re= Reynolds number} = Reynolds number = $\rho_{a}U_{a}d/\mu_{a}$\\
Nu\sindex[lis]{Nu= Nusselt number} = Nusselt Number relating conduction
heat loss to the total heat loss for dry air: 

~~~~~~~~~~typically Nu=$a_{0}\mathrm{Re}^{a_{1}}$\\
$C_{kg2g}=1000$\sindex[lis]{Ckg2g@$C_{kg2g}$= conversion factor, g to kg}
= grams per kilogram 

~~~~~~~~~~(unit conversion to conventional units for liquid
water content)\\
$\chi$ \sindex[lis]{chi
= liquid water content@$\chi$= liquid water content}= liquid water content (g/m$^{3}$) = PLWCC

\noindent \rule[0.5ex]{1\linewidth}{1pt}
\[
\mathrm{PLWC}=P_{D}+P_{W}
\]
where
\[
P_{D}=\pi\mathrm{Nu}\, L\lambda_{c}(T_{s}-T_{a})
\]
\[
P_{W}=L\, d[L_{v}(T_{b})+c_{w}(T_{b}-T_{a})]\, U_{a}\chi
\]
Result:
\[
\mathrm{PLWCC}=\chi=\frac{C_{kg2g}(\mathrm{\{PLWC\}}-P_{D})}{L\, d\, U_{a}[L_{v}(T_{b})+c_{w}(T_{b}-T_{a})]}
\]
%
\end{minipage}}\\
\\


\textbf{Raw Output Rosemount Icing Detector (V): }\textbf{\uline{RICE}}\sindex[var]{RICE}\index{RICE}\\
A Rosemount 871F\index{Rosemount 871F icing probe} ice-accretion
probe consists of a rod set in vibration by a piezoelectric crystal.
The oscillation frequency of the probe changes with ice loading, so
in supercooled cloud ice accumulates on the sensor, and the change
in oscillation frequency is transmitted as a DC voltage.. When the
probe loads to a trigger point, the probe heats the rod to remove
the ice. Its output voltage is a measure of the mass of the accreted
ice. The rate of voltage change is converted to an estimate of the
supercooled liquid water content\index{liquid water content}, as
described in the next paragraph. \\


\textbf{Derived Supercooled Liquid Water Content (g/m$^{3}$): }\textbf{\uline{SCLWC}}\sindex[var]{SCLWC}\index{SCLWC}
\{obsolete? Couldn't find code\}\\
This variable is the supercooled liquid water content obtained from
the change in accreted mass on the Rosemount 871F ice-accretion probe
over one second. Note that the output is not valid during the probe
deicing cycle. This cycle is apparent in the RICE output (a peak followed
by a decrease to near zero). Supercooled liquid water content is determined
by first calculating a water drop impingement rate which is a function
of the effective surface area, the collection efficiency, the true
airspeed, and the supercooled liquid water content. The impingement
rate obtained is equated to the accreted mass of ice collected by
the probe in one second (empirical voltage/mass relationship). The
resulting equation is solved for supercooled water content.\\
\framebox{\begin{minipage}[t]{0.9\textwidth}%
A = effective surface area of the probe (m$^{2}$)\\
$\Delta t$ = time interval during which an increment of mass accretes
(s)\\
$\Delta m$ = mass of ice accreted on the probe in the time interval
$\Delta t$ (g)\\
$U_{a}$ = true airspeed (m/s)\\
\\
\\
\rule[0.5ex]{1\linewidth}{1pt}

\[
\mathrm{SCLWC}=AU_{a}\frac{\Delta m}{\Delta t}
\]
%
\end{minipage}}
\end{hangparagraphs}

\subsection{Sensors of Individual Particles (1-D Probes)}

The RAF operates a set of hydrometeor detectors\index{hydrometeor detector}
that provide single-dimension measurements (i.e., not images) of individual
particle sizes. \href{http://www.eol.ucar.edu/raf/Bulletins/bulletin24.html}{RAF Bulletin 24}
contains extensive information on the operating principles and characteristics
of some of the older instruments. Here the focus will be on the meanings
of the variables in the archived data files.

Four- and five-character variable names\index{variable names!hydrometeor probes}
shown in this section are generic. The actual names appearing in NIMBUS-generated
production output data sets have appended to them an underscore (\_)
and three more characters which indicate a probe's specific aircraft
mounting location. For example, AFSSP\_RPI is the Total Accumulation
from an FSSP-100 probe mounted on the inboard, right-side pod. The
codes presently in use are given in the following table. For the GV,
there are 12 locations available, characterized by three letters.
The first is the wing (\{L,R\} for \{port,starboard\}), the second
is the pylon (\{I,M,O\} for inboard, middle, outboard\}), the third
is which of the two possible canister locations at the pylon is used
(\{I,O\} for \{inboard, outboard\}).

\noindent \begin{center}
\begin{tabular}{|c|c|c|}
\hline 
\textbf{Code} & \textbf{Location} & \textbf{Aircraft}\tabularnewline
\hline 
\hline 
OBL  & Outboard Left  & C-130Q \tabularnewline
\hline 
IBL  & Inboard Left  & C-130Q \tabularnewline
\hline 
OBR  & Outboard Right  & C-130Q \tabularnewline
\hline 
IBR  & Inboard Right  & C-130Q \tabularnewline
\hline 
LPO  & Left Pod Outboard  & C-130Q \tabularnewline
\hline 
LPI  & Left Pod Inboard  & C-130Q \tabularnewline
\hline 
LPC  & Left Pod Center  & C-130Q \tabularnewline
\hline 
RPO  & Right Pod Outboard  & C-130Q \tabularnewline
\hline 
RPI  & Right Pod Inboard  & C-130Q \tabularnewline
\hline 
RPC  & Right Pod Center  & C-130Q \tabularnewline
\hline 
OBL  & Left Wing  & Electra \tabularnewline
\hline 
IBL  & Left Pylon  & Electra \tabularnewline
\hline 
WDL  & Window Left  & Electra \tabularnewline
\hline 
OBR  & Right Wing  & Electra \tabularnewline
\hline 
IBR  & Right Pylon  & Electra \tabularnewline
\hline 
WDR  & Window Right  & Electra \tabularnewline
\hline 
\{L,R\}\{I,M,O\}\{I,O\} & see discussion above & GV\tabularnewline
\hline 
\end{tabular}
\par\end{center}

The probe type also is coded into each variable's name, sometimes
using four characters, sometimes only one: FSSP-100 (FSSP or F), FSSP-300
(F300 or 3), CDP (CDP or D), UHSAS (UHSAS or U), PCAS (PCAS or P),
OAP-200X (200X or X), OAP-260X (260X or 6) and OAP-200Y (200Y or Y).
Prefix letters are used to identify the type of measurement (A=accumulated
particle counts per time interval per channel, C = concentration per
channel, CONC = Concentration from all channels, DBAR = Mean Diameter,
DISP = Dispersion, PLWC =Liquid Water Content, DBZ = Radar Reflectivity
Factor).\index{hydrometeor probes!table of}

\noindent \begin{center}
\begin{tabular}{|c|c|c|c|c|c|c|}
\hline 
\textbf{\small Generic Name} &  & \textbf{\small Probe} & \textbf{\small Channels} & \textbf{\small Usable} & \textbf{\small Diameter Range} & \textbf{\small Bin Width}\tabularnewline
\hline 
\hline 
{\small FSSP} & {\small F} & {\small FSSP-100} & {\small 0-15} & {\small 1-16} & \multicolumn{2}{c|}{{\small (See FRNG below)}}\tabularnewline
\hline 
UHSAS & U & UHSAS & 0-63 & ?? & \multicolumn{1}{c||}{} & \tabularnewline
\hline 
CDP & D & CDP &  &  & \multicolumn{1}{c||}{} & \tabularnewline
\hline 
{\small F300} & {\small 3} & {\small FSSP-300} & {\small 0-31} & {\small 1-31} & {\small 0.3--20.0 $\mu m$} & {\small variable}\tabularnewline
\hline 
{\small PCAS} & {\small P} & {\small PCAS} & {\small 0-15} & {\small 1-15} & {\small 0.1--3.0 $\mu m$} & {\small variable}\tabularnewline
\hline 
{\small 200X} & {\small X} & {\small OAP-200X} & {\small 0-15} & {\small 1-15} & {\small 40--280 $\mu m$} & {\small 10 $\mu m$}\tabularnewline
\hline 
{\small 260X} & {\small 6} & {\small OAP-260X} & {\small 0-63} & {\small 3-62} & {\small 40-620 $\mu m$} & {\small 10 $\mu m$}\tabularnewline
\hline 
{\small 200Y} & {\small Y} & {\small OAP-200Y} & {\small 0-15} & {\small 1-15} & {\small 300--4500 $\mu m$} & {\small 300 $\mu m$}\tabularnewline
\hline 
\end{tabular}\\

\par\end{center}
\begin{hangparagraphs}
\textbf{Total Accumulation (counts per time interval): }\textbf{\uline{}}\\
\textbf{\uline{AFSSP}}\textbf{\sindex[var]{AFSSP}\index{AFSSP},
}\textbf{\uline{AF300}}\sindex[var]{AF300}\textbf{\index{AF300},
}\textbf{\uline{APCAS}}\textbf{\sindex[var]{APCAS}\index{APCAS},
}\textbf{\uline{A200X}}\textbf{\sindex[var]{A200X}\index{A200X},
}\textbf{\uline{A260X}}\textbf{\sindex[var]{A260X}\index{A260X},
}\textbf{\uline{A200Y}}\sindex[var]{A200Y}\index{A200Y}, \textbf{\uline{ACDP}}\sindex[var]{ACDP}\index{ACDP},
\textbf{\uline{AUHSAS\sindex[var]{AUHSAS}\index{AUHSAS}}}\\
\\
This measurement is the total number of particles detected by a PMS-1D
probe per unit time. These measurements have ``vector'' character
in the NetCDF\index{NetCDF} output files, with dimension equal to
the number of usable channels in the table above and one entry per
channel. 

\textbf{}%
\begin{minipage}[t]{1\columnwidth}%
\textbf{Concentration (per channel) ($cm^{-3}$): }\textbf{\uline{CFSSP}}\textbf{\index{CFSSP},
}\textbf{\uline{CF300}}\textbf{\index{CF300}, }\textbf{\uline{CPCAS\index{CPCAS},
CCDP\index{CCDP}, CUHSAS\index{CUHSAS}}}\\
\textbf{Concentration (per channel) ($liter^{-1}$): }\textbf{\uline{C200X}}\textbf{\index{C200X},
}\textbf{\uline{C260X}}\textbf{\index{C260X}, }\textbf{\uline{C200Y\index{C200Y}}}%
\end{minipage}\textbf{}\\
\textbf{}\\
These measurements give the particle concentrations\index{concentration!hydrometeor}\index{FSSP-100!size distribution}
in each usable bin of the probe. They have ``vector'' character
is the NETCDF output files, with dimension equal to the number of
usable channels in the table above and with one entry per channel.
For the scattering spectrometer\index{spectrometer!hydrometeor}\index{hydrometeor spectrometer}
probes (FSSP-100, FSSP-300\index{FSSP-300}, PCAS\index{PCAS}, CDP\index{CDP},
UHSAS\index{UHSAS}) the concentration value is modified by the probe
activity (FACT, PACT) as described below.\index{200X 1D probe}\index{260X 1D probe}\index{200Y 1D probe}
The concentration is obtained from the total number of particles detected
and a calculated, probe-dependent sample volume. For details, see
\href{http://www.eol.ucar.edu/raf/Bulletins/bulletin24.html}{RAF Bulletin 24}\index{Bulletin 24}.

\textbf{}%
\begin{minipage}[t]{1\columnwidth}%
\textbf{Concentration, sum over all channels ($cm^{-3}$): }\textbf{\uline{CONCF}}\textbf{\sindex[var]{CONCF}\index{CONCF},
}\textbf{\uline{CONC3}}\sindex[var]{CONC3}\textbf{\index{CONC3},
}\textbf{\uline{CONCP\sindex[var]{CONCP}\index{CONCP}, CONCD\sindex[var]{CONCD}\index{CONCD},
CONCU\sindex[var]{CONCU}\index{CONCU}}}\\
\textbf{Concentration, sum over all channels ($liter^{-1}$): }\textbf{\uline{CONCX}}\textbf{\sindex[var]{CONCX}\index{CONCX},
}\textbf{\uline{CONC6}}\sindex[var]{CONC6}\textbf{\index{CONC6},
}\textbf{\uline{CONCY\sindex[var]{CONCY}\index{CONCY}}}%
\end{minipage}\textbf{}\\
\textbf{}\\
These measurements are the particle concentrations\index{FSSP-100!concentration}
summed over all channels to give the total particle concentration
in the size range of the probe. For details, see \href{http://www.eol.ucar.edu/raf/Bulletins/bulletin24.html}{RAF Bulletin 24}.\\


\textbf{Mean Diameter ($\mu m$): }\textbf{\uline{DBARF}}\textbf{\index{DBARF},
}\textbf{\uline{DBAR3}}\textbf{\index{DBAR3}, }\textbf{\uline{DBARP}}\textbf{\index{DBARP},
}\textbf{\uline{DBARX}}\textbf{\index{DBARX}, }\textbf{\uline{DBAR6}}\textbf{\index{DBAR6},
}\textbf{\uline{DBARY}}\index{DBARY}, \textbf{\uline{DBARD}}\index{DBARD},
\textbf{\uline{DBARU}}\index{DBARU}\\
The mean diameter\index{diameter!mean}\index{FSSP-100!mean diameter}
is the arithmetic average of all particle diameters. It is calculated
as follows: \\
\\
\framebox{\begin{minipage}[t]{0.9\textwidth}%
\{Cy$_{i}$\} = concentration\sindex[lis]{Cyi@$Cy_{i}$= concentration from hydrometeor probe y in channel i}
from probe y in channel i\\
\hspace*{0.6cm}(e.g., y=FSSP to calculate DBARF)\\
i1 = lowest usable channel for the probe\\
i2 = highest usable channel for the probe\\
$d_{i}$\sindex[lis]{di@$d_{i}$= diameter of hydrometeor in channel $i$}
= mean diameter of particles in channel i for this probe ($\mu m$)\\
\\
\rule[0.5ex]{1\linewidth}{1pt}

\[
\mathrm{DBARx}=\frac{{\textstyle \sum_{i=i1}^{i2}}{\displaystyle {\displaystyle \left\{ \mathrm{Cy}_{i}\right\} d_{i}}}}{\sum_{i=i1}^{i2}\left\{ \mathrm{Cy}_{i}\right\} }
\]
%
\end{minipage}}\\


\textbf{Dispersion (dimensionless): }\textbf{\uline{DISPF}}\textbf{\sindex[var]{DISPF}\index{DISPF},
}\textbf{\uline{DISP3}}\textbf{\sindex[var]{DISP3}\index{DISP3},
}\textbf{\uline{DISPP}}\textbf{\sindex[var]{DISPP}\index{DISPP},
}\textbf{\uline{DISPX}}\textbf{\sindex[var]{DISPX}\index{DISPX},
}\textbf{\uline{DISP6}}\textbf{\sindex[var]{DISP6}\index{DISP6},
}\textbf{\uline{DISPY}}\sindex[var]{DISPY}\index{DISPY}, \textbf{\uline{DISPD}}\sindex[var]{DISPD}\index{DISPD},
\textbf{\uline{DISPU}}\sindex[var]{DISPU}\index{DISPU}\\
The dispersion\index{dispersion} is the ratio of the standard deviation
of particle diameters to the mean particle diameter.\index{FSSP-100!dispersion}
\\
\\
\framebox{\begin{minipage}[t]{0.9\textwidth}%
\{DBARx\} = mean particle diameter ($\mu m$)\\
\{Cy$_{i}$\} = concentration from probe y in channel i\\
\hspace*{0.6cm}(e.g., y=FSSP to calculate DISPF)\\
$d_{i}$ = diameter measured in channel $i$ of probe y\\
i1 = lowest usable channel for the probe\\
i2 = highest usable channel for the probe\\
\\
\\
\rule[0.5ex]{1\linewidth}{1pt}

\[
\mathrm{DISPx}==\frac{1}{\{\mathrm{DBARx}\}}\,\left\{ \frac{{\textstyle \sum_{i=i1}^{i2}}{\displaystyle {\displaystyle \left\{ \mathrm{Cy}_{i}\right\} d_{i}^{2}}}}{\sum_{i=i1}^{i2}\left\{ \mathrm{Cy}_{i}\right\} }-\{\mathrm{DBARx}\}^{2}\right\} ^{1/2}
\]
%
\end{minipage}}\\


\textbf{Liquid/Ice Water Content ($g/m{}^{3}$): }\textbf{\uline{}}\\
\textbf{\uline{PLWCF}}\textbf{\sindex[var]{PLWCF}\index{PLWCF},
}\textbf{\uline{PLWCX}}\textbf{\sindex[var]{PLWCX}\index{PLWCX},
}\textbf{\uline{PLWC6}}\textbf{\sindex[var]{PLWC6}\index{PLWC6},
}\textbf{\uline{PLWCY}}\sindex[var]{PLWCY}\index{PLWCY}, \textbf{\uline{PLWCD}}\sindex[var]{PLWCD}\index{PLWCD}\\
\index{liquid water content}\index{FSSP-100!liquid water content}These
variables are derived from the measured concentration (CONCx) and
the third moment of the equivalent droplet diameter. The equivalent
droplet diameter\index{diameter!equivalent} is the diameter that
represents the mass in the detected particle. The equivalent droplet
diameter is normally the measured diameter for liquid hydrometeors,
but some processing has used other assumptions and this is a choice
that can be made based on project needs. Using this definition allows
for the approximate estimation of \index{ice water content}ice water
content in cases where it is known that all hydrometeors are ice.
\\
\\
:\\
\framebox{\begin{minipage}[t]{0.9\textwidth}%
$d_{e,i}$\sindex[lis]{dei@$d_{e,i}$= equivalent melted diameter for channel i of a hydrometeor
spectrometer} = equivalent melted diameter for channel $i$ of probe x\\
\{Cy$_{i}$\} = concentration from probe y in channel i\\
\hspace*{0.6cm}(e.g., y=FSSP to calculate DISPx for x=F)\\
$\varrho_{w}$\sindex[lis]{rhow@$\rho_{w}$= density of liquid water}
= density of water ( $10{}^{3}kg/m^{3}$)\\
i1 = lowest usable channel for the probe\\
i2 = highest usable channel for the probe\\
\\
\rule[0.5ex]{1\linewidth}{1pt}
\[
\mathrm{PLWCx}=\frac{\pi\varrho_{w}}{6}{\textstyle \sum_{i=i1}^{i2}}{\displaystyle {\displaystyle \left\{ \mathrm{Cy}_{i}\right\} d_{e,i}^{3}}}
\]
(units and a scale factor are selected so that the output variable
is in units of $g/m^{3}$)%
\end{minipage}}\\


\textbf{Radar Reflectivity Factor (dbZ): }\textbf{\uline{DBZF}}\textbf{\sindex[var]{DBZF}\index{DBZF},
}\textbf{\uline{DBZX}}\textbf{\sindex[var]{DBZX}\index{DBZX},
}\textbf{\uline{DBZ6}}\textbf{\sindex[var]{DBZ6}\index{DBZ6},
}\textbf{\uline{DBZY}}\sindex[var]{DBZY}\index{DBZY}, \textbf{\uline{DBZD\sindex[var]{DBZD}\index{DBZD}}}\\
The radar reflectivity factor\index{reflectivity factor}\index{dBz}\index{FSSP-100!radar reflectivity factor}
for water is a measure of the product of the concentration and the
sixth moment of the droplet diameter. An equivalent radar reflectivity
factor can be calculated from the hydrometeor size distribution if
an assumption about composition of the particles is made, but this
variable is not part of normal data files. The radar reflectivity
factor is a characteristic only of the hydrometeor size distribution;
it is \emph{not }a measure of radar reflectivity, because the latter
also depends on wavelength, dielectric constant, and other characteristics
of the hydrometeors. The normally used radar reflectivity factor is
measured on a logarithmic scale that depends on a particular choice
of units, so (although it is not conventionally included) an appropriate
scale factor $Z_{r}$ is included in the following equation to satisfy
the convention that arguments of logarithms should be dimensionless\index{dimensionless equations}.
\\
\framebox{\begin{minipage}[t]{0.9\textwidth}%
$d_{i}$ = diameter for channel $i$ of probe x\\
\{Cy$_{i}$\} = concentration from probe y in channel i\\
\hspace*{0.6cm}(e.g., y=FSSP to calculate DISPx for x=F)\\
i1 = lowest usable channel for the probe\\
i2 = highest usable channel for the probe\\
$Z_{r}$ = reference factor for units = $1\, mm^{6}/m^{3}$\\
\\
\rule[0.5ex]{1\linewidth}{1pt}

\[
\mathrm{DBZx}=10\log_{10}\left({\textstyle \frac{1}{Z_{r}}\sum_{i=i1}^{i2}}{\displaystyle {\displaystyle \left\{ \mathrm{Cy}_{i}\right\} d_{i}^{6}}}\right)
\]
%
\end{minipage}}



\textbf{FSSP-100 Range (dimensionless): }\textbf{\uline{FRNG}}\textbf{\sindex[var]{FRNG}\index{FRNG},
}\textbf{\uline{FRANGE}}\sindex[var]{FRANGE}\index{FRANGE}\\
This variable records the size range in use for the FSSP-100\index{FSSP-100!range}
probe\\


\noindent %
\begin{minipage}[t]{0.9\textwidth}%
\noindent \hspace*{0.7in}%
\begin{tabular}{|c|c|c|}
\hline 
Range & \textbf{Nominal Size Range} & \textbf{Nominal Bin Width}\tabularnewline
\hline 
\hline 
0 & 2--47 $\mu m$ & 3 $\mu m$\tabularnewline
\hline 
1 & 2--32 $\mu m$ & 2 $\mu m$\tabularnewline
\hline 
2 & 1--15 $\mu m$ & 1 $\mu m$\tabularnewline
\hline 
3 & 0.5--7.5 $\mu m$ & 0.5 $\mu m$\tabularnewline
\hline 
\end{tabular}%
\end{minipage}\\
\\
In recent NETCDF data files, the actual bin boundaries used for processing
are recorded in the header. That header should be consulted because
processing sometimes uses non-standard sizes selected to adjust for
Mie scattering, which causes departures from the nominal linear bins.
\\


\textbf{FSSP-100 Fast Resets (number per sample interval): }\textbf{\uline{FRST}}\textbf{\sindex[var]{FRST}\index{FRST},
}\textbf{\uline{FRESET}}\textbf{\sindex[var]{FRESET}\index{FRESET}}\\
The FSSP\index{FSSP-100!fast resets} records events called ``fast
resets'' that occur when a particle traverses the beam outside the
depth-of-field and therefore is not accepted for sizing. To avoid
the processing time associated with sizing, the probe resets quickly
in this case, but there is still some dead time\index{FSSP-100!dead time}
when the probe cannot record another event. Fast resets consume a
time determined by circuit characteristics, so that time is determined
in laboratory tests of the FSSP circuitry. This variable is needed
in addition to the ``Total Stobes'' to determine what fraction of
the time the probe is unable to accept another particle, and this
``dead time'' enters calculation of the concentration. \\


\textbf{FSSP-100 Total Strobes (number per sample interval): }\textbf{\uline{FSTB}}\textbf{\sindex[var]{FSTB}\index{FSTB},
}\textbf{\uline{FSTROB}}\sindex[var]{FSTROB}\index{FSTROB}\\
A ``strobe'' is generated in the FSSP\index{FSSP-100!total strobes}
whenever a particle is detected within its depth-of-field. Not all
such particles are accepted for inclusion in the size distribution,
however, because some pass through the outer regions of the illuminating
laser beam and therefore produce shorter and smaller-amplitude pulses
than those passing through the center of the beam. The probe maintains
a running estimate of the average transit time and rejects particles
with transit times shorter than this average. The total number of
strobes recorded is therefore more than the number of sized particles,
and the ratio of strobes to accepted particles can indicate quality
of operation of the probe. Also, the strobes require processing and
so contribute to the dead time of the probe, affecting the concentration
unless a correction is made. See \href{http://www.eol.ucar.edu/raf/Bulletins/bulletin24.html}{RAF Bulletin 24}\index{Bulletin 24}
for more discussion on the operation of the FSSP.\\


\textbf{FSSP-100 Beam Fraction (dimensionless): }\textbf{\uline{FBMFR}}\sindex[var]{FBMFR}\index{FBMFR}\\
This variable\index{FSSP-100!beam fraction} records the ratio of
the number of velocity-accepted particles (particles that pass through
the effective beam diameter) to the total number of particles detected
in the depth-of-field of the beam (the total strobes). See the discussion
of Total Strobes for more information.\\
 \\
\framebox{\begin{minipage}[t]{0.9\textwidth}%
AFSSP = valid particles sized per sample interval\\
FSTROB = strobes generated by particles in the depth-of-field, \\
\hspace*{0.7in}per sample interval\\
\\
\rule[0.5ex]{1\linewidth}{1pt}
\[
\mathrm{FBMFR=\{AFSSP\}/\{FSTROB\}}
\]
%
\end{minipage}}\\


\textbf{FSSP-100 Calculated Activity Fraction (dimensionless): }\textbf{\uline{FACT}}\index{FACT}\\
This variable\index{FSSP-100!activity} represents the fraction of
the time that the FSSP is unable to count and size particles (its
``dead time\index{dead time!FSSP}\index{FSSP-100!dead time}'').
The activity fraction is not measured directly but is estimated from
fast resets and total strobes along with measurements of the dead
times associated with each (as determined in laboratory tests). The
characteristic times are in the NetCDF header (for recent projects).
.\\
\framebox{\begin{minipage}[t]{1\columnwidth}%
FSTROB = strobes generated by particles in the depth-of-field, \\
\hspace*{0.7in}per sample interval\\
FRESET = ``fast resets'' generated per sample interval\\
$t_{1}$ = slow reset time (for each strobe)\\
$t_{2}$ - fast reset time (for each fast reset)\\
\\
\rule[0.5ex]{1\linewidth}{1pt}

\[
FACT=\{FSTROB\}\, t_{1}+\mathrm{\{FRESET\}}\, t_{2}
\]
%
\end{minipage}}\\
\\


\textbf{PCAS Raw Activity (dimensionless); }\textbf{\uline{AACT}}\uline{\sindex[var]{AACT}\index{AACT},
}\textbf{\uline{PACT}}\sindex[var]{PACT}\index{PACT}\\
The PCAS probe provides this measure of dead time, the time that the
probe is unable to sample particles because the electronics are occupied
with processing particles. The manufacturer suggests that the actual
dead time ($f_{PCAS}$) is given by the following formula, which is
used in determining concentrations for the PCAS:\\
\[
f_{PCAS}=0.52\frac{\mathrm{\{PACT\}}}{F_{PCAS}}
\]
\\
where $F_{PCAS}=1024\, s^{-1}$. However, PACT (or AACT) is the variable
archived in the data files. \\


\textbf{PMS-2D Cloud Probe Particle Concentration ($L^{-1}$): }\textbf{\uline{CON2C1\index{CON2C1}}}
\\
This concentration of all particles sensed by the PMS-2D Cloud Probe
is based on the ``shadow-or'' count (SDWC1,SHDORC) from the probe.
This counter is triggered each time a particle passes through the
laser beam, so the rate at which these counts are produced can be
used with the sample volume of the probe and the flight speed to determine
this upper estimate of the particle concentration. \\


\textbf{PMS-2D Precip Probe Particle Concentration ($L^{-1}$): }\textbf{\uline{CON2P1}}\index{CON2P1}\\
This measurement is based on SDWP1 or SHDORP and is analogous to CON2C1
but for the PMS-2D Precip Probe. 
\end{hangparagraphs}

\subsection{Hydrometeor Imaging Probes}

-- NEEDED: 2DC, 2DP, CPI, etc


\section{RADIATION VARIABLES}


\subsection{Measurements of Irradiance\index{Irradiance} and Radiometric Temperature\index{Temperature!radiometric}}
\begin{hangparagraphs}
\textbf{Radiometric (Surface or Sky/Cloud-Base) Temperature ($^{\circ}C$):
}\textbf{\uline{RSTx}}\sindex[var]{RSTx}\index{RSTx}\\
Radiomatric temperature is the equivalent black body temperature measured
by one of two infrared radiometers. The x denotes either that the
instrument is mounted on the bottom (B) or top (T) of the aircraft.
Both of these instruments are calibrated using a black-body source
manufactured by Eppley. The measurements may come from either of the
following two instruments:\\
\begin{minipage}[t]{1\columnwidth}%
\begin{itemize}
\item a narrow bandwidth, narrow field-of-view (2$\text{�}$) Heimann Model
KT-19.85 precision radiation thermometer. The wavelength range is
9.6 to 11.5 $\mu m$.
\item a narrow bandwidth, narrow field-of-view (2$\text{�}$) Barnes Engineering
Model PRT-5 precision radiation thermometer. This instrument is now
retired. The spectral bandwidth available was either 8 to 14 $\mu m$
or 9.5 to 11.5 $\mu m$. Its cavity temperature was monitored and
recorded as either TCAVB or TCAVT.\end{itemize}
%
\end{minipage}\\


\textbf{Radiometer Sensor Head Temperature ($^{\circ}C$): }\textbf{\uline{TRSTB}}\sindex[var]{TRSTB}\index{TRSTB}\\
 This is the temperature of the sensing head of the Heimann radiometer\index{Heimann radiometer},
usually from RSTB, the primary down-looking instrument.\\


\textbf{Raw Pyrgeometer Output (W\,m$^{-2}$): }\textbf{\uline{IRx}}\sindex[var]{IRx}\index{IRx}\\
A pyrgeometer\index{pyrgeometer} manufactured by Eppley Laboratory,
Inc. measures \index{radiation!long-wave}long-wave irradiance using
a calibrated thermopile. It has a coated glass hemisphere that transmits
radiation in a bandwidth between 3.5 $\mu m$ and 50 $\mu m$. It
is calibrated at RAF according to procedures specified by Albrecht
and Cox (1977). The pyrgeometers are usually flown in pairs, one
up-looking and one down-looking. The letter 'x' denotes either bottom
(B) or top (T).\\


\textbf{Corrected Infrared Irradiance (W\,m$^{-2}$): }\textbf{\uline{IRxC}}\sindex[var]{IRxC}\index{IRxC}\\
Because the pyrgeometer measures net radiation, IRx must be corrected
for emission from the dome covering the sensor and for emission from
the thermopile itself. IRxC is the corrected infrared irradiance,
determined following procedures of Albrecht and Cox (1977). . \marginpar{change 2011}\\
\framebox{\begin{minipage}[t]{0.9\textwidth}%
IRx = raw pyrgeometer output {[}W\,m$^{-2}${]}\\
$T_{D}$ = dome temperature {[}K{]}\\
$T_{S}$ = ``sink'' temperature (approx.~the thermopile temperature)
{[}K{]}\\
$\epsilon$ = emissivity of the thermopile (dimensionless) = 0.986\\
$\beta$ = empirical constant dependent on the dome type = 5.5\\
$\sigma$ = Stephan-Boltzmann constant = 5.6704$\times10^{-8}$ W\,m$^{-2}$K$^{-4}$\\
\\
\rule[0.5ex]{1\linewidth}{1pt}
\[
\mathrm{IRxC}=\mathrm{IRx}-\beta\sigma(T_{D}^{4}-T_{S}^{4})+\epsilon\sigma T_{S}^{4}
\]
%
\end{minipage}}\\


\textbf{Shortwave Irradiance (W/m$^{2}$): }\textbf{\uline{SWx}}\sindex[var]{SWx}\index{SWx}\\
An Eppley Laboratory, Inc., pyranometer\index{pyranometer} measures
\index{radiation!short wave}short-wave irradiance. The dome normally
used is UG295 glass, which gives wide coverage of the solar spectrum
(from 0.285 $\mu m$ to 2.8 $\mu m$). Different bandwidths can be
obtained by use of different glass domes, available from RAF upon
request. (See Bulletin No. 25.) The pyranometers are usually flown
in pairs, one up-looking and one down-looking. They are calibrated
periodically at the NOAA Solar Radiation Facility in Boulder, Colorado.
The letter 'x' denotes either bottom (B) or top (T).

\textbf{Corrected Incoming Shortwave Irradiance (W/m$^{2}$): }\textbf{\uline{SWTC}}\sindex[var]{SWTC}\index{SWTC}\\
The down-welling shortwave irradiance measured by the difference between
SWT and SWB) is corrected to take into account the sun angle and small
variations in the aircraft attitude angles (pitch and roll). The correction
is limited to $\pm6^{\circ}$ in either angle, so these measurements
should be considered invalid beyond these limits. This is the derived
output of incoming (down-welling) shortwave irradiance, taking into
account both solar position (sun angle) and modest variations in aircraft
attitude (at present, restricted to less than 6$\text{�}$ in pitch
and/or roll). (For more information, refer to Bulletin No. 25.)

\textbf{Ultraviolet Irradiance (W/m$^{2}$): UVx}\index{UVxUVx@\sindex[var]{UVx}UVx}\\
A pair of UV radiometer/photometers measure either down-welling (x=T)
or up-welling (x=B) irradiance in the ultraviolet, approximately from
0.295 $\mu m$ to 0.385 $\mu m$. These units are periodically returned
to the Eppley Laboratories for recalibration. \\

\end{hangparagraphs}

\subsection{SOLAR ANGLES}

The calculations described in this group are used primarily for deriving
the Corrected Short Wave irradiance (SWTC) but can be used by themselves
or in conjunction with other measurements that need them.\\

\begin{hangparagraphs}
\textbf{Solar Declination Angle (radians): }\textbf{\uline{SOLDE\sindex[var]{SOLDE}\index{SOLDE}}}
\\
This is a calculation of the astronomical measurement of solar declination
angle\index{solar declination angle}, the angular distance of the
sun north or south of the earth's equator. (Positive values are north.)
To obtain this, the solar hour angle is calculated (taking leap years
into account). The calculations were adapted by Ron Ruth from an algorithm
developed by Lutz Bannehr.\\
\\
\framebox{\begin{minipage}[t]{0.9\textwidth}%
time = day number (corrected for leap year) since 1 January 1980\\
theta = coarse solar time (radians)\\
gg = equation-of-time term for calculating declination (radians)\\
el = equation-of-time term for calculating declination (radians)\\
eps = equation-of-time term for calculating declination (radians)\\
\\
\rule[0.5ex]{1\linewidth}{1pt}
\begin{lyxcode}
gg~=~-0.031271~-~(4.53963e-7~time)~+~theta

el~=~4.900968~+~(3.67474e-7~time)~

~~~~~+~\{{[}0.033434~-~(2.3e-9~time){]}~sin(gg)\}~

~~~~~+~{[}0.000349~sin(2~gg)~+~theta{]}

sel~=~sin(el);

eps~=~0.409140~-~(6.2149e-9~time)

SOLDE~=~asin~\{sel~sin(eps)\}

\end{lyxcode}
%
\end{minipage}}\\


\textbf{Solar Zenith Angle (radians): }\textbf{\uline{SOLZE}}\sindex[var]{SOLZE}\index{SOLZE}\\
This is the astronomical measurement of solar zenith angle\index{solar zenith angle},
the angle from zenith to the sun, complementary to the sun's elevation
angle:\\
\\
\framebox{\begin{minipage}[t]{0.9\textwidth}%
lat = latitude (radians)\\
lha = local hour angle (radians)\\
SOLDE = solar declination angle (radians)\\
\\
\rule[0.5ex]{1\linewidth}{1pt}
\begin{lyxcode}
SOLZE~=~asin~\{sin(lat)~sin(SOLDE)~

~~~~~~+~cos(lat)~cos(SOLDE)~cos(lha)\}\end{lyxcode}
%
\end{minipage}}\\


\textbf{Solar Azimuth Angle (radians): }\textbf{\uline{SOLAZ}}\sindex[var]{SOLAZ}\index{SOLAZ}\\
This is the astronomical measurement of solar azimuth angle\index{solar azimuth angle},
the angular distance between due south and the projection of the line
of sight to the sun on the ground. A positive solar azimuth angle
indicates a position east of south (i.e., morning).\\
\\
\framebox{\begin{minipage}[t]{0.95\textwidth}%
lha = local hour angle (radians)\\
SOLDE = solar declination angle (radians)\\
SOLZE = solar zenith angle (radians)\\
\\
\rule[0.5ex]{1\linewidth}{1pt}
\begin{lyxcode}
SOLAZ~=~asin~\{cos(SOLDE)~sin(lha)~/~cos(SOLZE)\}\end{lyxcode}
%
\end{minipage}}\\


\textbf{Solar Elevation Angle (radians): }\textbf{\uline{SOLEL}}\sindex[var]{SOLEL}\index{SOLEL}\\
This is the astronomical measurement of solar elevation angle\index{solar elevation angle},
describing how high the sun appears in the sky. The angle is measured
between an imaginary line between the observer and the sun and the
horizontal plane on which the observer is standing. The altitude angle
is negative when the sun drops below the horizon.\\
\\
\framebox{\begin{minipage}[t]{0.95\textwidth}%
SOLZE = solar zenith angle (radians)\\
\\
\rule[0.5ex]{1\linewidth}{1pt}
\begin{lyxcode}
SOLEL~=~Pi/2~-~SOLZE\end{lyxcode}
%
\end{minipage}}\end{hangparagraphs}


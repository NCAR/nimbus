
\section{Introduction\label{sub:General-Comments}}

This technical report defines the variables used in data sets that
are collected by the research aircraft operated by the Research Aviation
Facility (RAF) of the National Center for Atmospheric Research. Where
appropriate, it also documents the equations that are used by the
processing software (currently ``nimbus''\index{nimbus}) to calculate
the derived measurements that result from the use of one or more other
basic measurements (e.g., potential temperature). Since 1993, data
from research flights have been archived in NetCDF format\index{NetCDF!format}
(cf. \url{http://www.unidata.ucar.edu/software/netcdf/docs/}), and
the NetCDF header\index{NetCDF!header} for recent projects includes
detailed information on the measurements present in the file, how
they depend on other measurements, units, etc.

Before 1993, data were processed by a different program, ``GENPRO,''\index{GENPRO}
and a different output format (also named GENPRO) was used for archived
datasets. Appendix E in \href{http://www.eol.ucar.edu/raf/Bulletins/bulletin9.html}{RAF Bulletin 9}\index{Bulletin 9},
the previous description of RAF data products that is now superseded
by this technical note, describes that format. Some variable names
in this document, esp.~in section \ref{sec:OBSOLETE-VARIABLES},
refer to obsolete variable names, some used with GENPRO and others
referring to instruments that are now retired. These names are included
here so that this report can be a reference for older archived data
as well as for current data files.

In some cases redundant measurements are present, often for key measurements
like pressure or temperature.\index{variable names} When these are
used in subsequent calculation of derived variables like potential
temperature, some choice is usually made regarding which measurement
is considered most reliable for a particular project or flight, and
a single derived variable is calculated on the basis of the chosen
input variable(s). To record which measurements were so designated,
a reference measurement chosen from a group of redundant measurements
usually has a variable name ending with the letter(s) X or XC. %
\footnote{Some that do not follow this convention are ATTACK, SSLIP and EDPC;
see the individual descriptions that follow.%
}

The data system used to record most measurements from research flights
is currently the NCAR In situ Data Acquisition System (NIDAS\index{NIDAS}).
The data system has changed several times over the history of RAF.
For a discussion of the history of the data systems, see \url{ADShistory.pdf},
written by Richard Friesen. The versions of data systems that produced
most of the data still available were, approximately, as given in
the following table:

\begin{minipage}[t]{1\columnwidth}%
\noindent \begin{center}
\begin{tabular}{|c|c|c|>{\centering}p{2.3in}|}
\hline 
\textbf{Data System} & \textbf{start} & \textbf{end} & \textbf{Aircraft}\tabularnewline
\hline 
\hline 
ADS I & 1984 & 1992 & King Air 200T, Sabreliner (1987), Electra (1991)\tabularnewline
\hline 
ADS II & 1992 & 2007 & C-130\tabularnewline
\hline 
ADS III (NIDAS)%
\footnote{ADS III is the name given to the full data system, which includes
these components: NIDAS (for data acquisition and recording); NIMBUS
(for data processing,(both in flight and after the flight); AEROS
(for data display in flight); and the Mission Coordinator Station
and satellite communications system (for transmission of data to and
from the aircraft, display of such data for mission decisions, and
support for written ``chat'' communications among project participants
both on the aircraft and on the ground).%
} & 2005 &  & GV, C-130 (2007)\tabularnewline
\hline 
\end{tabular}
\par\end{center}%
\end{minipage}


\section[GENERAL INFORMATION]{GENERAL INFORMATION ABOUT DATA FILES}


\subsection{Alphabetical List of Variables }

At the end of this section, on page \vpageref{List of Variables},
there is a list of all the variable names\index{variable names} that
appear in standard data files along with links to the primary discussion
of those variables. The \url{index} to this technical report (\vpageref{IndexReference})
also includes all variables described here, and also some variables
not discussed in detail in this document. Where possible, reference
to those variables and information on the project(s) where they were
used have been included also. In cases with multiple references, the
bold entry is the primary discussion of the variable.


\subsection{System of Units\label{sub:System-of-Units}}

This report will use the SI system of units\index{system of units},
but with many exceptions. Among them are the following:
\begin{enumerate}
\item The millibar\index{millibar} (mb), equal to one hectopascal (hPa),
is sometimes used for pressure. 
\item Many variables are presented in the units most often used for that
variable, even when they involve CGS units or mixed CGS-MKS units,
as for example {[}$\mathrm{g\, m}{}^{-3}]$ for liquid water content
or {[}$\mathrm{cm}{}^{-3}${]} for droplet concentration. 
\item Flow rates\index{Flow rates} are often quoted in liters per minute
(LPM) or standard liters per minute (SLPM) because those terms are
linked to properties of commercially available instruments with flow
control. One liter is $10^{-3}\,\mathrm{m}^{3}$. Because there is
considerable ambiguity in the definition of ``standard'' conditions
(regarding choice of the reference pressure and temperature), the
particular usage will be documented when this term is used.
\item \label{enu:ppmv}The International Bureau of Weights and Measures
recommends against use of units like percent or parts per million,
but these are in common use in atmospheric chemistry and elsewhere
so data files continue to use those units for relative humidity or
the concentration of chemical species. Typical units are ppmv\index{ppmv}
or ppbv\index{ppbv} for parts per million by volume or parts per
billion by volume. Care must be taken to interpret ppbv especially,
because ``billion'' has different meaning in different languages
and different countries; herein, 1 ppbv means a volumetric ratio of
1:10$^{9}$. Many measurements produce native results in terms of
a mass ratio, often described as a mixing ratio $r_{m}$ in terms
of mass of the measured gas per unit mass of ``air'' (where the
weight of the ``air'' does not include the variable constituents,
usually only significant for water vapor). The perfect gas law relates
the density ratio of two gases ($\rho_{1}:\rho_{2})$ to the ratio
of their partial pressures ($p_{1}:p_{2}$) or number densities ($n_{1}:n_{2}$),
as follows:\\
\[
r_{m}=\frac{\rho_{1}}{\rho_{2}}=\frac{p_{1}M_{1}}{p_{2}M_{2}}=\frac{n_{1}M_{1}}{n_{2}M_{2}}
\]
where $M_{1}$ and $M_{2}$ are respective molecular weights for the
two gases. The ratio of number densities or, equivalently, partial
pressures, denoted here as $r_{v}$ because it is also the volumetric
mixing ratio, is related to the mass mixing ratio as follows:\\
\[
r_{v}=\frac{n_{1}}{n_{2}}=\left(\frac{M_{2}}{M_{1}}\right)r_{m}
\]
\\
When concentrations are recorded with units of ``ppmv'', ``ppbv''
or ``pptv'', these units refer respectively to $10^{6}r_{v}$, $10^{9}r_{v}$,
or $10^{12}r_{v}$ with $r_{v}$ given by the above equation.\\

\item The unit ``hertz'' (abbreviation Hz) is the proper unit for a periodic
sampling frequency and will be used here in place of the more awkward
``samples per second.''\index{samples per second} This usage is
favored by the International Bureau of Weights and Measures (cf.~\url{http://www.bipm.org/en/si/si_brochure/chapter2/2-2/table3.html#notes})
when the frequency represented refers to the rate of sampling. 
\item In some cases, particularly for older data files, speed has been recorded
in units of knot\index{knot}s (= 0.514444 m/s) and distance in nautical
mile\index{nautical mile}s (= 1852 m).
\end{enumerate}
There is a list of symbols used in this report, on page \pageref{sub:List of Symbols}.%
\footnote{ Some symbols used only once and defined where they are used are omitted
from this list%
}. The next table defines some abbreviations and symbols used in this
report, in addition to the standard abbreviations for the mks system
of units:

\begin{center}
\begin{minipage}[t]{1\columnwidth}%
\begin{center}
\begin{tabular}{|c|l|}
\hline 
\textbf{abbreviation/symbol} & \textbf{definition}%
\footnote{where the symbol $\equiv$ is used, the relationship is exact by definition%
}\tabularnewline
\hline 
\hline 
� & degree, angle measurement $\equiv$ ($\pi$/180) radian\tabularnewline
\hline 
ft & foo\index{foo}t $\equiv$ 0.3048 m\tabularnewline
\hline 
mb & millibar $\equiv$ 100 Pa $\equiv$ 1 hPa\tabularnewline
\hline 
ppmv & parts per million by volume (see subsection \ref{sub:System-of-Units}
item \ref{enu:ppmv})\tabularnewline
\hline 
ppbv & parts per billion ($10^{9})$ by volume (see subsection \ref{sub:System-of-Units}
item\ref{enu:ppmv})\tabularnewline
\hline 
pptv & parts per trillion ($10^{12})$ by volume (see subsection \ref{sub:System-of-Units}
item\ref{enu:ppmv})\tabularnewline
\hline 
n~mi & nautical mile $\equiv$1852 m\tabularnewline
\hline 
kt & knot (n~mi/hour) $\equiv$ (1852/3600) m/s = 0.514444... m/s\tabularnewline
\hline 
\end{tabular}
\par\end{center}%
\end{minipage}
\par\end{center}


\subsection{Variables Used To Denote Time\index{Time Variables}}

Although there are some exceptions in old archived data files, the
data in all modern output files are referenced to Coordinated Universal
Time (UTC). The time and date of the data acquisition system are synchronized
to the Global Positioning System (GPS) at the beginning of each flight,
and for data acquired by the present ADS-3 (NIDAS) data acquisition
system time is synchronized continuously with the GPS time. Time variables
vary for older archived data files; some of the following are obsolete,
but are included here for reference because they are important to
those wanting to use those archives.\\

\begin{hangparagraphs}
\textbf{Time (s): }\textbf{\uline{Time}}\sindex[var]{Time}\index{Time}\label{han:Time-(s):-TimeThis-1}\\
This is the reference-time \emph{counter }for the output \index{time variable in data files}data
files, used by data system versions beginning with ADS-3. It is an
integer output at 1 sample per second \emph{and has an initial value
of zero at the start of the flight}. Add this to the ``Time:units''
attribute found in the NETCDF header section for this variable to
obtain the UTC measurement time.\\
 \\
\begin{minipage}[t]{0.9\textwidth}%
Example attribute: \texttt{\small }~\\
\texttt{\hspace*{1cm}}\texttt{\small Time:units = ``seconds since
2006-04-26 12:55:00 +0000'' };

For code examples that show how to use ``Time'' see:\\
\texttt{\hspace*{1cm}}\url{http://www.eol.ucar.edu/raf/Software/TimeExamp.html}%
\end{minipage}

~

\textbf{Reference Start Time (s): }\textbf{\uline{base\_time}}\index{base time}
(\emph{Obsolete; versions before ADS-3 only})\\
This is the reference time for the NETCDF output data files for data
system versions before ADS-3. It represents the time of the first
data record. Its format is Unix time (elapsed seconds after midnight
1 January 1970). Use this measurement and add time\_offset (below)
to obtain the time for each data record. (Note: base\_time is a single
scalar, not a ``record'' variable, so it occurs just once in the
output file.) 

\textbf{Time Offset from Reference Start Time (s): }\textbf{\uline{time\_offset}}\index{time offset}
(\emph{Obsolete)} \\
This is the time offset from base\_time of each data record used for
the NETCDF output files produced by data system versions before ADS-3.
It starts at zero (0) and increments each second, so it can also be
thought of as a record counter. Use this measurement and add base\_time
to obtain the time for each data record.

\textbf{Raw Tape Time (hour, minute, s): }\textbf{\uline{HOUR}}\textbf{\index{HOUR},
}\textbf{\uline{MINUTE}}\textbf{\index{MINUTE}, }\textbf{\uline{SECOND}}\textbf{\index{SECOND}
}(\emph{Obsolete})\\
These three time variables are recorded directly from the aircraft's
data system. Since ADS-3, this information is replaced by the ``Time''
variable and the ``Time:units'' attribute of that variable.

\textbf{Date (m, d, y): }\textbf{\uline{MONTH}}\textbf{\index{MONTH},
}\textbf{\uline{DAY}}\textbf{\index{DAY}, }\textbf{\uline{YEAR}}\emph{\index{YEAR}
}(\emph{Obsolete})\\
These three variables represent the date when the aircraft's data
system began recording data. They are repeated as 1 Hz variables but
are NOT incremented if the time rolls over to the next day. Use base\_time
and time\_offset for reference timing. Since ADS-3, this information
is replaced by the ``Time'' variable and the ``Time:units'' attribute
of that variable.


\end{hangparagraphs}

\subsection{Other Comments On Terminology}


\subsubsection{Variable Names In Equations}

This report often uses variable names in equations, and sometimes
there is potential for confusion because the variable names consist
of multiple characters. In many cases, to denote that the variable
name is the variable in the equation (as opposed to each of the letters
in the variable name representing quantities to be multiplied together),
the variable name has been enclosed in brackets, as in \{TASX\}. In
addition, variable names are displayed with upright Roman character
sets, while other nomenclature in equations is represented by slanted
(script) character sets as is conventional for mathematical equations.
In cases where code segments (usually expressed in C code) are included
to document how calculations are performed, typewriter character sets
indicate that the segment is a representation of how the processing
could be coded. Such a code segment is not always a direct copy of
the code in use, but such code is often the most convenient way to
express the algorithm in use.


\subsubsection{Distinction Between Original Measurements and Derived Measurements}

Many of the variables in the data files and in this report are derived
from combinations of measurements. The terms ``raw'' or ``original''
measurement\index{measurement!raw}\index{measurement!original}\index{measurement!derived}
are sometimes used for a minimally processed output received directly
from a sensor or instrument. Such measurements may be converted to
engineering units via calibration coefficients, but otherwise they
are a direct representation of the output from a sensor.%
\footnote{Calibration coefficients, e.g. those used to convert from voltage
output from an analog sensor to a measured quantity with physical
units like $\text{�}$C), are not included or discussed in this report
but are part of project reports and, in recent years, are included
in the header of the NETCDF file.%
} In contrast, derived variables\index{derived variables} (e.g., potential
temperature) depend on one or more ``raw'' measurements and are
not direct results of output from an instrument. For most derived
measurements, a box that follows an introductory comment is used in
this report to document the processing algorithm. The box has a line
dividing top from bottom; in the top are definitions used and explanations
regarding variables that enter the calculation, while the bottom portion
contains the equation, algorithm, or code segment that documents how
the variable is calculated.


\subsubsection{Dimensions in Equations}

An effort has been made to avoid dimensions\index{dimensionless equations}
in equations except where it would be awkward otherwise. Some scale
factors are introduced for only this purpose (e.g., to avoid dimensions
in arguments to logarithmic or exponential functions), and some effort
was made to isolate dimensions to defined constants rather than requiring
that variables in equations be used with specific units. However,
some exceptions remain to be consistent with historical usage.


\subsection{Constants Used in Data Processing}

The following box contains values used for some constants in this
document. For reference, the symbols used here also appear at the
start of the index along with their meanings and links to where they
are defined. Where references are to the ``NIST Chemistry WebBook'',
the associated URL is \url{http://webbook.nist.gov}. References to
the CODATA Internationally recommended values of the Fundamental Physical
Constants are available at \url{http://physics.nist.gov/.cuu/.Constants}.
The optimization involved in adjustment of these coefficients is documented
in Mohr et al., 2008a and 2008b, referenced at that URL.%
\footnote{P. J. Mohr, B. N. Taylor, and D. B. Newell, Rev. Mod. Phys 80(2),
633-730(2008); P. J. Mohr, B. N. Taylor, and D. B. Newell, J. Phys.
Chem. Ref. Data 37(3), 1187-1284(2008).%
}\\
\framebox{\begin{minipage}[t]{1.05\textwidth}%
\label{ConstantsBox}\centerline{\bf\underbar{Table of Constants}}\index{constants!table}\index{constants|see{symbols}}

$g$ = \sindex[lis]{g@$g$= acceleration of gravity}acceleration of
gravity%
\footnote{In the International Standard Atmosphere $g$ is specified as 9.80665
m\,s$^{-2}$, $M_{w}=28.9644$ , and $R_{0}$=8.31432\texttimes{}103
J kmol\textminus{}1 K\textminus{}1 %
} at latitude $\lambda$ \sindex[lis]{lambda
= latitude@$\lambda$= latitude}and altitude\sindex[lis]{z@$z$ = height} $z$ above the WRS-80 \index{WRS-80 Geoid}geoid,%
\footnote{cf., e.g., Moritz, H., 1988: Geodetic Reference System 1980, Bulletin
Geodesique, Vol. 62 , no. 3. %
} \\
\begin{equation}
g_{\lambda}=g_{e}\left(\frac{1+g_{1}\sin^{2}(\lambda)}{(1-g_{2}\sin^{2}\lambda)^{1/2}}\right)-a_{z}z\label{eq:g_lambda}
\end{equation}
\hskip2em%
\parbox[t]{0.95\textwidth}{%
where $g_{e}=9.780327$\,m\,s$^{-2}$ is the reference value at
the surface and equator,\\
~~~~~~~~~~$g_{1}=0.001931851$\, $g_{2}=0.006694380$,
and $a_{z}=3.086\times10^{-6}$\,m$^{-1}$.%
} \\
$T_{0}$ = temperature in kelvin\sindex[lis]{T_{0}
= 273.15 K@$T_{0}$= 273.15\,K} corresponding to $0{}^{\circ}C$ = 273.15\,K\\
$T_{3}$ = temperature\sindex[lis]{T_{3}
= 273.16 K@$T_{3}$= 273.16\,K} corresponding to the triple point of water\index{triple point of water}
= 273.16\,K\\
$M_{a}$= \sindex[lis]{Md@$M_{d}$= molecular weight of dry air}molecular
weight of dry air$^{a}$\index{molecular weight of dry air}, 28.9637
kg\,kmol$^{-1}$~~~%
\footnote{Jones, F. E., 1978: J. Res. Natl. Bur. Stand., 83(5), 419, as quoted
by Lemmon, E. W., R. T. Jacobsen, S. G. Penoncello, and D. G., Friend,
J. Phys. Chem. Ref. Data, Vol. 29, No. 3, 2000, pp. 331-385. The quoted
values of mole fraction from Jones (1978) and the calculation of mean
molecular weight are tabulated below using values of molecular weights
taken from the NIST Standard Reference Database 69: NIST Chemistry
WebBook as of March 2011. However, CO$_{2}$ should be increased to
about 0.00039 and others decreased proportionately, leading to a mean
of 28.9637.

\begin{center}
\begin{tabular}{|c|c|c|c|}
\hline 
Gas & mole fraction $x$ & molecular weight $M$ & $x*M$\tabularnewline
\hline 
\hline 
N$_{2}$ & 0.78102 & 28.01340 & 21.87903\tabularnewline
\hline 
O$_{2}$ & 0.20946 & 31.99880 & 6.70247\tabularnewline
\hline 
Ar & 0.00916 & 39.94800 & 0.36592\tabularnewline
\hline 
CO$_{2}$ & 0.00033 & 44.00950 & 0.01452\tabularnewline
\hline 
Mean: &  &  & 28.96194\tabularnewline
\hline 
\end{tabular}
\par\end{center}%
}\\
$M_{w}$ = \sindex[lis]{Mw@$M_{w}$= molecular weight of water}molecular
weight of water\index{molecular weight of water}, 18.0153 kg\,kmol$^{-1}$~~~%
\footnote{NIST Standard Reference Database 69: NIST Chemistry WebBook as of
March 2011%
}\\
$R_{0}$ = \sindex[lis]{R@$R_{0}$= universal gas constant}universal
gas constant$^{a}$\index{universal gas constant} = 8.314472$\times10^{3}$
J\,kmol$^{-1}$K$^{-1}$~~~%
\footnote{2006 CODATA%
}\\
$R_{d}=(R_{0}/M_{d}$) = \sindex[lis]{Rd@$R_{d}=$gas constant for dry air}gas
constant for dry air\index{gas constant for dry air}\\
$R_{w}$ = ($R_{0}/M_{w})$ = \sindex[lis]{Rw@$R_{W}$= gas constant for water vapor}gas
constant for water vapor\index{gas constant for water vapor}\\
$R_{E}$ = \sindex[lis]{Re@$R_{E}$= radius of the Earth}radius of
the Earth\index{radius of the Earth} = 6.371229$\times$10$^{6}$
m ~~~%
\footnote{matching the value used by the inertial reference systems discussed
in \prettyref{sec:INS}%
} \\
$c{}_{p}$ = \sindex[lis]{cp@$c_{p}$= specific head of dry air at constant pressure}specific
heat of dry air at constant pressure\index{specific heat of dry air at constant pressure}
= $\frac{7}{2}R_{d}=$1.00473$\times10^{3}$ J$\,$kg$^{-1}$K$^{-1}$~~~%
\footnote{The specific heat of dry air at 1013 hPa and 250--280 K as given by
Lemmon et al. (2000) is 29.13 J/(mol-K), which translates to 1005.8$\pm0.3$
J/(kg-K). However, the uncertainty limit associated with values of
specific heat is quoted as 1\%, and the experimental data cited in
that paper show scatter that is at least comparable to several tenths
percent, so the ideal-gas value cited here is well within the range
of uncertainty. For this reason, and because this value is in widespread
use, the ideal-gas value is used throughout the algorithms described
here.%
}\\
~~~~~~~~~~(value at 0$^{\circ}C$; small variations with
temperature are not included here)\\
$c{}_{v}$ = \sindex[lis]{cv@$c_{v}$= specific heat of dry air at constant volume}specific
heat at constant volume\index{specific heat of dry air at constant volume}
= $\frac{5}{2}R_{d}=$0.71766$\times10^{3}$ J$\,$kg$^{-1}$K$^{-1}$\\
~~~~~~~~~~(value at 0$^{\circ}C$; small variations with
temperature are not included here)\\
$\gamma$ = \sindex[lis]{gamma
= ratio of specific heats of air, c_{p}/c_{v}
@$\gamma$= ratio of specific heats of air, $c_{p}/c_{v}$}ratio of specific heats\index{specific heat ratio, dry air}, $c_{p}/c_{v}$,
taken to be 1.4 (dimensionless) for dry air\\
$\Omega$ = \sindex[lis]{Omega
= angular rotation rate of the Earth@$\Omega$= angular rotation rate of the Earth}angular rotation rate of the Earth\index{Earth, angular rotation rate}
= 7.292115$\times10^{-5}$ radians/s\\
$\Omega_{Sch}$ = \sindex[lis]{Omega_{Sch}
= angular frequency of the Schuler oscillation@$\Omega_{Sch}$= angular frequency of the Schuler oscillation}angular frequency of the Schuler oscillation\index{Schuler oscillation}
= $\sqrt{\frac{g}{R_{E}}}$\\
$\sigma$ = \sindex[lis]{sigma
= Stephan-Boltzmann constant@$\sigma$= Stephan-Boltzmann constant}Stephan-Boltzmann Constant\index{Stephan-Boltzmann Constant} = 5.6704$\times10^{-8}$W\,m$^{-2}\mathrm{K}^{-4}$~~~%
\footnote{2006 CODATA%
} %
\end{minipage}}

\label{sub:List of Symbols}\index{symbols!table of}\printindex[lis]{}

\label{List of Variables}\index{variable names!table of}\printindex[var]{}

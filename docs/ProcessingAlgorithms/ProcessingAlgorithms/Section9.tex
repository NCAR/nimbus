
\section{WIND\label{sec:WIND}}

\href{http://www.eol.ucar.edu/raf/Bulletins/bulletin23.html}{RAF Bulletin 23}\index{Bulletin 23}
documents the calculation of wind components, both with respect to
the earth (UI, VI, WI, WS and WD) and with respect to the aircraft
(UX and VY). In data processing, a separate function (GUSTO in GENPRO,
gust in NIMBUS) is used to derive these wind components. That function
uses the Inertial Reference Unit (IRU\index{IRU}) measurements as
well as aircraft true airspeed, aircraft angle of attack, and aircraft
sideslip angle. The wind components calculated in GUSTO/gust are used
to derive the wind direction (WD) and wind speed (WS). Additional
variables UIC, VIC, WSC, WDC, UXC, and VYC are also calculated based
on the variables VNSC, VEWC discussed in section \ref{sub:IRS/GPS},
which combine IRU and GPS information to obtain improved measurements
of the aircraft motion that combine the high-frequency response of
the IRU measurements with the long-term accuracy of the GPS measurements.

The details contained in Bulletin 23 will not be repeated here; the
description there still describes the wind calculations in use, so
please consult that bulletin for the processing algorithms. The variables
pertaining to the relative wind are described in the next subsection,
and the variables characterizing the wind are then described briefly
in the last subsection. Some additional detail is included in cases
where procedures are not documented in that earlier bulletin.


\subsection{Relative Wind}

Wind\index{wind!relative} is measured by adding two vectors: (1)
the measured air motion relative to the aircraft (called the relative
wind), and (2) the motion of the aircraft relative to the Earth. The
following are the measurements used to determine the relative wind.
The motion of the aircraft relative to the ground is discussed in
the next section, and the combination of these two vectors to measure
the wind is described in \href{http://www.eol.ucar.edu/raf/Bulletins/bulletin23.html}{RAF Bulletin 23}.
The radome gust-sensing system is discussed briefly \vpageref{radome gust-sensing system}.
\begin{hangparagraphs}
\textbf{Dynamic Pressure (mb): }\textbf{\uline{QCx}}\textbf{\sindex[var]{QCx}\index{QCx},
}\textbf{\uline{QCxC}}\sindex[var]{QCxC}\index{QCxC}\\
Dynamic pressure\index{pressure!dynamic} is the difference between
the pitot (total) pressure\index{pressure!total}\index{pressure!pitot|see {pressure, total}}
resulting from the flight speed of the aircraft and the static pressure\index{pressure!ambient}\index{pressure!static|see {pressure ambient}}
that would be present in the absence of motion. The QCxC value is
corrected for local flow-field distortion, as described in\href{http://www.eol.ucar.edu/raf/Bulletins/bulletin21.html}{RAF Bulletin 21}\index{Bulletin 21}.
A Rosemount Model 1221 differential pressure transducer is used for
all measurements of dynamic pressure. This measurement enters the
calculation of true airspeed\index{true airspeed} and Mach number\index{Mach number}
and so is needed to calculate many derived variables. 

\textbf{Aircraft True Airspeed (m/s): }\textbf{\uline{TASx}}\index{TASx}\\
This derived measurement of the flight speed\index{flight speed|see{true airspeed}}
of the aircraft relative to the atmosphere is based on the Mach number
calculated from both the dynamic pressure at location x and the static
pressure\index{pressure!ambient}. See the derivation for ATx \vpageref{ambient temperature and TAS calculation}.
The different variables for TASx (TASF, TASR, etc) use different measurements
of QCxC in the calculation of Mach number.\\
\\
\framebox{\begin{minipage}[t]{0.95\textwidth}%
(see box for ATx\index{ATx} and MACHx\index{MACHx})\\
Note dependence of MACHx on choices for QCxC\index{QCxC} and PSXC\index{PSXC}\\
TASx\index{TASx} depends on QCxC, PSXC\index{PSXC}, ATX\index{ATX}\\
~~~~~where PSXC and ATX are the preferred choices\\
\\
\\
\rule[0.5ex]{1\columnwidth}{1pt}

\begin{equation}
\mathrm{TASx}=\mathrm{\{MACHx\}}\sqrt{\gamma R_{d}\mathrm{\,(\{ATX\}}+T_{0})}\label{eq:8.10-1}
\end{equation}
%
\end{minipage}}\\
\\


\textbf{Aircraft True Airspeed (Humidity Corrected) (m/s): }\textbf{\uline{TASHC}}\sindex[var]{TASHC}\index{TASHC}\\
 This derived measurement of \index{true airspeed}true airspeed takes
into account the deviations of the specific heats of moist air from
those of dry air. See List, 1971, pp 295, 331-339, and Khelif, et
al., 1999. The present form of this equation, from Khelif et al.~1999%
\footnote{Khelif, D., S.P. Burns, and C.A. Friehe, 1999: Improved wind measurements
on research aircraft. \emph{Journal of Atmospheric and Oceanic Technology,}
\textbf{16,} 860--875.%
}, adds a moisture correction to a true airspeed derived for dry air:\\
\\
 %
\framebox{\begin{minipage}[t]{0.95\columnwidth}%
SPHUM\index{SPHUM} = specific humidity\index{humidity!specific},
g/kg\\
\\
\\
\rule[0.5ex]{1\columnwidth}{1pt}
\begin{lyxcode}
TASHC~=~TASX~{*}~(1.0~+~0.000304~{*}~SPHUM)\end{lyxcode}
%
\end{minipage}}\\


\textbf{Attack Angle Differential Pressure (mb): }\textbf{\uline{ADIFR}}\sindex[var]{ADIFR}%
\footnote{\begin{hangparagraphs}
\textbf{Obsolete variable }\textbf{\uline{ADIF}}\textbf{\index{ADIF}
is a similar variable used for old gust-boom systems or for Rosemount
Model 858AJ flow-angle sensors.}\end{hangparagraphs}
%
}\index{ADIFR}\\
\index{attack, angle of}This variable records the pressure difference
between the top and bottom pressure ports of the \index{radome gust-sensing system}radome
gust-sensing system. 

\textbf{Sideslip Angle Differential Pressure (mb): }\textbf{\uline{BDIFR}}\sindex[var]{BDIFR}%
\footnote{\begin{hangparagraphs}
\textbf{Obsolete variable }\textbf{\uline{BDIF}}\textbf{\index{BDIF}
is a similar variable used for old gust-boom systems or for Rosemount
Model 858AJ flow-angle sensors.}\end{hangparagraphs}
%
}\index{BDIFR}\\
\index{sideslip, angle of}This variable records the pressure difference
between starboard and port pressure inlets of the radome gust-sensing
system\index{radome gust-sensing system}, in the horizontal plane
of the aircraft's flow-angle sensor system. 

\textbf{Attack Angle, Radome (}$\text{�}$\textbf{): }\textbf{\uline{AKRD}}\sindex[var]{AKRD}\index{AKRD}\\
This derived output represents the angle of attack\index{attack, angle of}
of the aircraft. It is obtained from ADIFR\index{ADIFR} and a dynamic
pressure\index{pressure!dynamic} using a sensitivity function that
has been determined empirically for each aircraft.\\
\framebox{\begin{minipage}[t]{0.95\textwidth}%
ADIFR\index{ADIFR} = attack differential pressure, radome\\
QCXC\index{QCXC} = reference dynamic pressure\\
XMACH2\index{XMACH2} = square of the Mach number\\
$c_{0},\, c_{1}$ = sensitivity coefficients determined empirically\\
\-\-\-\-\-\-\-\-\- = \{0.3843, (1/0.06653) $\text{�}$\}
for the C-130\\
~~~~~~~~~ = \{0.2571, (1/0.04724) $\text{�}$\} for the GV\\
$c_{3},\, c_{4}$ = additional coefficients for the GV \\
~~~~~~~~~ = \{0.6195$\text{�}$, -1.02758$\text{�}$\} for
XMACH2 > 0.194\\
~~~~~~~~~ = \{0.42$\text{�}$, 0\} for XMACH2 $\leq$ 0.194

\noindent \rule[0.5ex]{1\linewidth}{1pt}

For the C-130:

\[
\mathrm{AKRD}=c_{1}\left(\frac{\{\mathrm{ADIFR}\}}{\{\mathrm{QCXC}\}}+c_{0}\right)
\]


For the GV:

\[
\mathrm{AKRD}=c_{1}\left(\frac{\{\mathrm{ADIFR}\}}{\{\mathrm{QCXC}\}}+c_{0}\right)+c_{3}+c_{4}\{\mathrm{XMACH2}\}
\]
%
\end{minipage}}

\textbf{}%
\begin{comment}
Old Bulletin-9 code: omitted now

\textbf{Attack Angle, 858 (}$\text{�}$\textbf{): }\textbf{\uline{AKDF}}\index{AKDF}\\
This is a derived output of the aircraft's angle of attack obtained
from ADIF (the vertical differential pressure measured by a Rosemount
858AJ flow-angle sensor) and a dynamic pressure (Qc) using an empirically
determined sensitivity function. For the Rosemount 858AJ, the sensitivity
function (GR) is a constant 0.079 for Mach numbers less than 0.515.
For larger Mach numbers the sensitivity decreases as a function of
Mach number.\\
\framebox{\begin{minipage}[t]{0.9\textwidth}%
MACH = Aircraft Mach Number\\
GR = sensitivity function (given below)\\
Qc = dynamic pressure, mb

\begin{eqnarray*}
\mathrm{for\, MACH} & \leq & 0.515:\\
 &  & \mathrm{GR}=0.079\\
\mathrm{otherwise}:\\
 & \mathrm{GR} & =0.086577797-0.03560256\, MACH\\
 &  & +0.00006143\, MACH^{2}
\end{eqnarray*}


\noindent \rule[[0.5ex]]{1.0\linewidth}{1pt}

\noindent 
\[
\mathrm{AKDF=\frac{ADIF}{GR\, Qc}}
\]
%
\end{minipage}}
\end{comment}


\textbf{Reference Attack Angle (}$\text{�}$\textbf{): }\textbf{\uline{ATTACK}}\sindex[var]{ATTACK}\index{ATTACK}\\
This variable is the reference selected from other measurements of
angle of attack in the data set. In most projects, it is equal to
AKRD\index{AKRD}. It is used where attack angle is needed for other
derived calculations (e.g., wind measurements).

\textbf{Sideslip Angle (Differential Pressure) (}$\text{�}$\textbf{):
}\textbf{\uline{SSRD}}\sindex[var]{SSRD}\index{SSRD}\\
This variable is derived from BDIFR\index{BDIFR} and a dynamic pressure\index{pressure!dynamic}
using a sensitivity function that has been determined empirically
for each aircraft. \\
\framebox{\begin{minipage}[t]{0.9\textwidth}%
BDIFR = differential pressure between sideslip pressure ports, radome
(mb)\\
QCXC = dynamic pressure (mb)\\
$b_{0},\, b{}_{1}$ = empirical coefficients dependent on the aircraft
and radome configuration\\
~~~~~~~~~~ = \{-0.000983, (1/0.09189)$\text{�}$ \} for
the C-130\\
\_\_\_\_\_\_ = \{-0.0023, (1/0.04727)$\text{�}$\} for the GV

\noindent \rule[0.5ex]{1\linewidth}{1pt}

\[
\mathrm{SSRD=}b_{1}(\frac{\{BDIFR\}}{\{\mathrm{QCXC}\}}-b_{0})
\]
%
\end{minipage}}

\textbf{}%
\begin{comment}
The following is from old Bulletin 9:

\textbf{Sideslip Angle (Differential Pressure) (}$\text{�}$\textbf{):
}\textbf{\uline{SSDF}}\index{SSDF}\\
This variable is derived from BDIF and a dynamic pressure using an
empirically determined sensitivity function, as expressed in the following:\\
\framebox{\begin{minipage}[t]{0.9\textwidth}%
$M$ = Aircraft Mach Number\\
BDIF = differential pressure between sideslip pressure ports, 858
probe (mb)\\
QC = dynamic pressure from the 858 probe (mb)\\
$g_{r}$ = dimensionless sensitivity function (given below)
\begin{eqnarray*}
\mathrm{for\, M} & \leq & 0.515:\\
 & g_{r} & =0.079\\
\mathrm{otherwise}:\\
 & g_{r} & =0.086577797-0.03560256\, M\\
 &  & +0.00006143\, M^{2}
\end{eqnarray*}


\noindent \rule[0.5ex]{1\linewidth}{1pt}

\[
\mathrm{SSDF=}\frac{\mathrm{\{BDIF\}}}{g_{r}\mathrm{\{QC\}}}
\]
%
\end{minipage}}
\end{comment}


\textbf{Reference Sideslip Angle (}$\text{�}$\textbf{): }\textbf{\uline{SSLIP}}\sindex[var]{SSLIP}\index{SSLIP}\\
This variable is the reference selected from other measurements of
sideslip angle\index{sideslip, angle of} in the data set. In most
projects, it is equal to SSRD\index{SSRD}. It is used where sideslip
angle is needed for other derived calculations (e.g., wind measurements). 

\textbf{}
\end{hangparagraphs}

\subsection{Wind Components and the Wind Vector}
\begin{hangparagraphs}
\textbf{}%
\begin{minipage}[t]{1\columnwidth}%
\begin{hangparagraphs}
\textbf{Wind Vector East Component (m/s): }\textbf{\uline{UI\sindex[var]{UI}\index{UI}}}

\textbf{Wind Vector North Component (m/s): }\textbf{\uline{VI\sindex[var]{VI}\index{VI}}}

\textbf{Wind Vector Vertical Gust Component (m/s): }\textbf{\uline{WI\sindex[var]{WI}\index{WI}}}\end{hangparagraphs}
%
\end{minipage}\textbf{\uline{}}\\
\textbf{\uline{}}\\
These measurements comprise the three-dimensional wind\index{wind}\index{wind!vector}\index{wind!components}
vector with respect to the earth. UI is the east-west component with
positive values toward the east, VI is the north-south component with
positive values toward the north, and WI is the vertical component
with positive values toward the zenith.\\


\textbf{}%
\begin{minipage}[t]{1\columnwidth}%
\begin{hangparagraphs}
\textbf{Wind Speed (m/s): WS\sindex[var]{WS}\index{WS}}

\textbf{Wind Direction (}$\text{�}$\textbf{): WD}\sindex[var]{WD}\index{WD}\end{hangparagraphs}
%
\end{minipage}\\
\\
These variables\index{wind!speed}\index{wind!direction} are obtained
in a straightforward manner from UI and VI. The resulting wind direction
is relative to true north.\\
\\
\framebox{\begin{minipage}[t]{0.95\textwidth}%
UI = easterly component of the horizontal wind\\
VI = northerly component of the horizontal wind\\
atan2 = 4-quadrant arc-tangent function producing output in radians
from -$\pi$ to $\pi$\\
$C_{rd}$\sindex[lis]{Crd@$C_{rd}$= conversion factor, radians to degrees (180/$\pi$)}
= conversion factor, radians to degrees, = 180/$\pi$ {[}units: $^{\circ}/radian${]}\\
\\
\rule[0.5ex]{1\linewidth}{1pt} 
\begin{eqnarray}
\mathrm{WS} & = & \sqrt{\mathrm{\{UI\}}^{2}+\{\mathrm{VI\}}^{2}}\nonumber \\
\mathrm{WD} & = & C_{rd}\mathrm{\, atan2(\{UI\},\,\{VI\})}+180^{\circ}\label{eq:9.1}
\end{eqnarray}
%
\end{minipage}}\\


\textbf{}%
\begin{minipage}[t]{1\columnwidth}%
\begin{hangparagraphs}
\textbf{Wind Vector Longitudinal Component (m/s): }\textbf{\uline{UX\sindex[var]{UX}\index{UX}}}

\textbf{Wind Vector Lateral Component (m/s): }\textbf{\uline{VY}}\sindex[var]{VY}\index{VY}\end{hangparagraphs}
%
\end{minipage}\textbf{}\\
\textbf{}\\
These measurements represent the horizontal wind\index{wind!vector}
vector relative to the frame of reference attached to the aircraft
and moving with the aircraft. UX is parallel to the longitudinal axis,
and positive is toward the nose. VY is the component normal to the
longitudinal axis; positive is toward the port (or left) wing.\\


\textbf{}%
\begin{minipage}[t]{1\columnwidth}%
\begin{hangparagraphs}
\textbf{GPS-Corrected Wind Vector, East Component (m/s): }\textbf{\uline{UIC\sindex[var]{UIC}\index{UIC}}}

\textbf{GPS-Corrected Wind Vector, North Component (m/s): }\textbf{\uline{VIC}}\sindex[var]{VIC}\index{VIC}\end{hangparagraphs}
%
\end{minipage}\\
\\
These measurements\index{wind!GPS-corrected} give the horizontal
wind\index{wind!vector} components respectively toward the east and
toward the north. They are derived from measurements from an inertial
reference unit (IRU) and a Global Positioning System (GPS), as described
in the discussion of VEW and VNS above. They are calculated just as
for UX and VY except that the GPS-corrected values for the aircraft
groundspeed are used in place of the IRU-based values. They are considered
``corrected'' from the original measurements from the IRU or GPS,
as described in section \ref{sub:IRS/GPS}.\\


\textbf{Wind Vector, Vertical Component (m/s): }\textbf{\uline{WIC}}\sindex[var]{WIC}\index{WIC}\\
\\
\marginpar{change 2011}This is the standard calculation of vertical
wind\index{wind!vertical}.\textbf{}%
\footnote{This variable is named ``GPS-Corrected Wind Vector'' in some output
prior to 2011, but that name was incorrect because the algorithm does
not involve measurements from a GPS.%
} It is based on measurements from the IRU, including the measurement
VSPD that is the direct measurement of vertical speed of the aircraft
as discussed in section \ref{sec:INS}. This should be used in preference
to WI if the latter is present; see the discussion of WP3 in section
\ref{sec:INS}. Positive values are toward the zenith.\\


\textbf{}%
\begin{minipage}[t]{1\columnwidth}%
\begin{hangparagraphs}
\textbf{GPS-Corrected Wind Speed (m/s): }\textbf{\uline{WSC\sindex[var]{WSC}}}\index{WSC}

\textbf{GPS-Corrected Wind Direction (}$\text{�}$\textbf{): }\textbf{\uline{WDC}}\sindex[var]{WDC}\index{WDC}\end{hangparagraphs}
%
\end{minipage}\\
\\
These variables\index{wind!GPS-corrected} are obtained in a straightforward
manner from UIC and VIC, using equations analogous to (\ref{eq:9.1})
but with UIC and VIC as input measurements. They are expected to be
the best measurements of wind because they combine the best features
of the IRU and GPS measurements.\\


\textbf{}%
\begin{minipage}[t]{1\columnwidth}%
\begin{hangparagraphs}
\textbf{GPS-Corrected Wind Vector, Longitudinal Component (m/s): }\textbf{\uline{UXC\sindex[var]{UXC}\index{UXC}}}

\textbf{GPS-Corrected Wind Vector, Lateral Component (m/s): }\textbf{\uline{VYC}}\sindex[var]{VYC}\index{VYC}\end{hangparagraphs}
%
\end{minipage}\\
\\
These measurements give the longitudinal and lateral components of
the three-dimensional wind\index{wind!lateral component}\index{wind!longitudinal component},
similar to UX and VY, but corrected by the complementary-filter algorithm
that combines IRU and GPS measurements, as discussed in Section \ref{sub:IRS/GPS}.
The components UXC and VYC are toward the front of the aircraft and
toward the port (left) wing, respectively. \end{hangparagraphs}

